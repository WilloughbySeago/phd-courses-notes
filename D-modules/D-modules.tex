% !TeX program = lualatex
\documentclass[fleqn]{NotesClass}

\strictpagecheck

\usepackage{csquotes}

\usepackage{tikz}
\usetikzlibrary{external}
\tikzexternalize[prefix=tikz-external/]

\usepackage{tikz-cd}
\AtBeginEnvironment{tikzcd}{\tikzexternaldisable}
\AtEndEnvironment{tikzcd}{\tikzexternalenable}

\usepackage[pdfauthor={Willoughby Seago},pdftitle={Notes from D-modules Course},pdfkeywords={D-module},pdfsubject={D-modules}]{hyperref}  % Should be loaded second last (cleveref last)
\colorlet{hyperrefcolor}{blue!60!black}
\hypersetup{colorlinks=true, linkcolor=hyperrefcolor, urlcolor=hyperrefcolor}
\usepackage[
capitalize,
nameinlink,
noabbrev
]{cleveref} % Should be loaded last

% My packages
\usepackage{NotesBoxes}
\usepackage{NotesMaths2}

\setmathfont[range={\int, \oint, \otimes, \oplus, \bigotimes, \bigoplus}]{Latin Modern Math}


% Highlight colour
\definecolor{my blue}{HTML}{084887}
\definecolor{my red}{HTML}{CA1551}
\definecolor{my green}{HTML}{17C3B2}
\definecolor{my yellow}{HTML}{F58A07}
\definecolor{my purple}{HTML}{CB9CF2}
\colorlet{highlight}{my blue}

% Title page info
\title{Algebraic Geometry}
\author{Willoughby Seago}
\date{October 6th, 2025}
\subtitle{Notes from}
\subsubtitle{University of Glasgow}
\renewcommand{\abstracttext}{These are my notes from the SMSTC course \emph{\(\D\)-modules} taught by Prof Gwyn Bellamy. This is a subject for which I don't have all of the prerequisites, so I'm sure these notes will be full of errors, please let me know if you find any. These notes were last updated at \printtime{} on \today{}.}

% Commands
% Maths
\newcommand{\isomorphic}{\cong}
\newcommand{\equivalent}{\approxeq}
\newcommand{\D}{\symcal{D}}
\newcommand{\sheaf}[1]{\symcal{#1}}
\DeclareMathOperator{\Hom}{Hom}
\DeclareMathOperator{\End}{End}
\newcommand{\bounded}{\symrm{b}}
\newcommand{\reghol}{\symrm{rh}}
\newcommand{\constructable}{\symrm{cs}}
\DeclarePairedDelimiterX{\bracket}[2]{[}{]}{#1 , #2}
\newcommand{\affine}{\symbb{A}}

\includeonly{}

\begin{document}
    \frontmatter
    \titlepage
    \innertitlepage{}
    \tableofcontents
    % \listoffigures
    \mainmatter
    
    \chapter{Introduction}
    These are my notes from the SMSTC course on \(\D\)-modules.
    I am taking the SMSTC algebraic geometry course at the same time, so I'm lacking some prerequisites for this course.
    Therefore I'm sure there are many errors in these notes, which are my fault, please tell me if you find any.
    I will assume the reader is up to the same part of the algebraic geometry course as I am, so for example for the first lecture of \(\D\)-modules I knew what an affine variety was, because that was covered in the first algebraic geometry lecture the day before.
    
    We will work entirely over \(\complex\), although many results generalise to other fields with sufficient adjectives (algebraically closed, characteristic zero, etc.).
    
    \section{Why Care About \texorpdfstring{\(\D\)}{D}-Modules?}
    \subsection{What (Roughly) are \texorpdfstring{\(\D\)}{D}-modules?}
    Differential functions, such as \(D_1 = \partial_1 \coloneq \difsp{}{x_1}\) and \(D_2 = x_1 - 3\partial_1\partial_2\), act on (sufficiently smooth) functions \(\complex^n \to \complex\) where \(n\) is the number of variables.
    
    We can compose two differential operators, and we can add them.
    This makes the set, \(\D(\complex^n)\), of all differential operators on such functions into a ring.
    Then a \(\D\)-module is just a module over this ring.
    
    Of course, there are more generic spaces of functions and function-like things than just functions \(\complex^n \to \complex\), and so there are more general \(\D\)-modules.
    
    A key thing to notice is that \(D_1 \circ D_2 \ne D_2 \circ D_1\), and so \(\D(\complex^n)\) is non-commutative, which makes doing algebraic geometry with it harder.
    
    \subsection{Why Where \texorpdfstring{\(\D\)}{D}-Modules Invented?}
    In the 1970s Mikio Sato, and his School\footnote{The Sato School, based in Kyoto, includes mathematicians such as Kashiwara, Jimbo, and Miwa.}
    His original goal was to study systems of linear differential equations, introducing \(\D\)-modules as a way of bringing the powerful tools of algebra and geometry to this analysis problem.
    In doing so he invented the field of \emph{algebraic analysis}.
    
    Around the same time Joseph Bernstein came up with a more algebraic version of this theory in order to prove a conjecture on the meromorphic extensions of certain holomorphic functions.
    
    We won't go into details on the problems they were trying to solve, instead we'll look at a simple example.
    Consider the differential operators
    \begin{equation}
        P_1 = x_1 + \partial_2, \quad P_2 = x_2^2 \partial_1 - 4, \qand P_3 = x_1^2 \partial_1 - x_2 \partial_2^2.
    \end{equation}
    Our goal is to study functions, \(f\), satisfying the linear system \(P_1(f) = P_2(f) = P_3(f) = 0\).
    The question we'll ask is what is the dimension of \(S(P_{\bullet})\), the solution space to this system of equations?
    
    Notice that we're fairly vague about what we mean by \enquote{function}, it might be a polynomial, or holomorphic function, or even a distribution.
    We're looking for general statements about solutions depending only on the \(P_i\) and not the specific types of functions we consider.
    
    Let \(\sheaf{O}\) be the space of functions under consideration.
    Define \(M = \D(\complex^2) / (P_1, P_2, P_3)\).
    Then we have the following result.
    
    \begin{lma}{}{}
        \(\Hom_{\D(\complex^2)}(M, \sheaf{O}) = S(P_{\bullet})\).
        \begin{proof}
            As a \(\D(\complex^2)\)-module \(M\) is generated by \(1\), since \(M\) is just \(\D(\complex^2)\) after setting \(P_i = 0\).
            Then we have a map
            \begin{align}
                \Hom_{\D(\complex^2)}(M, \sheaf{O}) &\to S(P_{\bullet})\\
                \Phi &\mapsto \Phi(1).
            \end{align}
            Since \(S(P_{\bullet})\) is exactly the subspace of \(\D(\complex^2)\) on which the \(P_i\) act as zero this is an isomorphism.
        \end{proof}
    \end{lma}
    
    The right-hand-side of this equality can be viewed in a purely analytic way, as the solution space to some differential equations.
    The left-hand-side can be viewed in a purely algebraic way, as a quotient of some module.
    To do so we introduce new variables, \(y_i\), which we replace the derivatives with, imposing the required commutation relations from the derivatives.
    To make further process we introduce geometry to our toolkit.
    
    We can approximate \(M\) by taking the \(P_i\) and considering their leading terms, where we introduce a grading \(\deg x_i = \deg y_i = 1\).
    Then we have \(Q_1 = x_1 + y_2\), \(Q_2 = x_2^2 \partial_1\), and \(Q_3 = x_1^2y_1 - x_2y_2^2\).
    Writing \(V(Q_1, Q_2, Q_3)\) for the set of common zeros of the \(Q_i\) it can be shown that
    \begin{equation}
        \dim V(Q_1, Q_2, Q_3) = 1.
    \end{equation}
    
    A result linking this variety to symplectic geometry implies that \(M = 0\), since in particular we must have that the \enquote{characteristic variety} of \(M\) has even dimension, and \(M = 0\) is the only way that we can have \(\dim V(Q_1, Q_2, Q_3) = 1\).
    
    If \(M = 0\) then \(\Hom_{\D(\complex^2)}(M, \sheaf{O}) = 0\) so \(S(P_{\bullet}) = 0\).
    That is, there are no non-trivial solutions to our system of linear differential equations.
    
    In other words, if \(M = 0\) then it must be that \((P_1, P_2, P_3) = \D(\complex^2)\), and so it must be that \(1 \in (P_1, P_2, P_3)\).
    That is, there are differential operators, \(D_i \in \D(\complex^2)\) such that \(D_1P_1 + D_2P_2 + D_3P_3 = 1\).
    Then if we're to have \(P_i = 0\) we must have that
    \begin{equation}
        0 = D_1P_1f + D_2P_2f + D_3P_3f = f
    \end{equation}
    and so we have only the trivial solution.
    When stated in this way it is clear that it doesn't matter what \enquote{type} of function (or function-like-thing) \(f\) is.
    
    \subsection{Why Should You Care About \texorpdfstring{\(\D\)}{D}-Modules?}
    This section is mostly a list of buzz-words you may or may not care about.
    
    Rings of differential operators on smooth varieties provide an interesting class of noncommutative rings.
    
    Geometric representation theory makes use of \(D\)-modules to study representations of Lie algebras, Hecke algebras, doubly affine Hecke algebras, quantum-things, and more.
    An important example is the proof of the Kazhdan--Lustzig conjecture, relating characters of certain representations of certain Lie algebras to the transition coefficients between different bases of Hecke algebras.
    
    There are also applications in algebraic geometry, in particular to the theory of mixed Hodge modules.
    
    \section{The Riemann--Hilbert Correspondence}
    The main goal of this course is to develop enough understanding to state and sketch a proof of the Riemann--Hilbert correspondence.
    This can be viewed as the use of analysis to form a bridge between \(\D\)-modules and topology.
    
    I now repeat exactly what is in the notes, because I don't understand it yet.
    The solution functor, \(M \to \Hom_{\D}(M, \sheaf{O})\), gives an equivalence of bounded derived categories,
    \begin{equation}
        D^{\bounded}_{\reghol}(\D_X) \equivalent D^{\bounded}_{\constructable}(\complex_X).
    \end{equation}
    The left-hand-side here is the bounded derived category of \(\D\)-modules with regular holonomic cohomology and the right-hand-side is the derived category of constructable sheaves.
    
    The goal is to understand what these words mean.
    
    \section{Prerequisites}
    Some knowledge of sheaves is useful, but we'll avoid using them too much until they're unavoidable, at which point we'll black-box some results from sheaf theory.
    Many of the early results we state, typically in terms of modules over the ring of differential operators on an affine variety, have sheaf-theoretic analogues, which we may sometimes state.
    
    Some knowledge of derived categories is useful, but again we'll avoid these where possible and black-box any results we need.
    Proofs of key results won't use derived categories.
    
    \chapter{\texorpdfstring{\(\D\)}{D}-Modules}
    In this chapter we aim to introduce \(\D\)-modules on an affine variety.
    We start with the original motivating example, the Weyl algebra, which consists of differential operators on polynomials.
    
    \section{The Weyl Algebra}
    In this section we study an algebra initially introduced by Hermann Weyl, who was doing quantum mechanics, where this algebra arises as the algebra of position and momentum generators.
    
    \begin{dfn}{Weyl Algebra}{}
        The \defineindex{Weyl algebra}, \(A_n\), is the \(\complex\)-algebra with generators \(x_1, \dotsc, x_n\) and \(y_1, \dotsc, y_n\) and defining relations
        \begin{equation}
            \bracket{x_i}{x_j} = \bracket{y_i}{y_j} = 0, \qqand \bracket{x_i}{y_j} = \delta_{ij}.
        \end{equation}
    \end{dfn}
    
    In the above definition \(\bracket{a}{b} = ab - ba\) is the commutator and \(\delta_{ij}\) is the Kronecker delta, defined by \(\delta_{ij} = 1\) when \(i = j\) and \(\delta_{ij} = 0\) otherwise.
    
    Notice that this is, up to factors of \(i\) and \(\hbar\), the same algebra that is satisfied by the position and momentum generators, \(x_i\) and \(p_i\).
    
    This definition is of the abstract Weyl algebra, it's given in terms of generators and relations.
    For \(f \in \complex[x_1, \dotsc, x_n]\) the product rule gives us
    \begin{equation}
        \diffp{}{x_i}(x_j f) = \diffp{x_i}{x_i} f + x_i \diffp{f}{x_i} = \delta_{ij}f + x_j \diffp{f}{x_i}.
    \end{equation}
    Rearranging this, we have
    \begin{equation}
        \diffp{}{x_i}(x_j f) - x_j \diffp{}{x_i}f = \bracket{\partial_i}{x_j}f = \delta_{ij}f
    \end{equation}
    where \(\partial_i = \difsp{}{x_j}\), and so
    \begin{equation}
        \bracket{\partial_i}{x_j} = \delta_{ij}.
    \end{equation}
    Note that this is an equation in \(\End_{\complex}(\complex[x_1, \dotsc, x_n])\).
    
    This leads us to the non-abstract definition of the Weyl algebra, it is the subalgebra, \(\D(\affine^n)\), of \(\End_{\complex}(\complex[x_1, \dotsc, x_n])\) generated by the operators \(x_i\) and \(\partial_i\) for \(i = 1, \dotsc, n\).
    Note here we've viewing \(x_i\) as the operation of \enquote{multiplication by \(x_i\)}, so \(x_i\) acts on \(f\) to give \(x_if\).
    We call \(\D(\affine^n)\) the \defineindex{ring of differential operators} on \(\affine^n\).
    
    \begin{lma}{}{}
        There is a surjective algebra homomorphism
        \begin{align}
            A_n &\twoheadrightarrow \D(\affine^n)\\
            x_i &\mapsto x_i\\
            y_i &\mapsto \partial_i.
        \end{align}
    \end{lma}
    
    \begin{ntn}{Multi-Index Notation}{}
        For\footnote{\(\naturals = \{0, 1, 2, \dotsc\}\).} \(\alpha = (\alpha_1, \dotsc, \alpha_n) \in \naturals^n\) let
        \begin{equation}
            x^\alpha = x_1^{\alpha_1} \dotsm x_n^{\alpha_n}, \qqand \partial^\alpha = \partial_1^{\alpha_1} \dotsm \partial_n^{\alpha_n}.
        \end{equation}
    \end{ntn}
    
    \begin{lma}{}{}
        In \(\D(\affine^n)\) for \(\alpha, \beta \in \naturals^n\) we have
        \begin{equation}
            \bracket{\partial_i}{x^\alpha} = \alpha_i x^{\alpha - e_i}, \qqand \bracket{x_i}{\partial^\beta} = -\beta_i \partial^{\beta - e_i}
        \end{equation}
        where \(e_i = (0, \dotsc, 0, 1, 0, \dotsc, 0) \in \naturals^n\) has a \(1\) in the \(i\)th position and \(0\) everywhere else.
        \begin{rmk}
            Note that when \(\alpha_i = 0\) technically \(\alpha - e_i\) isn't a well defined multi-index, having a negative entry.
            However, when this happens the \(\alpha_i\) factor in front of the term kills it anyway, so we don't have to worry about this.
        \end{rmk}
        \begin{proof}
            Let \(f, g \in \complex[x_1, \dotsc, x_n]\).
            Then we have
            \begin{align}
                \bracket{\partial_i}{f}g &= \partial_i(fg) - f\partial_ig\\
                &= (\partial_if)g + f\partial_ig - f\partial_ig\\
                &= (\partial_if)g
            \end{align}
            and so
            \begin{equation}
                \bracket{\partial_i}{f} = \partial_i f.
            \end{equation}
            Thus, taking \(f(x) = x^\alpha\) we have that
            \begin{align}
                \bracket{\partial_i}{x^\alpha} &= \partial_i x^\alpha\\
                &= \partial_i(x_1^{\alpha_1} \dotsm x_n^{\alpha_n})\\
                &= \alpha_i x_1^{\alpha_1} \dotsm x_i^{\alpha_i - 1} \dotsm x_n^{\alpha_n}\\
                &= \alpha_i x^{\alpha - e_i}.
            \end{align}
            
            The second result follows by induction on \(\abs{\beta} = \beta_1 + \dotsb + \beta_n\).
            For \(\abs{\beta} = 1\) we just have \(\partial^\beta = \partial_j\) for some \(j \in \{1, \dotsc, n\}\).
            For \(f \in \complex[x_1, \dotsc, x_n]\) we then have
            \begin{align}
                \bracket{x_i}{\partial^\beta}f &= \bracket{x_i}{\partial_j}f\\
                &= x_i\partial_jf - \partial_j(x_i f)\\
                &= x_i\partial_jf - (\partial_j x_i) f - x_i \partial_j f\\
                &= -(\partial_j x_i) f\\
                &= -\delta_{ij} f
            \end{align}
            and so we have
            \begin{equation}
                \bracket{x_i}{\partial^\beta} = -\delta_{ij}f
            \end{equation}
            which is exactly the same action as \(-\beta_i \partial^{\beta - e_i}\) since by assumption \(\beta_j = 1\) \(\beta_k = 0\) for \(k \ne j\) so for \(i = j\) we have \(-\beta_j\partial^{\beta - e_j}\), which is \(-1\partial^{(0, \dotsc, 0)} = -1\) and for \(i \ne j\) we have \(-\beta_i = 0\) so we just get \(0\).
            
            We then proceed to induct on \(\abs{\beta}\).
        \end{proof}
    \end{lma}
    
    
%    \appendixpage
%    \begin{appendices}
%        \chapter{Commutative Algebra}
Here we collect some results from commutative algebra which we'll make use of in the course.
This won't be very well organised, and is more for reference than actual reading.
The conditions to be included here are pretty much \enquote{I had to look it up} or \enquote{I had to think about it for more than 10 seconds} while writing these notes, or \enquote{I thought it was worth recapping}.

\section{Ideals}

\begin{dfn}{Prime Ideal}{}
    A proper ideal, \(\ideal{p} \subideal R\), is \define{prime}\index{prime ideal} if whenever \(ab \in \ideal{p}\) for \(a, b \in R\) then either \(a \in \ideal{p}\) or \(b \in \ideal{p}\).
    
    Equivalently, \(\ideal{p}\) is prime if \(R/\ideal{p}\) is an integral domain.
\end{dfn}

\begin{dfn}{Maximal Ideal}{}
    A proper ideal, \(\ideal{m} \subideal R\), is \define{maximal}\index{maximal ideal} if whenever there is another ideal, \(I \subideal R\), with \(\ideal{m} \subseteq I\) then either \(I = \ideal{m}\) or \(I = R\).
    
    Equivalently, \(\ideal{m}\) is maximal if \(R/\ideal{m}\) is a field.
\end{dfn}

\begin{lma}{}{lma:product of ideals subset of intersection}
    Let \(R\) be a ring with ideals \(I\) and \(J\).
    Then \(IJ \subseteq I \cap J\).
    \begin{proof}
        If \(a \in I\) and \(b \in J\) then \(ab \in I\) and \(ab \in J\) by definition of an ideal.
        Then \(ab \in I \cap J\).
    \end{proof}
\end{lma}

\begin{lma}{}{lma:radical of product is radical of intersection}
    Let \(R\) be a ring with ideals \(I\) and \(J\).
    Then \(\sqrt{IJ} = \sqrt{I \cap J} = \sqrt{I} \cap \sqrt{J}\).
    \begin{proof}
        We prove a circle of inclusions.
        We start with \(\sqrt{IJ} \subseteq \sqrt{I \cap J}\), which follows from \cref{lma:product of ideals subset of intersection}.
        
        If \(a \in \sqrt{I \cap J}\) then \(a^k \in I \cap J\) for some \(k \in \naturals\).
        Thus, \(a^k \in I\) and \(a^k \in J\).
        Hence, \(a \in \sqrt{I} \cap \sqrt{J}\).
        
        If \(a \in \sqrt{I} \cap \sqrt{J}\) then \(a^k \in I\) and \(a^{\ell} \in J\) for some \(k, \ell \in \naturals\).
        Then \(a^k a^{\ell} = a^{k + \ell} \in IJ\), and so \(a \in \sqrt{IJ}\).
    \end{proof}
\end{lma}

\begin{lma}{}{lma:prime ideal is radical}
    Every prime ideal is radical.
    \begin{proof}
        Let \(\ideal{p}\) be a prime ideal of a ring, \(R\).
        Consider \(\sqrt{\ideal{p}}\).
        If \(a \in \sqrt{\ideal{p}}\) then there exists some \(k \in \naturals\) such that \(a^k \in \ideal{p}\).
        Suppose that \(k\) is minimal in making this true.
        If \(k = 1\) then \(a \in \ideal{p}\).
        If \(k > 1\) then by the definition of a prime ideal have \(x \cdot x^{k-1} \in \ideal{p}\) implying \(x \in \ideal{p}\) or \(x^{k-1} \in \ideal{p}\).
        However, the later cannot be the case because \(k\) was assumed minimal.
        Therefore, \(x \in \ideal{p}\), and since \(\ideal{p} \subseteq \sqrt{\ideal{p}}\) (\cref{lma:ideal is subset of its radical}) it must be that \(\ideal{p} = \sqrt{\ideal{p}}\).
    \end{proof}
\end{lma}

\section{Noetherian Rings}
\begin{dfn}{Noetherian}{}
    A ring, \(R\), is noetherian if it satisfies the ascending chain condition.
    That is, if every chain of ideals,
    \begin{equation}
        I_1 \subseteq I_2 \subseteq I_3 \subseteq \dotsb
    \end{equation}
    terminates, so \(I_{n+1} = I_n\) for sufficiently large \(n\).
\end{dfn}

Note that all fields are noetherian, and so is \(\integers\).

\begin{lma}{}{lma:noetherian iff all ideals finitely generated}
    Let \(R\) be a ring.
    The following are equivalent:
    \begin{enumerate}
        \item \(R\) is a noetherian.
        \item Every ideal of \(R\) is finitely generated.
    \end{enumerate}
\end{lma}

\subsection{Hilbert's Basis Theorem}
\begin{thm}{Hilbert's Basis Theorem}{thm:hilberts basis theorem}
    If \(R\) is a noetherian ring then \(R[x]\) is also Noetherian.
\end{thm}

\begin{crl}{}{crl:poly ring over noetherian is noetherian}
    If \(R\) is a noetherian ring then \(R[x_1, \dotsc, x_n]\) is noetherian.
\end{crl}


%    \end{appendices}
    \backmatter
    \renewcommand{\glossaryname}{Acronyms}
    \printglossary[acronym]
    \printindex
\end{document}