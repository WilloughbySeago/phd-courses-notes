% !TeX program = lualatex
\documentclass[fleqn]{NotesClass}

\strictpagecheck

\usepackage{csquotes}
\usepackage{tensor}

\usepackage{tikz}
\usetikzlibrary{external}
\tikzexternalize[prefix=tikz-external/]

\usepackage{tikz-cd}

\usepackage[pdfauthor={Willoughby Seago},pdftitle={Notes from Conformal Field Theory and Vertex Operator Algebras},pdfkeywords={conformal field theory,CFT,vertex operator algebra,voa},pdfsubject={Conformal Field Theory and Vertex Operator Algebras}]{hyperref}  % Should be loaded second last (cleveref last)
\colorlet{hyperrefcolor}{blue!60!black}
\hypersetup{colorlinks=true, linkcolor=hyperrefcolor, urlcolor=hyperrefcolor}
\usepackage[
capitalize,
nameinlink,
noabbrev
]{cleveref} % Should be loaded last

% My packages
\usepackage{NotesBoxes}
\usepackage{NotesMaths2}

\setmathfont[range={\int, \oint, \otimes, \oplus, \bigotimes, \bigoplus}]{Latin Modern Math}


% Highlight colour
\definecolor{highlight}{HTML}{710D78}
\definecolor{my blue}{HTML}{2A0D77}
\definecolor{my red}{HTML}{770D38}
\definecolor{my green}{HTML}{14770D}
\definecolor{my yellow}{HTML}{E7BB41}

% Title page info
\title{Conformal Field Theory and Vertex Operator Algebras}
\author{Willoughby Seago}
\date{September 26th, 2024}
\subtitle{Notes from}
\subsubtitle{SMSTC}
\renewcommand{\abstracttext}{These are my notes from the SMSTC course \emph{Conformal Field Theories and Vertex Operator Algebras} taught by Dr Anatoly Konechny. These notes were last updated at \printtime{} on \today{}.}

% Commands
% Maths
\newcommand{\manifold}{\symcal{M}}
\renewcommand{\dd}[1]{\,\symrm{d}#1}
\renewcommand{\dl}[1]{\symrm{d}#1}
\DeclarePairedDelimiterX{\innerproduct}[2]{(}{)}{#1, #2}
\newcommand{\pt}{\symrm{pt}}
\newcommand{\isomorphic}{\cong}
\DeclareMathOperator{\Aff}{Aff}

\begin{document}
    \frontmatter
    \titlepage
    \innertitlepage{}
    \tableofcontents
    % \listoffigures
    \mainmatter
    \chapter{Conformal Geometry}
    \section{Local Conformal Maps}
    Intuitively, we want conformal maps to preserve angles, but not necessarily distances.
    To do so consider how the angle between two vectors in, say, \(\reals^3\) is computed,
    \begin{equation}
        \cos(\theta_{\vv{u}, \vv{v}}) = \frac{\vv{u} \cdot \vv{v}}{\norm{\vv{u}} \norm{\vv{v}}} .
    \end{equation}
    We see here that if each vector were made longer by some positive constant, \(\rho\), then we have
    \begin{equation}
        \cos(\theta_{\rho \vv{u}, \rho \vv{v}})\frac{\rho \vv{u} \cdot \rho \vv{v}}{\norm{\rho \vv{u}} \norm{\rho \vv{v}}} = \cos(\theta_{\vv{u}, \vv{v}}).
    \end{equation}
    We can think of scaling all of the vectors here by \(\rho\) as the same as scaling the metric by \(1/\rho\).
    It's \(1/\rho\) because if we make the units of a measurement smaller then the number we measure gets bigger (hence, \emph{contravariant} vectors).
    
    The following is really just fancy differential-geometry-speak for this rescaling of the metric, and we also allow the scaling of the metric to depend on position.
    
    \begin{dfn}{Local Conformal Map}{}
        Let \((\manifold_1, g_1)\) and \((\manifold_2, g_2)\) be \(n\)-dimensional Riemannian manifolds.
        Let \(U_1 \subseteq \manifold_1\) and \(U_2 \subseteq \manifold_2\) be open subsets.
        A (local) \defineindex{conformal transformation} is a smooth, injective map, \(\varphi \colon U_1 \to U_2\), satisfying the pullback condition
        \begin{equation}
            \varphi^*g_2 = \Lambda g_1
        \end{equation}
        for some function \(\Lambda \colon U_1 \to \reals_{>0}\).
        
        A conformal map defined on all of \(\manifold_1\) is a global conformal map.
    \end{dfn}
    
    \begin{remark}{}{}
        It is possible to relax the conditions on \(\varphi\), and require only that it is differentiable.
        However, requiring smoothness and injectivity is common when it comes to applications, so we make it a basic requirement.
        It's also common to further restrict to orientation-preserving maps, but we won't do that just yet.
    \end{remark}
    
    We can express the pullback condition, \(\varphi^*g_2 = \Lambda g_1\), in local coordinates.
    Let \(x = (x^1, \dotsc, x^n)\) be coordinates covering \(U_1\), and \(y = (y^1, \dotsc, y^n)\) coordinates covering \(U_2\).
    Then \(\varphi\) is fully specified by the functions \(\varphi^i\) which are defined such that \(y^i = \varphi^i(x)\).
    The metrics, \(g_i\), may be specified by their components, \((g_i)_{jk} \colon U_i \to \reals\).
    In these coordinates the pullback condition becomes
    \begin{equation}
        \sum_{k,l} \diffp{\varphi^k}{x^i} \diffp{\varphi^l}{x^j} (g_2)_{kl}(\varphi(x)) = \Lambda(x) (g_1)_{ij}(x).
    \end{equation}
    Taking determinants of either side of this equation we have
    \begin{equation}
        \det\left( \diffp{\varphi^k}{x^i} \right) \det(g_2) \det\left( \diffp{\varphi^l}{x^j} \right) = \Lambda^n \det (g_1).
    \end{equation}
    Now, \(\Lambda^n \ne 0\) and \(\det(g_i) \ne 0\), so it follows that \(\det(\difsp{\varphi^k}{x^i}) \ne 0\), meaning that the matrix with components \(\difsp{\varphi^k}{x^i}\) is invertible.
    This means that a conformal map is always \defineindex{locally invertible}.
    That is, for any \(p \in U_1\) we have a neighbourhood \(V_1 \subseteq U_1\) with \(p \in V_1\) such that \(\varphi\) restricted to \(V_1\) is a bijection.
    
    Note that it's possible to be locally invertible but not fully invertible.
    There may be a point in \(U_1 \setminus V_1\) which maps to the same point as a point in \(V_1\), so the function will not be injective.
    
    \subsection{Conformal Maps Preserve Angles}
    Consider two vectors \(u, v \in T_p\manifold_1\) and some \(p \in \manifold_1\).
    Taking some open neighbourhood of \(p\), \(U_1 \subseteq \manifold_1\), we can also take coordinates \(x = (x^1, \dotsc, x^n)\) covering \(U_1\).
    This gives a basis \(\{\difsp{}{x^i}|_p\}\) for \(T_p\manifold_1\).
    The angle between these vectors is given, as in \(\reals^3\), by
    \begin{equation}
        \cos(\theta_{u, v}) = \frac{\innerproduct{u}{v}}{\norm{u} \norm{v}} = \frac{u^i (g_1)_{ij} v^j}{\sqrt{u^l(g_1)_{lk}u^k v^p (g_1)_{pq} v^q}}.
    \end{equation}
    Here we've started to employ the Einstein summation convention, and we shall do so from now on.
    Let \(\varphi \colon U_1 \to U_2\) be a conformal transformation and suppose that \(U_2\) is covered by coordinates \(y = (y^1, \dotsc, y^n)\).
    Consider the pushforward
    \begin{equation}
        \dl{\varphi_p} \colon T_p \manifold_1 \to T_{\varphi(p)} \manifold_2.
    \end{equation}
    Under this the vectors \(u\) and \(v\) map to the vectors
    \begin{equation}
        \tilde{u} = \dl{\varphi_p}(u), \qqand \tilde{v} = \dl{\varphi_p}(v),
    \end{equation}
    which have coordinates
    \begin{equation}
        \tilde{u}^i = \diffp{\varphi^i}{x^j} u^j, \qqand \tilde{v}^i = \diffp{\varphi^i}{x^j} v^j.
    \end{equation}
    We can now calculate the angle between these vectors as follows:
    \begin{align}
        \cos(\theta_{\tilde{u}, \tilde{v}}) &= \frac{\innerproduct{\tilde{u}}{\tilde{v}}}{\norm{u} \norm{v}}\\
        &= \frac{\tilde{u}^a (g_2)_{ab} \tilde{v}^b}{\sqrt{\tilde{u}^c(g_2)_{cd} \tilde{u}^d \tilde{v}^e (g_2)_{ef} \tilde{v}^f}}\\
        &= \frac{\diffp{\varphi^a}{x^i} u^i (g_2)_{ab} \diffp{\varphi^b}{x^j} u^j}{\sqrt{\diffp{\varphi^c}{x^l} u^l (g_2)_{cd} \diffp{\varphi^d}{x^k} u^k \diffp{\varphi^e}{x^p} v^p (g_2)_{ef} \diffp{\varphi^f}{x^q}}}\\
        &= \frac{u^i \Lambda(g_1)_{ij} v^j}{\sqrt{u^l \Lambda(g_1)_{lk}u^k v^p \Lambda(g_1)_{pq}v^q}}\\
        &= \frac{u^i (g_1)_{ij} v^j}{\sqrt{u^l (g_1)_{lk}u^k v^p (g_1)_{pq}v^q}}\\
        &= \cos(\theta_{u,v})
    \end{align}
    where we've used the pullback condition
    \begin{equation}
        \diffp{\varphi^a}{x^i} (g_2)_{ab} \diffp{\varphi^b}{x^j} = \Lambda (g_1)_{ij}.
    \end{equation}
    This shows that conformal transformations really do preserve angles as we were looking for.
    
    \section{Conformal Transformations on \(\reals^2\)}
    We now restrict ourselves to the case \(\manifold_1 = \manifold_2 = \reals^2\) with the standard Euclidean metric.
    In Cartesian coordinates, \((x^1, x^2)\), the metric corresponds to the matrix
    \begin{equation}
        \begin{pmatrix}
            1 & 0\\
            0 & 1
        \end{pmatrix}
        .
    \end{equation}
    It turns out that it's useful to identify \(\reals^2\) with \(\complex\) using the complex coordinates
    \begin{equation}
        z = x^1 + ix^2, \qqand \overbar{z} = x^1 - ix^2.
    \end{equation}
    In the coordinates \((z, \overbar{z})\) the metric corresponds to the matrix
    \begin{equation}
        \frac{1}{2}
        \begin{pmatrix}
            0 & 1\\
            1 & 0
        \end{pmatrix}
        .
    \end{equation}
    To see this notice that
    \begin{equation}
        x^1 = \frac{1}{2}(z + \overbar{z}), \qqand x^2 = \frac{1}{2i}(z - \overbar{z}).
    \end{equation}
    Thus, by the chain rule, we have
    \begin{equation}
        \dl{x^1} = \diffp{x^1}{z} \dd{z} + \diffp{x^1}{\overbar{z}} \dd{\overbar{z}}, \qqand \dl{x^2} = \diffp{x^2}{z} \dd{z} + \diffp{x^2}{\overbar{z}}.
    \end{equation}
    Inserting this into the usual Euclidean metric we have
    \begin{align}
        g &= (\dl{x^1})^2 + (\dl{x^2})^2\\
        &= \left( \diffp{x^1}{z}\dd{z} + \diffp{x^1}{\overbar{z}} \dd{\overbar{z}} \right)^2 + \left( \diffp{x^2}{z} \dd{z} + \diffp{x^2}{\overbar{z}} \dd{\overbar{z}} \right)^2\\
        &= \left( \frac{1}{2} \dl{z} + \frac{1}{2}\dl{\overbar{z}} \right)^2 + \left( \frac{1}{2i} \dl{z} - \frac{1}{2i} \dl{\overbar{z}} \right)^2\\
        &= \dl{z} \dd{\overbar{z}}.
    \end{align}
    Thus, we must have that
    \begin{equation}
        g_{zz} \dd{z} \dd{z} = g_{\overbar{z}\overbar{z}} \dd{\overbar{z}} \dd{\overbar{z}} = 0, \qand g_{z\overbar{z}}\dd{z}\dd{\overbar{z}} + g_{\overbar{z}z}\dd{\overbar{z}}\dd{z} = 1.
    \end{equation}
    The requirement that \(g\) is symmetric implies the matrix form must be symmetric, which is how we end up with the \(1\) evenly split between the off-diagonal elements.
    
    \subsection{Analytic Structure of Conformal Transformations}
    In the complex coordinates, \((z, \overbar{z})\), we may consider a conformal transformation, \(\varphi \colon U \to V\), as consisting of two maps,
    \begin{equation}
        z \mapsto \varphi(z, \overbar{z}), \qqand \overbar{z} \mapsto \overbar{\varphi}(z, \overbar{z}) = \overline{\varphi(z, \overbar{z})}.
    \end{equation}
    Since the second is just the conjugate of the first we need only specify one of these maps.
    
    We can now write out the pullback condition and check what this imposes on \(\varphi\):
    \begin{align}
        \varphi^*(\dl{z}, \dl{\overbar{z}}) &= \dl{\varphi} \, \overline{\dl{\varphi}}\\
        &= \left( \diffp{\varphi}{z} \dd{z} + \diffp{\varphi}{\overbar{z}} \dd{\overbar{z}} \right)\left( \diffp{\overbar{\varphi}}{z} \dd{z} + \diffp{\overbar{\varphi}}{\overbar{z}} \dd{\overbar{z}} \right)\\
        &= \diffp{\varphi}{z}\diffp{\overbar{\varphi}}{z} (\dl{z})^2 + \diffp{\varphi}{z} \diffp{\overbar{\varphi}}{\overbar{z}} \dd{z} \dd{\overbar{z}} + \diffp{\varphi}{\overbar{z}} \diffp{\overbar{\varphi}}{z} \dd{\overbar{z}} \dd{z} + \diffp{\varphi}{\overbar{z}} \diffp{\overbar{\varphi}}{\overbar{z}} (\dl{\overbar{z}})^2 \notag\\
        &= \left( \diffp{\varphi}{z} \diffp{\overbar{\varphi}}{\overbar{z}} + \diffp{\varphi}{\overbar{z}} \diffp{\overbar{\varphi}}{z} \right) \dd{z} \dd{\overbar{z}} + \diffp{\varphi}{z} \diffp{\overbar{\varphi}}{z} (\dl{z})^2 + \diffp{\varphi}{\overbar{z}} \diffp{\overbar{\varphi}}{\overbar{z}} (\dl{\overbar{z}})^2.
    \end{align}
    Introducing the notation \(\partial_w f = \difsp{f}{w}\) we can now use
    \begin{equation}
        \overline{\partial_w f} = \overline{\diffp{f}{w}} = \diffp{\overbar{f}}{\overbar{w}} = \partial_{\overbar{w}} \overbar{f}
    \end{equation}
    to rewrite this in terms of \(\difsp{\varphi}{z}\) and \(\difsp{\varphi}{\overbar{z}}\):
    \begin{multline}
        \varphi^*(\dl{z}, \dl{\overbar{z}}) = ( (\partial_z \varphi) (\overline{\partial_{z} \varphi}) + (\partial_{\overbar{z}} \varphi) (\overline{\partial_{\overbar{z}} \varphi}) ) \dd{z} \dd{\overbar{z}}\\
        + (\partial_z \varphi) (\overline{\partial_{\overbar{z} \varphi}}) (\dl{z})^2 + (\partial_{\overbar{z}} \varphi) (\overline{\partial_z \varphi}) (\dl{\overbar{z}})^2.
    \end{multline}
    Now, if the pullback condition is to hold we must have
    \begin{equation}
        \varphi^*(\dl{z}, \dl{\overbar{z}}) = \Lambda(z, \overbar{z}) \dd{z} \dd{\overbar{z}}.
    \end{equation}
    So, we require that the \((\dl{z})^2\) and \((\dl{\overbar{z}})^2\) terms vanish, which imposes
    \begin{equation}
        (\partial_z \varphi) (\overline{\partial_{\overbar{z}} \varphi}) = 0.
    \end{equation}
    
    \subsection{Conformal Transformations Examples}
    We can now ask what conformal transformations \(\varphi \colon \complex \to \complex\) exist.
    We'll also consider conformal transformations defined only on \(\complex\setminus\{\pt\}\) where \(\pt\) is some point where the transformation need not be defined.
    
    \subsubsection{Rigid Transformations}
    The rigid transformations, or \define{isometries}\index{isometry} of \(\complex\) are the distance preserving transformations.
    As such they preserve the metric, and thus are conformal with \(\Lambda = 1\).
    It is well known that the isometries of the plane fall into three different classes.
    
    \paragraph{Translation}
    A translation, \(T_b \colon \complex \to \complex\), by \(b \in \complex\) is defined by
    \begin{equation}
        T_b(z) = z + b.
    \end{equation}
    This clearly preserves the metric since the distance between \(z\) and \(w\) is \(\abs{z - w}\), and the distance between \(T_b(z)\) and \(T_b(w)\) is \(\abs{T_b(z) - T_b(w)} = \abs{z + b - w - b} = \abs{z - w}\).
    Clearly \(T_b\) is bijective and smooth also, and it's defined everywhere, so it's a global conformal transformation.
    
    Translations form a group, which is simply the additive group \((\complex, +)\).
    
    \paragraph{Rotation}
    An anticlockwise rotation, \(R_{\alpha, 0} \colon \complex \to \complex\), by an angle \(\alpha\) about the origin, \(0\), is defined by
    \begin{equation}
        R_{\alpha, 0}(z) = \e^{i\alpha}z.
    \end{equation}
    This preserves the metric, since \(\abs{R_{\alpha, 0}(z) - R_{\alpha, 0}(w)} = \abs{\e^{i\alpha}z - \e^{i\alpha}w} = \abs{\e^{i\alpha}(z - w)} = \abs{z - w}\).
    Again, this is bijective, smooth, and defined everywhere, so it is also a global conformal transformation.
    
    Rotations about the origin form a group, which is simply \(\specialOrthogonal(2, \reals) \isomorphic \unitary(1)\).
    
    A rotation about some point, \(c \in \complex\), by an angle \(\alpha\) is simply given by
    \begin{equation}
        R_{\alpha, c} = T_c \circ R_{\alpha, 0} \circ T_{-c}.
    \end{equation}
    That is, we first translate the centre of rotation to the origin, then rotate, then translate back.
    This is again a conformal transformation, as all composites of conformal transformations are.
    
    Rotations and translations form a group.
    
    \paragraph{Reflection}
    A reflection, \(C_{0, 0} \colon \complex \to \complex\), in the real axis, that is \(x^2 = 0\), is given by conjugation
    \begin{equation}
        C_{0, 0}(z) = \overbar{z}.
    \end{equation}
    This preserves the metric, since \(\abs{C_{0, 0}(z) - C_{0, 0}(w)} = \abs{\overbar{z} - \overbar{w}} = \abs{\overline{z - w}} = \abs{z - w}\).
    Again, this is bijective, smooth, and defined everywhere, so it is also a global conformal transformation.
    
    Reflections in the real axis form a group isomorphic to \(\integers/2\integers\).
    
    A reflection in an arbitrary line can similarly be given as a composition of previously defined conformal transformations.
    To do so we parametrise the line by a point, \(c \in \complex\), through which it passes, and the angle, \(\alpha\), that the line makes to the positive real axis.
    Then a reflection in this line is given by
    \begin{equation}
        C_{\alpha, c} = T_b \circ R_{-\alpha, 0} \circ C_{0, 0} \circ R_{\alpha, 0} \circ T_{-b}.
    \end{equation}
    This is again a conformal transformation.
    
    \subsubsection{Dilation}
    A \defineindex{dilation} or \defineindex{dilitation} is a global conformal transformation, \(D_\rho \colon \complex \to \complex\), for \(\rho \in \reals_{>0}\), defined by
    \begin{equation}
        D_\rho(z) = \rho z.
    \end{equation}
    Intuitively, for \(\rho > 1\) this corresponds to stretching space out evenly in all directions, and for \(\rho < 1\) it's a contraction.
    Clearly this is bijective and smooth, and it is conformal with \(\Lambda = \rho^2\), since \(\dl{\rho z} \dd{\rho \overbar{z}} = \rho^2 \dd{z}\dd{\overbar{z}}\).
    
    Dilations form a group, which is just the multiplicative group \((\complex^{\times}, \cdot)\).
    
    Just looking at the formulae we see that rotations about the origin and dilations are very similar, and in fact, we can combine them into a single transformation \(D_a \colon \complex \to \complex\) with \(a \in \complex\) and \(a = \rho \e^{i\alpha}\) for \(\rho \in \reals_{>0}\) and \(\alpha \in \reals\) and
    \begin{equation}
        D_a(z) = az = \rho \e^{i\alpha} z,
    \end{equation}
    which is a dilation by \(\rho\) and a rotation by \(\alpha\) (the order of which is not important).
    That is, \(D_a = D_\rho \circ R_{\alpha, 0} = R_{\alpha, 0} \circ D_\rho\), and \(R_{\alpha, 0} = D_{\e^{i\alpha}}\).
    
    The combination of all translations, dilations, and rotations forms a group, \(\Aff(\complex)\), the affine group.
    
    \subsubsection{Circle Inversion}
    
    
    
%	 Appdendix
	\appendixpage
	\begin{appendices}
    	\chapter{Differential Geometry}
        \section{Tangent Space}
        Let \(\manifold\) be a \(d\)-dimensional manifold.
        The tangent space at \(p \in M\) is a \(d\)-dimensional vector space \(T_p\manifold\).
        One definition of this is the vector space of derivations at \(p\), where a derivation is a linear map \(D \colon C^{\infty}(\manifold) \to \reals\) satisfying
        \begin{equation}
            D(f g) = D(f) g(x) + f(x) D(g).
        \end{equation}
        Clearly derivatives are derivations, this is just the product rule, and in fact given a coordinate chart \((U, x)\) with \(p \in U\) and \(x = (x^1, \dotsc, x^d)\) we have a basis for \(T_p\manifold\) given by
        \begin{equation}
            \left\{ \diffp{}{x^1}\bigg|_p, \dotsc, \diffp{}{x^d}\bigg|_p \right\}.
        \end{equation}
        
        Once we have tangent spaces it makes sense to consider the collection of all tangent vectors at any point \(p \in \manifold\).
        This gives us the tangent bundle
        \begin{equation}
            TM = \bigsqcup_{p \in \manifold} T_pM.
        \end{equation}
        This is a bundle since we have the natural projection \(\pi \colon TM \twoheadrightarrow \manifold\) sending a tangent vector \(v \in T_pM\) to the point \(p \in M\).
        
        \section{Pushforward and Pullback}
        Let \(\varphi \colon \manifold \to \symcal{N}\) be a smooth map between manifolds.
        The \defineindex{pushforward}, also called the \defineindex{differential}, of \(\varphi\) at \(p \in \manifold\) is the linear map
        \begin{equation}
            \dl{\varphi_p} \colon T_p\manifold \to T_{\varphi(p)}\symcal{N}
        \end{equation}
        defined to act on a derivation, \(X \colon C^{\infty}(\manifold) \to \reals\), by sending it to the derivation \(\dl{\varphi_p}(X) \colon C^{\infty}(\symcal{N}) \to \reals\) defined to act on \(f \in C^{\infty}(\symcal{N})\) by
        \begin{equation}
            \dl{\varphi_p}(X)(f) = X(f \circ \varphi).
        \end{equation}
        That is, \(\dl{\varphi_p}\) is nothing but precomposition with \(\varphi\) followed by evaluation.
        
        Fix charts \((U, x)\) and \(V, y\) for neighbourhoods of \(p \in \manifold\) and \(\varphi(p) \in \symcal{N}\). s
        Then \(T_p\manifold\) and \(T_{\varphi(p)}\symcal{N}\) have bases \(\{\difsp{}{x^i}|_p\}\) and \(\{\difsp{}{y^i}|_{\varphi(p)}\}\).
        In these bases \(\dl{\varphi_p}\) may be expressed as a matrix
        \begin{equation}
            \tensor{(\dl{\varphi_p})}{^i_j} = \diffp{\varphi^i}{x^j}
        \end{equation}
        where \(\varphi^j\) is such that \(y^j = \varphi(x^j)\).
        
        The \defineindex{pullback} of \(\varphi\) is the map \(\varphi^* \colon C^{\infty}(\symcal{N}) \to C^{\infty}(\manifold)\) defined by \((\varphi^*f)(x) = f(\varphi(x))\).
        That is, \(\varphi^*\) is precomposition with \(\varphi\).
        We can also define the pullback of a \(k\)-form, \(\omega\), as
        \begin{equation}
            (\varphi^*\omega)_p(X_1, \dotsc, X_k) = \omega_{\varphi(p)}(\dl{\varphi_p}(X_1), \dotsc, \dl{\varphi_p}(X_k)).
        \end{equation}
        This will be particularly important for a 2-form, \(g\), where we have
        \begin{equation}
            (\varphi^*g)_p(X_1, X_2) = g_{\varphi(p)}(\dl{\varphi_p}(X_1), \dl{\varphi_p}(X_2)).
        \end{equation}
        
        
        \section{Riemannian Manifolds}
        A \defineindex{Riemannian manifold}, \((\manifold, g)\), is a manifold, \(\manifold\), equipped with a \defineindex{Riemannian metric}, \(g\), which assigns to each tangent space, \(T_p\manifold\), a positive-definite inner product
        \begin{equation}
            g_p \colon T_p\manifold \times T_p\manifold \to \reals,
        \end{equation}
        such that the component functions, \(g_{ij} \colon U \to \reals\), are smooth on any chart \((U, x)\).
        These components are defined for a basis \(\{e_i\}\) of \(T_p\manifold\) by
        \begin{equation}
            g_{ij} = g_p(e_i, e_j).
        \end{equation}
        These are such that
        \begin{equation}
            g = \sum_{i,j} g_{ij} \dd{x^i} \dd{x^j}
        \end{equation}
        where \(\dl{x^i}\) is the dual basis to \(\{e_i\}\), defined by \(\dl{x^i}(e_j) = \tensor{\delta}{^{i}_j}\).
    \end{appendices}

%	\backmatter
%	\renewcommand{\glossaryname}{Acronyms}
%	\printglossary[acronym]
%	\printindex
\end{document}
