% !TeX program = lualatex
\documentclass[fleqn, a4paper, openany]{memoir}

\strictpagecheck

\usepackage{mathtools}

\usepackage[math-style=upright]{unicode-math}
\setmainfont{TeX Gyre Pagella}
\setmathfont{Euler Math}[Scale=MatchLowercase]
\setsansfont{Optima}
\setmonofont{inconsolata}

\usepackage{csquotes}
\usepackage{enumitem}

\usepackage{amsthm}

\usepackage{biblatex}
\addbibresource{ref.bib}

\usepackage{tikz}
\usetikzlibrary{arrows.meta}
\usetikzlibrary{external}
\tikzexternalize[prefix=tikz-external/]
\usetikzlibrary{calc}

\usepackage{tikz-cd}
\AtBeginEnvironment{tikzcd}{\tikzexternaldisable}
\AtEndEnvironment{tikzcd}{\tikzexternalenable}

\usepackage[pdfauthor={Willoughby Seago},pdftitle={Notes on the Dynkin Classification},pdfkeywords={Dynkin diagram, Coxeter, Lie Algebra, Cluster Algebra, Quiver Representation, Simple Singularity},pdfsubject={Dynkin Classification}]{hyperref}  % Should be loaded second last (cleveref last)
\colorlet{hyperrefcolor}{blue!60!black}
\hypersetup{colorlinks=true, linkcolor=hyperrefcolor, urlcolor=hyperrefcolor}
\usepackage[
capitalize,
nameinlink,
noabbrev
]{cleveref} % Should be loaded last

% My packages
\usepackage{NotesBoxes}
%\usepackage{NotesMaths2}

% Title Page
\makeatletter
\newcommand{\@subtitle}{Mathematics}
\newcommand{\subtitle}[1]{%
    \renewcommand{\@subtitle}{#1}
}
\newcommand{\@subsubtitle}{Notes}
\newcommand{\subsubtitle}[1]{%
    \renewcommand{\@subsubtitle}{#1}
}

\newcommand{\titlepage}{%
    \begin{titlingpage}
        \begingroup%
        \raggedleft
        \vspace*{\baselineskip}
        {\LARGE \theauthor}\\[0.167\textheight]
        {\large\bfseries \@subtitle}\\[\baselineskip]
        {\HUGE\bfseries\textcolor{gray}{\thetitle}}\\[\baselineskip]
        {\large\thedate}\par
        \vspace*{2\baselineskip}
        \vfill
        {\LARGE\scshape \@subsubtitle}\par
        \vspace*{3\baselineskip}
        \endgroup
    \end{titlingpage}
}
\makeatother

\newcommand{\innertitlepage}[1]{%
    \maketitle  % Print the normal title
    \begin{abstract}  % Information about the course and this document
        \abstracttext
    \end{abstract}
    % Add interesting image from the course
    \begingroup\centering
    \vfill
    % Test to see if an argument has been provided
    \ifx&#1&%
    % No argument, put a demo image
%    \includegraphics[width=0.75\textwidth]{example-image-a}
    \else
    % Argument, put the argument as the image
    \includegraphics[width=0.75\textwidth]{#1}
    \fi
    \vfill
    \endgroup
}

\newcommand{\abstracttext}{These are my notes from the spring school \enquote{The Dynkin Classification}. This spring school was run from 31st March to 4th April 2025, taking place at Ruhr Universit\"at Bochum in Germany as part of the \enquote{Combinatorial Synergies} program. I am greatful for funding recieved from \enquote{DFG priority program 2458 Combinatorial Synergies} and \enquote{EPSRC AGQ CDT (EP/Y035232/1)} which made this trip possible.}

% Numbering of equations etc.
\counterwithin{equation}{section}
\counterwithin{figure}{chapter}
\counterwithin{table}{chapter}

% Highlight colour
\definecolor{dark orange}{HTML}{E84E1B}
\definecolor{mid orange}{HTML}{ED6910}
\definecolor{mid light orange}{HTML}{F18505}
\definecolor{light orange}{HTML}{F28D02}
\colorlet{highlight}{light orange}

% Title page info
\title{Dynkin Classification}
\author{Willoughby Seago}
\date{31st March to 4th April 2025}
\subtitle{Notes on the}
\subsubtitle{Bochum Spring School}


% Commands
% Text
\newcommand{\ADE}{\ensuremath{\symrm{A\mkern-2muD\mkern-2muE}}}
\newcommand{\define}[1]{\textbf{#1}}

% Maths
\NewDocumentCommand{\A}{o}{\symrm{A}\IfNoValueF{#1}{_{#1}}}
\NewDocumentCommand{\D}{o}{\symrm{D}\IfNoValueF{#1}{_{#1}}}
\NewDocumentCommand{\E}{o}{\symrm{E}\IfNoValueF{#1}{_{#1}}}
\NewDocumentCommand{\B}{o}{\symrm{B}\IfNoValueF{#1}{_{#1}}}
\NewDocumentCommand{\C}{o}{\symrm{C}\IfNoValueF{#1}{_{#1}}}
\NewDocumentCommand{\G}{O {2}}{\symrm{G}\IfNoValueF{#1}{_{#1}}}
\NewDocumentCommand{\F}{O {4}}{\symrm{F}\IfNoValueF{#1}{_{#1}}}
\RenewDocumentCommand{\H}{o}{\symrm{H}\IfNoValueF{#1}{_{#1}}}
\NewDocumentCommand{\I}{o o}{\symrm{I}\IfNoValueF{#1}{_{#1}(#2)}}
\NewDocumentCommand{\affA}{o}{\widetilde{\symrm{A}}\IfNoValueF{#1}{_{#1}}}
\NewDocumentCommand{\affD}{o}{\widetilde{\symrm{D}}\IfNoValueF{#1}{_{#1}}}
\NewDocumentCommand{\affE}{o}{\widetilde{\symrm{E}}\IfNoValueF{#1}{_{#1}}}
\NewDocumentCommand{\affB}{o}{\widetilde{\symrm{B}}\IfNoValueF{#1}{_{#1}}}
\NewDocumentCommand{\affC}{o}{\widetilde{\symrm{C}}\IfNoValueF{#1}{_{#1}}}
\NewDocumentCommand{\affG}{O {2}}{\widetilde{\symrm{G}}\IfNoValueF{#1}{_{#1}}}
\NewDocumentCommand{\affF}{O {4}}{\widetilde{\symrm{F}}\IfNoValueF{#1}{_{#1}}}
\newcommand{\reals}{\symbb{R}}
\newcommand{\isomorphic}{\cong}
\newcommand{\lie}[1]{\symfrak{#1}}
\newcommand{\specialLinearLie}{\lie{sl}}
\newcommand{\specialOrthogonalLie}{\lie{so}}
\newcommand{\symplecticLie}{\lie{sp}}
\newcommand{\generalLinearLie}{\lie{gl}}
\DeclarePairedDelimiterX{\innerprod}[2]{\langle}{\rangle}{#1 , #2}
\newcommand{\generalLinear}{\symrm{GL}}
\newcommand{\id}{\symrm{id}}
\newcommand{\orthogonal}{\symrm{O}}
\newcommand{\Dih}{\symrm{Dih}}
\newcommand{\Sym}{\symrm{Sym}}
\newcommand{\integers}{\symbb{Z}}
\ExplSyntaxOn
% Create LaTeX interface command
\NewDocumentCommand{\cycle}{ O{\,} m }{  % optional arg is separator, mandatory
    %arg is comma separated list
    (
    \willoughby_cycle:nn { #1 } { #2 }
    )
}

\clist_new:N \l_willougbhy_cycle_clist  % Create new clist variable
\cs_new_protected:Npn \willoughby_cycle:nn #1 #2 {  % create LaTeX3 function
    \clist_set:Nn \l_willougbhy_cycle_clist { #2 }  % set clist variable with
    %clist #2 passed by user
    \clist_use:Nn \l_willougbhy_cycle_clist { #1 }  % print list separated by #1
}
\ExplSyntaxOff
\newcommand{\quaternions}{\symbb{H}}
\newcommand{\qand}{\quad\text{and}\quad}
\newcommand{\qqand}{\qquad\text{and}\qquad}

\begin{document}
    \frontmatter
    \titlepage
    \innertitlepage{}
    \newpage
    \tableofcontents
    \mainmatter
    
    \addtocounter{chapter}{-1}
\chapter{Dynkin Classification}
\section{On the Ubiquity of Dynkin Classifications}
The Dynkin classification arises in many areas of mathematics, from representation theory to algebraic geometry, and from combinatorics to string theory and spin structures, and many more.
Of these we will only touch on the first three.

The structure of these notes follows that of the original spring school.
Each day was focused on a different topic, and each day refers to a chapter here.
Some of these days focused on the classification more than others, but all were about objects for which the Dynkin classification plays an important role.

Exactly which Dynkin diagrams are allowed in a given classification (or which of the related Coxeter diagrams appears for the first topic) depends on the situation at hand, but in all cases we get at least the \(\A\), \(\D\), and \(\E\) types, leading to the so-called \ADE-classification.

The precise reason why Dynkin diagrams, particularly the \ADE-types, appear so frequently is somewhat of a mystery.
Often, but not always, at least not in an obvious way, it's because there is some root system involved, which is a set of vectors in \(\reals^n\) subject to some conditions (), the structure of which may be used to generate more complicated objects.
% TODO: reference to definition of abstract root system
Each root system has a corresponding bilinear form, and it is the properties of this inner product which usually restrict which root systems appear in a given classification.
The Dynkin diagrams then are just a neat way of encapsulating the minimum amount of information required to define a root system.
This is done by first picking a special basis for the root system (a set of so-called simple roots) and then encoding the relative positions of these basis vectors in a graph.
The vertices of the graph correspond to elements of this basis, and the number of edges is controlled by the value of the bilinear form when evaluated at the corresponding basis vectors.
The direction of any directed edges is chosen such that the arrow points towards the shorter vector.
One may then argue that the omnipresence of the Dynkin classification is simply because many problems come down to, or can be phrased as, arranging some points in space subject to some relation between the position vectors relative angles and lengths.

\section{Coxeter and Dynkin Diagrams}
Throughout the first day the following Dynkin diagrams were on the board for reference.
We start with our main focus, the (finite type) Dynkin diagrams.
We'll then list the (finite type) Coxeter diagrams, which are similar but distinct.
Finally, we'll list the affine Coxeter diagrams.

\subsection{Dynkin Diagrams}
Here are the (\define{finite type}) \define{Dynkin diagrams}.
The subscript in each case is the number of nodes, which is also known as the \define{rank}.
Dotted edges just mean a chain of nodes connected by single edges.
When there are multiple edges there is an arrow, which indicates the direction of all of those edges.
These finite-type Dynkin diagrams correspond to root systems which generate a positive definite bilinear form, giving a genuine inner product.
\begingroup
\allowdisplaybreaks
\begin{align}
    \A[n] &= \qquad
    \tikzsetnextfilename{dynkin-diagram-A}
    \begin{tikzpicture}[dynkin node/.style = {fill}, dynkin wire/.style = {thick}]
        \draw [dynkin node] (0, 0) circle [radius=0.1cm];
        \draw [dynkin wire] (0, 0) -- ++ (2, 0);
        \draw [dynkin node] (1, 0) circle [radius=0.1cm];
        \draw [dynkin node] (2, 0) circle [radius=0.1cm];
        \draw [dynkin wire, dashed] (2, 0) -- ++ (1, 0);
        \draw [dynkin wire] (3, 0) -- (5, 0);
        \draw [dynkin node] (3, 0) circle [radius=0.1cm];
        \draw [dynkin node] (4, 0) circle [radius=0.1cm];
        \draw [dynkin node] (5, 0) circle [radius=0.1cm];
    \end{tikzpicture}
    \\
    \B[n] &= \qquad
    \tikzsetnextfilename{dynkin-diagram-B}
    \begin{tikzpicture}[dynkin node/.style = {fill}, dynkin wire/.style = {thick}]
        \draw [dynkin node] (0, 0) circle [radius=0.1cm];
        \draw [dynkin wire] (0, 0) -- ++ (2, 0);
        \draw [dynkin node] (1, 0) circle [radius=0.1cm];
        \draw [dynkin node] (2, 0) circle [radius=0.1cm];
        \draw [dynkin wire, dashed] (2, 0) -- ++ (1, 0);
        \draw [dynkin wire] (3, 0) -- (4, 0);
        \draw [dynkin wire] (4, -0.05) -- ++ (1, 0);
        \draw [dynkin wire] (4, 0.05) -- ++ (1, 0);
        \draw [dynkin node] (3, 0) circle [radius=0.1cm];
        \draw [dynkin node] (4, 0) circle [radius=0.1cm];
        \draw [dynkin node] (5, 0) circle [radius=0.1cm];
        \draw [-{>[width=0.3cm, length=0.15cm]}, dynkin wire] (4.55, 0) -- ++ (0.001, 0);
    \end{tikzpicture}
    \\
    \C[n] &= \qquad
    \tikzsetnextfilename{dynkin-diagram-C}
    \begin{tikzpicture}[dynkin node/.style = {fill}, dynkin wire/.style = {thick}]
        \draw [dynkin node] (0, 0) circle [radius=0.1cm];
        \draw [dynkin wire] (0, 0) -- ++ (2, 0);
        \draw [dynkin node] (1, 0) circle [radius=0.1cm];
        \draw [dynkin node] (2, 0) circle [radius=0.1cm];
        \draw [dynkin wire, dashed] (2, 0) -- ++ (1, 0);
        \draw [dynkin wire] (3, 0) -- (4, 0);
        \draw [dynkin wire] (4, -0.05) -- ++ (1, 0);
        \draw [dynkin wire] (4, 0.05) -- ++ (1, 0);
        \draw [dynkin node] (3, 0) circle [radius=0.1cm];
        \draw [dynkin node] (4, 0) circle [radius=0.1cm];
        \draw [dynkin node] (5, 0) circle [radius=0.1cm];
        \draw [-{>[width=0.3cm, length=0.15cm]}, dynkin wire] (4.45, 0) -- ++ (-0.001, 0);
    \end{tikzpicture}
    \\
    \D[n] &= \qquad
    \tikzsetnextfilename{dynkin-diagram-D}
    \begin{tikzpicture}[dynkin node/.style = {fill}, dynkin wire/.style = {thick}, baseline=-0.1cm]
        \draw [dynkin node] (0, 0) circle [radius=0.1cm];
        \draw [dynkin wire] (0, 0) -- ++ (2, 0);
        \draw [dynkin node] (1, 0) circle [radius=0.1cm];
        \draw [dynkin node] (2, 0) circle [radius=0.1cm];
        \draw [dynkin wire, dashed] (2, 0) -- ++ (1, 0);
        \draw [dynkin wire] (3, 0) -- (4, 0);
        \draw [dynkin wire] (4, 0) -- ++ (60:1) coordinate (A);
        \draw [dynkin wire] (4, 0) -- ++ (-60:1) coordinate (B);
        \draw [dynkin node] (3, 0) circle [radius=0.1cm];
        \draw [dynkin node] (4, 0) circle [radius=0.1cm];
        \draw [dynkin node] (A) circle [radius=0.1cm];
        \draw [dynkin node] (B) circle [radius=0.1cm];
    \end{tikzpicture}
    \\
    \E[6] &= \qquad
    \tikzsetnextfilename{dynkin-diagram-E6}
    \begin{tikzpicture}[dynkin node/.style = {fill}, dynkin wire/.style = {thick}, baseline=-0.1cm]
        \draw [dynkin wire] (0, 0) -- ++ (4, 0);
        \draw [dynkin wire] (2, 0) -- ++ (0, 1);
        \draw [dynkin node] (0, 0) circle [radius=0.1cm];
        \draw [dynkin node] (1, 0) circle [radius=0.1cm];
        \draw [dynkin node] (2, 0) circle [radius=0.1cm];
        \draw [dynkin node] (3, 0) circle [radius=0.1cm];
        \draw [dynkin node] (4, 0) circle [radius=0.1cm];
        \draw [dynkin node] (2, 1) circle [radius=0.1cm];
    \end{tikzpicture}
    \\
    \E[7] &= \qquad
    \tikzsetnextfilename{dynkin-diagram-E7}
    \begin{tikzpicture}[dynkin node/.style = {fill}, dynkin wire/.style = {thick}, baseline=-0.1cm]
        \draw [dynkin wire] (0, 0) -- ++ (5, 0);
        \draw [dynkin wire] (2, 0) -- ++ (0, 1);
        \draw [dynkin node] (0, 0) circle [radius=0.1cm];
        \draw [dynkin node] (1, 0) circle [radius=0.1cm];
        \draw [dynkin node] (2, 0) circle [radius=0.1cm];
        \draw [dynkin node] (3, 0) circle [radius=0.1cm];
        \draw [dynkin node] (4, 0) circle [radius=0.1cm];
        \draw [dynkin node] (5, 0) circle [radius=0.1cm];
        \draw [dynkin node] (2, 1) circle [radius=0.1cm];
    \end{tikzpicture}
    \\
    \E[8] &= \qquad
    \tikzsetnextfilename{dynkin-diagram-E8}
    \begin{tikzpicture}[dynkin node/.style = {fill}, dynkin wire/.style = {thick}, baseline=-0.1cm]
        \draw [dynkin wire] (0, 0) -- ++ (6, 0);
        \draw [dynkin wire] (2, 0) -- ++ (0, 1);
        \draw [dynkin node] (0, 0) circle [radius=0.1cm];
        \draw [dynkin node] (1, 0) circle [radius=0.1cm];
        \draw [dynkin node] (2, 0) circle [radius=0.1cm];
        \draw [dynkin node] (3, 0) circle [radius=0.1cm];
        \draw [dynkin node] (4, 0) circle [radius=0.1cm];
        \draw [dynkin node] (5, 0) circle [radius=0.1cm];
        \draw [dynkin node] (6, 0) circle [radius=0.1cm];
        \draw [dynkin node] (2, 1) circle [radius=0.1cm];
    \end{tikzpicture}
    \\
    \G[2] &= \qquad
    \tikzsetnextfilename{dynkin-diagram-G2}
    \begin{tikzpicture}[dynkin node/.style = {fill}, dynkin wire/.style = {thick}]
        \draw [dynkin wire] (0, 0) -- ++ (1, 0);
        \draw [dynkin wire] (0, -0.05) -- ++ (1, 0);
        \draw [dynkin wire] (0, 0.05) -- ++ (1, 0);
        \draw [dynkin node] (0, 0) circle [radius=0.1cm];
        \draw [dynkin node] (1, 0) circle [radius=0.1cm];
        \draw [-{>[width=0.3cm, length=0.15cm]}, dynkin wire] (0.55, 0) -- ++ (0.001, 0);
    \end{tikzpicture}
    \\
    \F[4] &= \qquad
    \tikzsetnextfilename{dynkin-diagram-F4}
    \begin{tikzpicture}[dynkin node/.style = {fill}, dynkin wire/.style = {thick}]
        \draw [dynkin wire] (0, 0) -- ++ (1, 0);
        \draw [dynkin wire] (1, -0.05) -- ++ (1, 0);
        \draw [dynkin wire] (1, 0.05) -- ++ (1, 0);
        \draw [dynkin wire] (2, 0) -- ++ (1, 0);
        \draw [dynkin node] (0, 0) circle [radius=0.1cm];
        \draw [dynkin node] (1, 0) circle [radius=0.1cm];
        \draw [dynkin node] (2, 0) circle [radius=0.1cm];
        \draw [dynkin node] (3, 0) circle [radius=0.1cm];
        \draw [-{>[width=0.3cm, length=0.15cm]}, dynkin wire] (1.55, 0) -- ++ (0.001, 0);
    \end{tikzpicture}
\end{align}
\endgroup
Note that some of these are the same, we have the following graph isomorphisms:
\begin{itemize}
    \item \(\A[1] \isomorphic \B[1] \isomorphic \C[1] \isomorphic \D[1]\), all of which are just a single node;
    \item \(\B[2] \isomorphic \C[2]\), both of which are two nodes connected by a double directed edge;
    \item \(\A[3] \isomorphic \D[3]\), both of which are three nodes connected in a row by single edges;
\end{itemize}
Note that the sensible definition of \(\D[2]\) is two disconnected notes, just the ones appearing in the far right of the picture above.
This gives us the graph isomorphism \(\D[2] \isomorphic \A[1] \sqcup \A[1]\).

There are two ways around these isomorphisms, one is to restrict the valid indices as follows:
\begin{itemize}
    \item for \(\B[n]\) we require \(n \ge 2\);
    \item for \(\C[n]\) we require \(n \ge 3\);
    \item for \(\D[n]\) we require \(n \ge 4\).
\end{itemize}
The other is to just allow these, but to keep track of whether we think of one of these graphs as being part of, say the \(\A\) or \(\D\) series.
I prefer this second approach, since it will give us interesting isomorphisms between the objects we classify.
These so-called exceptional isomorphisms are best known on the level of Lie algebras, where they correspond to
\begin{itemize}
    \item \(\specialLinearLie_2 \isomorphic \specialOrthogonalLie_3 \isomorphic \symplecticLie_2\);
    \item \(\specialOrthogonalLie_5 \isomorphic \symplecticLie_4\);
    \item \(\specialLinearLie_4 \isomorphic \specialOrthogonalLie_6\);
    \item \(\specialLinearLie_2 \oplus \specialLinearLie_2 \isomorphic \specialOrthogonalLie_2\).
\end{itemize}

\subsection{Coxeter Diagrams}
The \define{Coxeter diagrams} are similar to Dynkin diagrams, and the two are often conflated.
Coxeter diagrams are undirected graphs, with edge labellings.
The rule is that any unlabelled edges are labelled \(3\), and any two nodes not connected by an edge actually have an undrawn edge labelled \(2\).
Every Dynkin diagram has, ignoring ordering, a corresponding Coxeter diagram, in which the directed edges are replaced by labelled edges with labels greater than \(3\).
There are also some Coxeter diagrams which don't have corresponding Dynkin diagrams.
We will see why these distinctions are made later, for now we just list the Coxeter diagrams.
The Coxeter diagrams are as below, they have the same labelling as the corresponding Dynkin diagrams, further confusing the two.
\begingroup
\allowdisplaybreaks
\begin{align}
    \A[n] &= \qquad
    \tikzsetnextfilename{coxeter-diagram-A}
    \begin{tikzpicture}[coxeter node/.style = {fill, inner sep=3pt}, coxeter wire/.style = {thick}]
        \draw [coxeter wire] (0, 0) -- ++ (2, 0);
        \draw [coxeter wire, dashed] (1.9, 0) -- ++ (1.05, 0);
        \draw [coxeter wire] (3, 0) -- (5, 0);
        \node [coxeter node] at (0, 0) {};
        \node [coxeter node] at (1, 0) {};
        \node [coxeter node] at (2, 0) {};
        \node [coxeter node] at (3, 0) {};
        \node [coxeter node] at (4, 0) {};
        \node [coxeter node] at (5, 0) {};
    \end{tikzpicture}
    \\
    \B[n] = \C[n] &= \qquad
    \tikzsetnextfilename{coxeter-diagram-BC}
    \begin{tikzpicture}[coxeter node/.style = {fill, inner sep=3pt}, coxeter wire/.style = {thick}]
        \draw [coxeter wire] (0, 0) -- ++ (2, 0);
        \draw [coxeter wire, dashed] (1.9, 0) -- ++ (1.05, 0);
        \draw [coxeter wire] (3, 0) -- (5, 0);
        \node [coxeter node] at (0, 0) {};
        \node [coxeter node] at (1, 0) {};
        \node [coxeter node] at (2, 0) {};
        \node [coxeter node] at (3, 0) {};
        \node [coxeter node] at (4, 0) {};
        \node [coxeter node] at (5, 0) {};
        \node [above] at (4.5, 0) {\(4\)};
    \end{tikzpicture}
    \\
    \D[n] &= \qquad
    \tikzsetnextfilename{coxeter-diagram-D}
    \begin{tikzpicture}[coxeter node/.style = {fill, inner sep=3pt}, coxeter wire/.style = {thick}, baseline=-0.1cm]
        \draw [coxeter wire] (3, 0) -- (4, 0);
        \draw [coxeter wire] (4, 0) -- ++ (60:1) coordinate (A);
        \draw [coxeter wire] (4, 0) -- ++ (-60:1) coordinate (B);
        \draw [coxeter wire] (0, 0) -- ++ (2, 0);
        \draw [coxeter wire, dashed] (2, 0) -- ++ (1, 0);
        \node [coxeter node] at (0, 0) {};
        \node [coxeter node] at (1, 0) {};
        \node [coxeter node] at (2, 0) {};
        \node [coxeter node] at (3, 0) {};
        \node [coxeter node] at (4, 0) {};
        \node [coxeter node] at (A) {};
        \node [coxeter node] at (B) {};
    \end{tikzpicture}
    \\
    \E[6] &= \qquad
    \tikzsetnextfilename{coxeter-diagram-E6}
    \begin{tikzpicture}[coxeter node/.style = {fill, inner sep=3pt}, coxeter wire/.style = {thick}, baseline=-0.1cm]
        \draw [coxeter wire] (0, 0) -- ++ (4, 0);
        \draw [coxeter wire] (2, 0) -- ++ (0, 1);
        \node [coxeter node] at (0, 0) {};
        \node [coxeter node] at (1, 0) {};
        \node [coxeter node] at (2, 0) {};
        \node [coxeter node] at (3, 0) {};
        \node [coxeter node] at (4, 0) {};
        \node [coxeter node] at (2, 1) {};
    \end{tikzpicture}
    \\
    \E[7] &= \qquad
    \tikzsetnextfilename{coxeter-diagram-E7}
    \begin{tikzpicture}[coxeter node/.style = {fill, inner sep=3pt}, coxeter wire/.style = {thick}, baseline=-0.1cm]
        \draw [coxeter wire] (0, 0) -- ++ (5, 0);
        \draw [coxeter wire] (2, 0) -- ++ (0, 1);
        \node [coxeter node] at (0, 0) {};
        \node [coxeter node] at (1, 0) {};
        \node [coxeter node] at (2, 0) {};
        \node [coxeter node] at (3, 0) {};
        \node [coxeter node] at (4, 0) {};
        \node [coxeter node] at (5, 0) {};
        \node [coxeter node] at (2, 1) {};
    \end{tikzpicture}
    \\
    \E[8] &= \qquad
    \tikzsetnextfilename{coxeter-diagram-E8}
    \begin{tikzpicture}[coxeter node/.style = {fill, inner sep=3pt}, coxeter wire/.style = {thick}, baseline=-0.1cm]
        \draw [coxeter wire] (0, 0) -- ++ (6, 0);
        \draw [coxeter wire] (2, 0) -- ++ (0, 1);
        \node [coxeter node] at (0, 0) {};
        \node [coxeter node] at (1, 0) {};
        \node [coxeter node] at (2, 0) {};
        \node [coxeter node] at (3, 0) {};
        \node [coxeter node] at (4, 0) {};
        \node [coxeter node] at (5, 0) {};
        \node [coxeter node] at (6, 0) {};
        \node [coxeter node] at (2, 1) {};
    \end{tikzpicture}
    \\
    \G[2] &= \qquad
    \tikzsetnextfilename{coxeter-diagram-G2}
    \begin{tikzpicture}[coxeter node/.style = {fill, inner sep=3pt}, coxeter wire/.style = {thick}]
        \draw [coxeter wire] (0, 0) -- ++ (1, 0);
        \node [above] at (0.5, 0) {\(6\)};
        \node [coxeter node] at (0, 0) {};
        \node [coxeter node] at (1, 0) {};
    \end{tikzpicture}
    \\
    \F[4] &= \qquad
    \tikzsetnextfilename{coxeter-diagram-F4}
    \begin{tikzpicture}[coxeter node/.style = {fill, inner sep=3pt}, coxeter wire/.style = {thick}]
        \draw [coxeter wire] (0, 0) -- ++ (3, 0);
        \node [coxeter node] at (0, 0) {};
        \node [coxeter node] at (1, 0) {};
        \node [coxeter node] at (2, 0) {};
        \node [coxeter node] at (3, 0) {};
        \node [above] at (1.5, 0) {\(4\)};
    \end{tikzpicture}
    \\
    \H[2] &= \qquad
    \tikzsetnextfilename{coxeter-diagram-H2}
    \begin{tikzpicture}[coxeter node/.style = {fill, inner sep=3pt}, coxeter wire/.style = {thick}]
        \draw [coxeter wire] (0, 0) -- ++ (1, 0);
        \node [above] at (0.5, 0) {\(5\)};
        \node [coxeter node] at (0, 0) {};
        \node [coxeter node] at (1, 0) {};
    \end{tikzpicture}
    \\
    \H[3] &= \qquad
    \tikzsetnextfilename{coxeter-diagram-H3}
    \begin{tikzpicture}[coxeter node/.style = {fill, inner sep=3pt}, coxeter wire/.style = {thick}]
        \draw [coxeter wire] (0, 0) -- ++ (2, 0);
        \node [above] at (0.5, 0) {\(5\)};
        \node [coxeter node] at (0, 0) {};
        \node [coxeter node] at (1, 0) {};
        \node [coxeter node] at (2, 0) {};
    \end{tikzpicture}
    \\
    \H[4] &= \qquad
    \tikzsetnextfilename{coxeter-diagram-H4}
    \begin{tikzpicture}[coxeter node/.style = {fill, inner sep=3pt}, coxeter wire/.style = {thick}]
        \draw [coxeter wire] (0, 0) -- ++ (3, 0);
        \node [above] at (0.5, 0) {\(5\)};
        \node [coxeter node] at (0, 0) {};
        \node [coxeter node] at (1, 0) {};
        \node [coxeter node] at (2, 0) {};
        \node [coxeter node] at (3, 0) {};
    \end{tikzpicture}
    \\
    \I[2][m] &= \qquad
    \tikzsetnextfilename{coxeter-diagram-I2}
    \begin{tikzpicture}[coxeter node/.style = {fill, inner sep=3pt}, coxeter wire/.style = {thick}]
        \draw [coxeter wire] (0, 0) -- ++ (1, 0);
        \node [above] at (0.5, 0) {\(m\)};
        \node [coxeter node] at (0, 0) {};
        \node [coxeter node] at (1, 0) {};
    \end{tikzpicture}
\end{align}
\endgroup

As with the Dynkin diagrams there are certain graph isomorphisms here for specific values of indices.
In addition to the isomorphisms of the Dynkin diagrams we also have the following:
\begin{itemize}
    \item \(\G[2] \isomorphic \I[2][6]\);
    \item \(\H[2] \isomorphic \I[2][5]\).
\end{itemize}

\subsection{Affine Coxeter Diagrams}
The final type of diagram we'll need are the affine Coxeter diagrams.
We won't be particularly interested in the structures that these classify, they tend to be infinite versions of the finite versions the finite types generate, but they are needed to prove the classification of Coxeter groups.
These affine types are formed from the finite types by adding another node.
There are generally many ways to do this, but most end up giving isomorhpic graphs.
In fact, for every outer automorphism of the original graph there is a unique (up to isomorhpism) way to add an extra node.
We will give only the simplest here, which are the ones corresponding to the identity automorphism.
The other types are called \enquote{twisted} affine Coxeter graphs.
There are two common notations for the affine Coxeter graph, they are \(\affA[n]\) and \(\A[n]^{(1)}\).
The advantage of the second is that the superscript can index the corresponding outer automorhpism, since we're not interested in the twisted case the tilde will be sufficient for our purposes.

Here are all the untwisted affine Coxeter graphs.
The new node is the one in colour.
Note that the index is now one fewer than the number of nodes.
\begingroup
\allowdisplaybreaks
\begin{align}
    \affA[1] &= \qquad
    \tikzsetnextfilename{affine-coxeter-diagram-A1}
    \begin{tikzpicture}[coxeter node/.style = {fill, inner sep=3pt}, coxeter wire/.style = {thick}, baseline=-0.1cm]
        \draw [coxeter wire] (0, 0) -- (1, 0);
        \node [above] at (0.5, 0) {\(\infty\)};
        \node [coxeter node] at (0, 0) {};
        \node [coxeter node, dark orange] at (1, 0) {};
    \end{tikzpicture}
    \\
    \affA[n] &= \qquad
    \tikzsetnextfilename{affine-coxeter-diagram-A}
    \begin{tikzpicture}[coxeter node/.style = {fill, inner sep=3pt}, coxeter wire/.style = {thick}]
        \draw [coxeter wire] (0, 0) -- ++ (2, 0);
        \draw [coxeter wire, dashed] (1.9, 0) -- ++ (1.05, 0);
        \draw [coxeter wire] (2.5, 1.5) -- (0, 0);
        \draw [coxeter wire] (2.5, 1.5) -- (5, 0);
        \draw [coxeter wire] (3, 0) -- (5, 0);
        \node [coxeter node] at (0, 0) {};
        \node [coxeter node] at (1, 0) {};
        \node [coxeter node] at (2, 0) {};
        \node [coxeter node] at (3, 0) {};
        \node [coxeter node] at (4, 0) {};
        \node [coxeter node] at (5, 0) {};
        \node [coxeter node, dark orange] at (2.5, 1.5) {};
    \end{tikzpicture}
    \\
    \affB[n] &= \qquad
    \tikzsetnextfilename{affine-coxeter-diagram-B}
    \begin{tikzpicture}[coxeter node/.style = {fill, inner sep=3pt}, coxeter wire/.style = {thick}]
        \draw [coxeter wire, dashed] (1.9, 0) -- ++ (1.05, 0);
        \draw [coxeter wire] (1, 0) -- (2, 0);
        \draw [coxeter wire] (3, 0) -- (5, 0);
        \draw [coxeter wire] (1, 0) -- ++ (120:1) coordinate (A);
        \draw [coxeter wire] (1, 0) -- ++ (-120:1) coordinate (B);
        \node [coxeter node] at (A) {};
        \node [coxeter node, dark orange] at (B) {};
        \node [coxeter node] at (1, 0) {};
        \node [coxeter node] at (2, 0) {};
        \node [coxeter node] at (3, 0) {};
        \node [coxeter node] at (4, 0) {};
        \node [coxeter node] at (5, 0) {};
        \node [above] at (4.5, 0) {\(4\)};
    \end{tikzpicture}
    \\
    \affC[n] &= \qquad
    \tikzsetnextfilename{affine-coxeter-diagram-C}
    \begin{tikzpicture}[coxeter node/.style = {fill, inner sep=3pt}, coxeter wire/.style = {thick}]
        \draw [coxeter wire] (-1, 0) -- ++ (3, 0);
        \draw [coxeter wire, dashed] (1.9, 0) -- ++ (1.05, 0);
        \draw [coxeter wire] (3, 0) -- (5, 0);
        \node [coxeter node, dark orange] at (-1, 0) {};
        \node [coxeter node] at (0, 0) {};
        \node [coxeter node] at (1, 0) {};
        \node [coxeter node] at (2, 0) {};
        \node [coxeter node] at (3, 0) {};
        \node [coxeter node] at (4, 0) {};
        \node [coxeter node] at (5, 0) {};
        \node [above] at (4.5, 0) {\(4\)};
        \node [above] at (-0.5, 0) {\(4\)};
    \end{tikzpicture}
    \\
    \D[n] &= \qquad
    \tikzsetnextfilename{affine-coxeter-diagram-D}
    \begin{tikzpicture}[coxeter node/.style = {fill, inner sep=3pt}, coxeter wire/.style = {thick}, baseline=-0.1cm]
        \draw [coxeter wire] (3, 0) -- (4, 0);
        \draw [coxeter wire] (4, 0) -- ++ (60:1) coordinate (C);
        \draw [coxeter wire] (4, 0) -- ++ (-60:1) coordinate (D);
        \draw [coxeter wire] (1, 0) -- (2, 0);
        \draw [coxeter wire] (3, 0) -- (4, 0);
        \draw [coxeter wire] (1, 0) -- ++ (120:1) coordinate (A);
        \draw [coxeter wire] (1, 0) -- ++ (-120:1) coordinate (B);
        \node [coxeter node] at (A) {};
        \node [coxeter node, dark orange] at (B) {};
        \draw [coxeter wire, dashed] (2, 0) -- ++ (1, 0);
        \node [coxeter node] at (1, 0) {};
        \node [coxeter node] at (2, 0) {};
        \node [coxeter node] at (3, 0) {};
        \node [coxeter node] at (4, 0) {};
        \node [coxeter node] at (C) {};
        \node [coxeter node] at (D) {};
    \end{tikzpicture}
    \\
    \E[6] &= \qquad
    \tikzsetnextfilename{affine-coxeter-diagram-E6}
    \begin{tikzpicture}[coxeter node/.style = {fill, inner sep=3pt}, coxeter wire/.style = {thick}, baseline=-0.1cm]
        \draw [coxeter wire] (0, 0) -- ++ (4, 0);
        \draw [coxeter wire] (2, 0) -- ++ (0, 2);
        \node [coxeter node] at (0, 0) {};
        \node [coxeter node] at (1, 0) {};
        \node [coxeter node] at (2, 0) {};
        \node [coxeter node] at (3, 0) {};
        \node [coxeter node] at (4, 0) {};
        \node [coxeter node] at (2, 1) {};
        \node [coxeter node, dark orange] at (2, 2) {};
    \end{tikzpicture}
    \\
    \E[7] &= \qquad
    \tikzsetnextfilename{affine-coxeter-diagram-E7}
    \begin{tikzpicture}[coxeter node/.style = {fill, inner sep=3pt}, coxeter wire/.style = {thick}, baseline=-0.1cm]
        \draw [coxeter wire] (-1, 0) -- ++ (6, 0);
        \draw [coxeter wire] (2, 0) -- ++ (0, 1);
        \node [coxeter node] at (0, 0) {};
        \node [coxeter node] at (1, 0) {};
        \node [coxeter node] at (2, 0) {};
        \node [coxeter node] at (3, 0) {};
        \node [coxeter node] at (4, 0) {};
        \node [coxeter node] at (5, 0) {};
        \node [coxeter node] at (2, 1) {};
        \node [coxeter node, dark orange] at (-1, 0) {};
    \end{tikzpicture}
    \\
    \E[8] &= \qquad
    \tikzsetnextfilename{affine-coxeter-diagram-E8}
    \begin{tikzpicture}[coxeter node/.style = {fill, inner sep=3pt}, coxeter wire/.style = {thick}, baseline=-0.1cm]
        \draw [coxeter wire] (0, 0) -- ++ (7, 0);
        \draw [coxeter wire] (2, 0) -- ++ (0, 1);
        \node [coxeter node] at (0, 0) {};
        \node [coxeter node] at (1, 0) {};
        \node [coxeter node] at (2, 0) {};
        \node [coxeter node] at (3, 0) {};
        \node [coxeter node] at (4, 0) {};
        \node [coxeter node] at (5, 0) {};
        \node [coxeter node] at (6, 0) {};
        \node [coxeter node] at (2, 1) {};
        \node [coxeter node, dark orange] at (7, 0) {};
    \end{tikzpicture}
    \\
    \G[2] &= \qquad
    \tikzsetnextfilename{affine-coxeter-diagram-G2}
    \begin{tikzpicture}[coxeter node/.style = {fill, inner sep=3pt}, coxeter wire/.style = {thick}]
        \draw [coxeter wire] (0, 0) -- ++ (2, 0);
        \node [above] at (0.5, 0) {\(6\)};
        \node [coxeter node] at (0, 0) {};
        \node [coxeter node] at (1, 0) {};
        \node [coxeter node, dark orange] at (2, 0) {};
    \end{tikzpicture}
    \\
    \F[4] &= \qquad \tikzexternaldisable
    \tikzsetnextfilename{affine-coxeter-diagram-F4}
    \begin{tikzpicture}[coxeter node/.style = {fill, inner sep=3pt}, coxeter wire/.style = {thick}]
        \draw [coxeter wire] (0, 0) -- ++ (4, 0);
        \node [coxeter node] at (0, 0) {};
        \node [coxeter node] at (1, 0) {};
        \node [coxeter node] at (2, 0) {};
        \node [coxeter node] at (3, 0) {};
        \node [above] at (1.5, 0) {\(4\)};
        \node [coxeter node, dark orange] at (4, 0) {};
    \end{tikzpicture}
\end{align}
\endgroup
First note that \(\affA[1]\) is \enquote{weird}, it's sort of a degenerate case in which adding the extra node as we would for \(\affA[n]\) gives us two nodes with edges between them and this \(2\)-cycle somehow explodes into an infinite loop.
Note also that \(\affB[n]\) and \(\affC[n]\) are different Coxeter diagrams, even though \(\B[n]\) and \(\C[n]\) are the same Coxeter diagram.
    
    \part{Coxeter Groups}
    \chapter{Reflection Groups}
    The recommended reference for this topic is \cite{Humphreys.CoxeterGroups}.
    Throughout this part of the notes \(V\) will always be a real Euclidean space.
    That is, \(V\) is equipped with some positive definite symmetric bilinear form, \(\innerprod{-}{-} \colon V \times V \to \reals\).
    
    \section{Definition}
    \begin{dfn}{Reflection}{}
        Fix some \(\alpha \in V\).
        We say that \(s \in \generalLinear(V)\) is a \define{reflection} along \(\alpha \in V \setminus 0\) if \(s(\alpha) = -\alpha\) and the hyperplane orthogonal to \(\alpha\),
        \begin{equation}
            H_\alpha \coloneq (\reals \alpha)^{\perp} = \{\lambda \in \mid \innerprod{\alpha}{\lambda} = 0\},
        \end{equation}
        is fixed pointwise, so \(s(\lambda) = \lambda\) for \(\lambda \in H_\alpha\), in other words \(s\) acts as the identity on \(H_\alpha\), so \(s|_{H_\alpha} = \id_{H_\alpha}\).
    \end{dfn}
    
    \begin{lma}{}{}
        The map defined by
        \begin{equation}
            s_\alpha(\lambda) = \lambda - 2\frac{\innerprod{\alpha}{\lambda}}{\innerprod{\alpha}{\alpha}} \alpha
        \end{equation}
        is a reflection in \(H_\alpha\).
        \begin{proof}
            First note that \(s_\alpha\) is clearly linear
            Then we have
            \begin{equation}
                s_\alpha(\alpha) = \alpha - 2\frac{\innerprod{\alpha}{\alpha}}{\innerprod{\alpha}{\alpha}} \alpha = -\alpha
            \end{equation}
            and if \(\lambda \in H_\alpha\), so \(\innerprod{\alpha}{\lambda} = 0\), we simply have
            \begin{equation}
                s_\alpha(\lambda) = \lambda - 2\frac{\innerprod{\alpha}{\lambda}}{\innerprod{\alpha}{\lambda}} = \lambda.
            \end{equation}
            Finally, \(s_\alpha\) is its own inverse:
            \begin{align}
                s_\alpha(s_\alpha(\lambda)) &= s_\alpha\left( \lambda - 2\frac{\innerprod{\alpha}{\lambda}}{\innerprod{\alpha}{\alpha}}\alpha \right)\\
                &= s_\alpha(\lambda) - 2\frac{\innerprod{\alpha}{\lambda}}{\innerprod{\alpha}{\alpha}}s_\alpha(\alpha)\\
                &= \lambda - 2\frac{\innerprod{\alpha}{\lambda}}{\innerprod{\alpha}{\alpha}}\alpha + 2\frac{\innerprod{\alpha}{\lambda}}{\innerprod{\alpha}{\alpha}}\alpha\\
                &= \lambda. \notag \qedhere
            \end{align}
        \end{proof}
    \end{lma}
    
    Notice that by the definition of \(H_\alpha\) we can always decompose our space as \(V = H_\alpha \oplus \reals \alpha\), so the information about what happens to \(\alpha\) and anything orthogonal to it is enough to fully determine \(s_\alpha\).
    Further, note that any nonzero scalar multiple of \(\alpha\) also determines the same reflection.
    All reflections may be specified in this way, and we'll simply write \(s_\alpha\) for the reflection in the hyperplane orthogonal to \(\alpha\).
    
    \begin{lma}{}{}
        Reflections are orthogonal transformations.
        \begin{proof}
            To show that \(s_\alpha \in \orthogonal(V)\) we need to show that \(\innerprod{s_\alpha(\lambda)}{s_\alpha(\mu)} = \innerprod{\lambda}{\mu}\) for all \(\lambda, \mu \in V\).
            We can do this with the explicit formula above:
            \begin{align}
                \innerprod{s_\alpha(\lambda)}{s_\alpha(\mu)} &= \innerprod*{\lambda - \frac{\innerprod{\alpha}{\lambda}}{\innerprod{\alpha}{\alpha}}\alpha}{\mu - \frac{\innerprod{\alpha}{\mu}}{\innerprod{\alpha}{\alpha}}\alpha}\\
                &= \innerprod{\lambda}{\mu} - 2\frac{\innerprod{\alpha}{\lambda}}{\innerprod{\alpha}{\alpha}}\innerprod{\alpha}{\mu} - 2\frac{\innerprod{\alpha}{\mu}}{\innerprod{\alpha}{\alpha}}\innerprod{\lambda}{\alpha} \\
                &\qquad+ 4\frac{\innerprod{\alpha}{\lambda} \innerprod{\alpha}{\mu}}{\innerprod{\alpha}{\alpha}^2} \innerprod{\alpha}{\alpha} \notag\\
                &= \innerprod{\lambda}{\mu}. \notag\qedhere
            \end{align}
        \end{proof}
    \end{lma}
    
    \begin{dfn}{Reflection Group}{}
        A \define{reflection group} is a group, \(W\), is a subgroup of \(\orthogonal(V)\) generated by reflections.
    \end{dfn}
    
    We will almost entirely restrict our study to finite reflection groups, where \(W\) is a finite group.
    In this part of the notes \(W\) is a finite reflection group unless stated otherwise.
    
    We will see that finite reflection groups are classified by finite type Dynkin diagrams.
    We will write \(W(R)\) for the finite reflection group corresponding to the Dynkin type \(R\) under this correspondence.
    
    \section{Examples}
    \subsection{Dihedral Groups: \texorpdfstring{\(\I[2][m]\)}{I2m}}
    Consider the dihedral group, \(W(\I[2][m]) = \Dih_{m}\), of symmetries of the regular \(m\)-gon.
    Note that this group has order \(2m\).
    We will illustrate things with \(\Dih_5\), the symmetries of the regular pentagon, as seen in \cref{fig:regular pentagon reflections}.
    With \(V = \reals^2\) we can naturally extend the symmetries of the regular \(m\)-gon (centred on the origin) to the entire plane.
    A rotation by \(2\pi/m\) can be generated by two reflections.
    For example, \cref{fig:regular pentagon reflections generate rotations} shows two reflections generating a rotation of \(2\pi/5\) by applying two rotations, and from this we can generate all rotations.
    
    \begin{figure}
        \tikzsetnextfilename{regular-pentagon-reflections}
        \centering
        \begin{tikzpicture}
            \draw [very thick, rotate=90] (0:2) -- (72:2) -- (144:2) -- (216:2) -- (288:2) -- cycle;
            \draw [thick, dashed, gray, rotate=90] (36.5:-3) -- (36.5:3);
            \draw [thick, dashed, gray, rotate=90+72] (36.5:-3) -- (36.5:3);
            \draw [thick, dashed, gray, rotate=90+2*72] (36.5:-3) -- (36.5:3);
            \draw [thick, dashed, gray, rotate=90+3*72] (36.5:-3) -- (36.5:3);
            \draw [thick, dashed, gray, rotate=90+4*72] (36.5:-3) -- (36.5:3);
        \end{tikzpicture}
        \caption[Reflections of the pentagon.]{The regular pentagon. A reflection in any of the five dashed lines is a symmetry.}
        \label{fig:regular pentagon reflections}
    \end{figure}
    
    \begin{figure}
        \tikzsetnextfilename{regular-pentagon-reflections-generate-rotations}
        \centering
        \begin{tikzpicture}[scale=0.7]
            \draw [very thick, rotate=90] (0:2) -- (72:2) -- (144:2) -- (216:2) -- (288:2) -- cycle;
            \draw [ultra thick, rotate=90, dark orange, cap=round] (144:2) -- (216:2);
            \draw [ultra thick, rotate=90, light orange, cap=round] (216:2) -- (288:2);
            \draw [thick, dashed, gray, rotate=90+72] (36.5:-2.5) -- (36.5:2.5);
            \begin{scope}[xshift=6cm]
                \draw [very thick, rotate=90] (0:2) -- (72:2) -- (144:2) -- (216:2) -- (288:2) -- cycle;
                \draw [ultra thick, rotate=90, dark orange, cap=round] (0:2) -- (72:2);
                \draw [ultra thick, rotate=90, light orange, cap=round] (288:2) -- (0:2);
                \draw [thick, dashed, gray, rotate=90+3*72] (36.5:-2.5) -- (36.5:2.5);
            \end{scope}
            \begin{scope}[xshift=12cm]
                \draw [very thick, rotate=90] (0:2) -- (72:2) -- (144:2) -- (216:2) -- (288:2) -- cycle;
                \draw [ultra thick, rotate=90, dark orange, cap=round] (72:2) -- (144:2);
                \draw [ultra thick, rotate=90, light orange, cap=round] (144:2) -- (216:2);
            \end{scope}
        \end{tikzpicture}
        \caption[Reflections generate rotations.]{Successive reflections generate a rotation by \(2\pi/5\). The coloured edges are just there to help visualise the reflections occurring.}
        \label{fig:regular pentagon reflections generate rotations}
    \end{figure}
    
    \subsection{Platonic Solids: \texorpdfstring{\(\A[3]\)}{A3}, \texorpdfstring{\(\B[3]\)}{B3}, \texorpdfstring{\(\H[3]\)}{H3}}
    The platonic solids, the simplex, cube, octahedron, dodecahedron, and icosahedron, are the strictest generalisation of regular shapes to three dimensions.
    Like the regular \(m\)-gons the symmetry groups of the platonic solids are all reflection groups upon extending the action from the polyhedron to all of \(\reals^3\).
    
    The 3-simplex, also known as the tetrahedron, has symmetry group \(W(\A[3]) = \Sym_4\), which simply acts by permuting the faces.
    
    The cube and its dual, the octahedron, have \(\Sym_4 \times \integers_2\).
    The \(\integers_2\) corresponds to a reflection in some chosen plane through three non-adjacent vertices, inverting the cube.
    The action of \(\Sym_4\) is to permute the diagonals, which are the lines that are invariant under the different choices of inversion above.
    
    However, it turns out that if we want to generalise in the future it is actually better to think of the symmetries of the cube as being \(W(\B[3]) = \Sym_3 \ltimes \integers_2^3\).
    Here \(\Sym_3\) acts on \(\reals^3\) by permuting the standard basis, and each copy of \(\integers_2\) acts as a reflection in one of the coordinate planes.
    Note that in this description the cube is thought of as \([-1, 1]^3\), centred on the origin.
    
    The dodecahedron has symmetry group \(W(\H[3])\), of order \(120\).
    It's a bit complicated to specify how it acts here, and it also doesn't generalise so we won't go into details.
    
    \subsection{Simplices and Permutations: \texorpdfstring{\(\A[n]\)}{An}}
    The \(n\)-simplex is the convex hull of the \(n\)-basis vectors of \(\reals^{n+1}\).
    This is perhaps a confusing definition, but hopefully the pictures of the \(1\)-, and \(2\)-simplices in \cref{fig:simplices} help.
    The \(1\)-simplex is simply a line segment, the \(2\)-segment is an equilateral triangle, and the \(3\)-simplex is the regular tetrahedron.
    The symmetries of the \(n\)-simplex are simply given by permuting \(n\)-cells (faces of the tetrahedron, edges of the triangle, the two end points of the line), and there are always \(n + 1\) \(n\)-cells.
    The symmetry group of the \(n\)-simplex is thus \(W(\A[n]) = \Sym_{n+1}\).
    
    \begin{figure}
        \centering
        \tikzsetnextfilename{simplices}
        \begin{tikzpicture}
            \draw [thick, ->] (0, 0) -- (2, 0);
            \draw [thick, ->] (0, 0) -- (0, 2);
            \draw [light orange] (1, 0) -- (0, 1);
            \fill [dark orange] (1, 0) circle [radius=0.1cm];
            \fill [dark orange] (0, 1) circle [radius=0.1cm];
            
            \begin{scope}[xshift=5cm]
                \draw [thick, ->] (0, 0, 0) -- (2, 0, 0);
                \draw [thick, ->] (0, 0, 0) -- (0, 2, 0);
                \draw [thick, ->] (0, 0, 0) -- (0, 0, 2);
                \draw [thick, dark orange, fill=light orange, opacity=0.5] (1, 0, 0) -- (0, 1, 0) -- (0, 0, 1) -- cycle;
                \fill [dark orange] (1, 0, 0) circle [radius=0.1cm];
                \fill [dark orange] (0, 1, 0) circle [radius=0.1cm];
                \fill [dark orange] (0, 0, 1) circle [radius=0.1cm];
            \end{scope}
        \end{tikzpicture}
        \caption{Simplices: The \(1\)-simplex is just an interval and the \(2\)-simplex is a triangle.}
        \label{fig:simplices}
    \end{figure}
    
    Recall that \(\Sym_{n+1}\) is generated by transpositions of neighbouring elements, \(\cycle{i,i+1}\) for \(1 \le i \le n\).
    When realised as a reflection group the transposition \(\cycle{i,i+1}\) corresponds to the reflection \(s_{e_i - e_{i+1}}\).
    
    \subsection{Signed Permutations: \texorpdfstring{\(\B[n]\)}{Bn}}
    The \(n\)-cube is \([-1, 1]^n \subset \reals^n\).
    The symmetry group of the \(n\)-cube is the group of \emph{signed} permutations of \(n\)-letters.
    A \define{signed permutation} is a permutation of \(\{-n, -n+1, \dotsc, -1, 1, \dotsc, n-1, n\}\) which preserves pairs \(\{i, -i\}\).
    In other words, it's a combination of some permutation of this set satisfying \(w(-i) = -w(i)\).
    The operation of this group is composition of such permutations.
    A common notation for both signed and unsigned permutations, \(w\), is \([w(1), \dotsc, w(n)]\).
    When using this notation it is common to write \(\overbar{i}\) in place of \(-i\).
    
    For example, \([3, 1, 2]\) is the permutation sending \(1\) to \(3\), \(2\) to \(1\), and \(3\) to \(2\).
    This acts trivially on the negative terms, so it's also just a regular permutation
    In cycle notation it is \(\cycle{1,3,2}\).
    Similarly, \([3, \overbar{1}, 2]\) is the signed permutation sending \(1\) to \(3\), \(2\) to \(-1\), and \(3\) to \(2\).
    The action on \(-1\) is \(w(-1) = -w(1) = -3\), the action on \(-2\) is \(w(-2) = -w(2) = 1\), and finally, the action on \(-3\) is \(w(-3) = -w(3) = -2\).
    So long as \(i\) and \(\overbar{i}\) don't both appear in this notation and no numbers are repeated the result will always be a valid signed permutation.
    If no \(\overbar{i}\)s appear then it's just a normal permutation.
    
    Notice that \(\Sym_n = W(\A[n-1])\) is a subgroup of \(W(\B[n])\), and \(W(\B[n])\) is a subgroup of \(\Sym_{2n} = W(\A[2n-1])\).
    
    As an example of composition consider \(W(\B[3])\).
    We have \([3, \overbar{1}, 2]\) as described above, and also \([\overbar{1}, 2, 3]\) which sends \(1\) to \(-1\) and leaves \(2\) and \(3\) invariant.
    We can determine the composite \([\overbar{1}, 2, 3] [3, \overbar{1}, 2]\) (acting to the right).
    First \(1\) maps to \(3\), then this maps to \(3\).
    Second, \(2\) maps to \(-1\), which then maps to \(1\).
    Finally, \(3\) maps to \(2\), which maps to \(2\).
    Thus, the composite is \([3, 1, 2]\).
    
    One way to think of signed permutations is as acting by first a permutation, \(\sigma\), and then acting as a possible sign change for each term.
    For example, \([3, \overbar{1}, 2]\) is first the permutation \(\sigma = \cycle{1,3,2}\), then the sign change \(({-},{+},{+})\), changing the sign of \(1\) and not \(2\) or \(3\).
    This allows us to encode this as a semidirect product, \(\integers_2^n \rtimes \Sym_n\), in which \(\integers_2^n\) is interpreted as a list of \(n\) signs.
    This is a special case of the more general wreath product, in our case \(\integers_2^n \rtimes \Sym_n = \integers_2 \wr \Sym_n\).
    
    To realise this as a reflection group we again act on \(\reals^n\) by permuting the standard basis with \(\Sym_n\), and then the \(i\)th copy of \(\integers_2\) corresponds to \(s_{e_i}\).
    
    \subsection{Signed Permutations with an Even Number of Sign Changes: \texorpdfstring{\(\D[n]\)}{Dn}}
    The group of signed permutations on \(n\) letters has an index 2 subgroup, \(W(\D[n])\), defined to be the subgroup consisting of all signed permutations which flip an even number of signs.
    This is a reflection group still.
    Taking \(\reals^n\) with the standard basis this group is generated by the \(s_{e_i - e_{i+1}}\) giving the permutations, and the reflections \(s_{e_i + e_j}\) for \(i \ne j\) flipping an even number of signs.
    
    \subsection{Others}
    It turns out that \(W(R)\) represents the symmetries of a \emph{regular} polytope if and only if the Dynkin diagram of \(R\) is \define{simply laced}, that is, each vertex has degree at most \(2\).
    So, \(\A[n]\), \(\B[n]\), \(\G[2]\), \(\F[4]\), \(\H[3]\), \(\H[4]\), and \(\I[2][m]\) all correspond to symmetries of regular polytopes.
    
    We've seen that \(\A[n]\) corresponds to the symmetries of the \(n\)-simplex, \(\B[n]\) the symmetries of the \(n\)-cube, \(\H[3]\) the symmetries of the dodecahedron and \(\I[2][m]\) the symmetries of the regular \(m\)-gon.
    Noting that \(\G[2]\) is just \(\I[2][6]\) we see that \(\G[2]\) corresponds to the symmetries of the regular hexagon.
    
    It's not obvious, but \(\F[4]\) corresponds to the symmetries of a regular \(24\)-cell, a \(4\)-dimensional shape.
    This shape is the convex hull of its vertices, which can be given by either the \(24\) permutations of \((\pm 1, \pm 1, 0, 0) \in \reals^{4}\), or the \(8\) permutations of \((\pm 1, 0, 0, 0)\) and the 16 vertices \((\pm 1/2, \pm 1/2, \pm 1/2, \pm 1/2)\).
    This later description is particularly interesting because if we identify \(\reals^4\) with \(\quaternions\) then we find that these correspond to the Hurwitz quaternions.
    
    Finally, \(\H[4]\) corresponds to the symmetries of the regular \(120\)-cell, another 4-dimensional shape.
    This can similarly be defined as the convex hull of its vertices, but now the vertices are somewhat more complicated to state (there are 600 of them).
    The coordinates of the vertices can be given as all permutations of
    \begin{gather}
        (\pm 2, \pm 2, 0, 0), \quad (\pm \varphi, \pm \varphi, \pm \varphi, \pm \varphi^{-2}), \quad (\pm 1, \pm 1, \pm 1, \pm \sqrt{5}), \notag\\
        \text{and}\quad (\pm \varphi^{-1}, \pm \varphi^{-1}, \pm \varphi^{-1}, \pm \varphi^2)
    \end{gather}
    and all even permutations of
    \begin{equation}
        (0, \pm \varphi^{-1}, \pm \varphi, \pm \sqrt{5}), \quad (0, \pm \varphi^{-2}, \pm 1, \pm \varphi^2), \qand (\pm \varphi^{-1}, \pm 1, \pm \varphi, \pm 2). \notag
    \end{equation}
    Here \(\varphi = (1 + \sqrt{5})/2\) is the golden ration.
    Note that the first coordinates here, permutations of \((\pm 2, \pm 2, 0, 0)\), are a (scaled) \(24\)-cell.
    
    The other Dynkin types can still be realised as the symmetries of non-regular polytopes.
    However, these tend to be very complicated (not helped by the fact that the rank of the Dynkin diagram corresponds to the dimension of the shape), and are typically constructed especially for the purpose of having the specified symmetries.
    For example, one shape with its symmetry corresponding to \(\E[6]\) is the so-called \(2_{21}\)-polytope.
    This is a \(6\)-dimensional shape which is constructed by taking \(W(\E[6])\), acting as reflections of \(\reals^6\) and then taking the orbit of a nonzero point under this group to define the vertices of the polytope.
    This method will work for generating a polytope with any given reflection symmetry group.
\end{document} 