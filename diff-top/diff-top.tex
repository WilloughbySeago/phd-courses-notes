% !TeX program = lualatex
\documentclass[fleqn]{NotesClass}

\strictpagecheck

\usepackage{csquotes}
\usepackage{tensor}

\usepackage{tikz}
\usetikzlibrary{external}
\tikzexternalize[prefix=tikz-external/]

\usepackage{tikz-cd}

\usepackage[pdfauthor={Willoughby Seago},pdftitle={Notes from differential topology},pdfkeywords={differential topology, manifold, tangent space, homology, cohomology},pdfsubject={Differential Topology}]{hyperref}  % Should be loaded second last (cleveref last)
\colorlet{hyperrefcolor}{blue!60!black}
\hypersetup{colorlinks=true, linkcolor=hyperrefcolor, urlcolor=hyperrefcolor}
\usepackage[
capitalize,
nameinlink,
noabbrev
]{cleveref} % Should be loaded last

% My packages
\usepackage{NotesBoxes}
\usepackage{NotesMaths2}

\setmathfont[range={\int, \oint, \otimes, \oplus, \bigotimes, \bigoplus}]{Latin Modern Math}


% Highlight colour
\definecolor{highlight}{HTML}{710D78}
\definecolor{my blue}{HTML}{2A0D77}
\definecolor{my red}{HTML}{770D38}
\definecolor{my green}{HTML}{14770D}
\definecolor{my yellow}{HTML}{E7BB41}

% Title page info
\title{Differential Topology}
\author{Willoughby Seago}
\date{October 10th, 2024}
\subtitle{Notes from}
\subsubtitle{SMSTC}
\renewcommand{\abstracttext}{These are my notes from the SMSTC course \emph{Differential Topology} taught by Proff Murad Alim. These notes were last updated at \printtime{} on \today{}.}

% Commands
% Maths
\renewcommand{\dl}{\symrm{d}}
\newcommand{\id}{\symrm{id}}
\newcommand{\atlas}{\symcal{A}}
\newcommand{\torus}{\symbb{T}}
\newcommand{\isomorphic}{\cong}
\newcommand{\tangrel}[1][p]{\mathrel{\stackrel{\scriptscriptstyle #1}{\sim}}}
\newcommand{\vectorFields}{\symfrak{X}}
\DeclarePairedDelimiterX{\bracket}[2]{[}{]}{#1, #2}
\DeclareMathOperator{\Der}{Der}
\newcommand{\derivations}{\Der}
\DeclareMathOperator{\End}{End}

\begin{document}
    \frontmatter
    \titlepage
    \innertitlepage{}
    \tableofcontents
    % \listoffigures
    \mainmatter
    
    \chapter{Smooth Manifolds}
    \section{Definitions in \texorpdfstring{\(\reals^n\)}{Rn}}
    \begin{dfn}{Smooth Map}{def:smooth map euclidean spaces}
        Let \(U \subseteq \reals^m\) and \(V \subseteq \reals^n\) be open sets\footnote{Here, and throughout, we assume the standard topology on Euclidean space, \(\reals^m\), so a set, \(U\), is open if and only if every point \(x \in U\) there is an open ball centred on \(x\) contained entirely within \(U\).}.
        A map \(f \colon U \to V\) is \define{smooth}\index{smooth!map of Euclidean spaces} is called \defineindex{smooth} if it is infinitely differentiable.
        That is, all partial derivatives of all orders\footnote{we take \(0 \in \naturals\)},
        \begin{equation}
            \partial^\alpha f = \frac{\partial^{\alpha_1 + \dotsb + \alpha_k}}{\partial x_1^{\alpha_1} \dotsm \partial x_k^{\alpha_k}}, \qqwhere \alpha = (\alpha_1, \dotsc, \alpha_k) \in \naturals^{k}
        \end{equation}
        exist and are continuous.
    \end{dfn}
    
    \begin{dfn}{}{def:differential of map between Rn}
        Let \(U \subseteq \reals^m\) and \(V \subseteq \reals^n\) be open sets.
        For a smooth map \(f = (f_1, \dotsc, f_n) \colon U \to V\) and a point \(x \in U\) the \defineindex{derivative}, \defineindex{differential}, or \defineindex{pushforward} of \(f\) at \(x\) is the linear map
        \begin{equation}
            f_*(x) = \dl f_x = \dl f(x) \colon \reals^m \to \reals^n
        \end{equation}
        defined by acting on \(\xi \in \reals^m\) according to
        \begin{equation}
            f_*(x)\xi = \dl f_x(\xi) = \dl(f)(x)\xi = \diff{}{t}\bigg|_{t = 0} f(x + t\xi) = \lim_{t \to 0} \frac{f(x + t\xi) - f(x)}{t}.
        \end{equation}
    \end{dfn}
    
    Fixing coordinates \(x = (x^1, \dotsc, x^m)\) the differential is represented by the Jacobian matrix
    \begin{equation}
        (\dl f(x))_{ij} = \left( \diffp{f_i}{x^j} \right) = 
        \begin{pmatrix}
            \diffp{f_1}{x^1}(x) & \dots & \diffp{f_1}{x^m}(x)\\
            \vdots & \ddots & \vdots\\
            \diffp{f_n}{x^1}(x) & \dots & \diffp{f_n}{x^m}(x)
        \end{pmatrix}
        \in \matrices[n]{m}{\reals}.
    \end{equation}
    
    The differential is defined in terms of a derivative of a function \(\reals \to \reals\) (that function is \(t \mapsto f(x + t\xi)\)), and as such has many of the properties of this derivative, including linearity and the chain rule.
    That is, if \(U \subseteq \reals^m\), \(V \subseteq \reals^n\), and \(W \subseteq \reals^p\) are open and \(f \colon U \to V\) and \(g \colon V \to W\) are smooth then the differential of \(g \circ f \colon U \to W\) at \(x \in U\) is the linear map \(\dl (g \circ f)(x) \colon \reals^m \to \reals^p\) given by
    \begin{equation}
        \dl (g \circ f)(x) = \dl g(f(x)) \circ \dl f(x).
    \end{equation}
    
    The identity map \(\id_U \colon U \to U\), given by \(x \mapsto x\), is smooth, its derivatives all vanish.
    Thus, we have
    \begin{equation}
        \dl (\id_U) (x) = \diff{}{t} \bigg|_{t = 0} \id_U(x + t\xi) = \diff{}{t} \bigg|_{t = 0} (x + t\xi) = \xi,
    \end{equation}
    and so \(\dl (\id_U)(x) = \id_{\reals^n}\).
    
    If \(f \colon U \to V\) is a \defineindex{diffeomorphism} (smooth map with smooth inverse) then for \(x \in V\) we can use the above result to get
    \begin{align}
        \dl (f \circ f^{-1})(x) = \dl \id_U(x) = \id_{\reals^n}(x) = x,
    \end{align}
    and we can use the chain rule to get
    \begin{equation}
        \dl (f \circ f^{-1})(x) = \dl f(f^{-1}(x)) \circ \dl f^{-1}(x).
    \end{equation}
    Thus, we may conclude that
    \begin{equation}
        \dl f(f^{-1}(x)) \circ \dl f^{-1}(x) = \id_{\reals^n}
    \end{equation}
    which tells us that \(\dl f(x)\) is invertible, with inverse \(\dl f(f^{-1}(x))\).
    This requires that \(m = n\).
    
    \section{Smooth Manifold}
    A smooth manifold is a space that locally looks like \(\reals^m\) for some fixed \(m\).
    
    \begin{dfn}{Smooth Manifold}{}
        A \defineindex{smooth manifold}, \((M, \atlas)\), of dimension \(m \in \naturals\), is a topological space, \(M\), equipped with an open cover, \(\{U_\alpha\}_{\alpha \in A}\) for some indexing set \(A\).
        This open cover must be equipped with a collection of homeomorphisms, \(\varphi_\alpha \colon U_\alpha \to \Omega_{\alpha}\), where \(\Omega_\alpha \subseteq \reals^m\) is open.
        These maps must be such that for any \(\alpha, \beta \in A\) the \defineindex{transition map}
        \begin{equation}
            \varphi_{\beta \alpha} \coloneq \varphi_\beta \circ \varphi_\alpha^{-1} \colon \varphi_\alpha(U_\alpha \cap U_\beta) \to \varphi_\beta(U_\alpha \cap U_\beta)
        \end{equation}
        is smooth\footnote{Note that this is a map between open subsets of \(\reals^m\), so \cref{def:smooth map euclidean spaces} applies}.
        We call the pair \((U_\alpha, \varphi_\alpha)\) a \defineindex{coordinate chart}, and the collection \(\atlas = \{(U_\alpha, \varphi_\alpha)\}_{\alpha \in A}\) is called an atlas.
    \end{dfn}
    
    The requirement that the transition maps are diffeomorphisms is why we call these manifolds \emph{smooth}.
    If we just require that they by homeomorphisms then we get the notion of a topological manifold, and if we require they're \(k\)-times differentiable we get a \(C^k\)-manifold.
    It's also possible to replace \(\reals\) with \(\complex\) in all of our previous definitions and consider complex manifolds.
    This is mostly done when we then require that the transition maps are holomorphic, and we get a holomorphic manifold.
    
    It can be shown that \(U \subseteq M\) is open if and only if \(\varphi_\alpha(U \cap U_\alpha)\) is open in \(\reals^m\) for all \(\alpha \in A\).
    Thus, the charts uniquely determine the topology on \(M\) and vice versa.
    
    If we have a homeomorphism \(\psi \colon V \to \Omega\) for \(V \subseteq M\) and \(\Omega \subseteq \reals^m\) open then we call \((\psi, V)\) \defineindex{compatible} with the atlas \(\atlas\) if the transition map \(\varphi_\alpha \circ \varphi^{-1} \colon \psi(V \cap U_\alpha) \to \varphi_\alpha(U \cap U_\alpha)\) is a diffeomorphism for all \(\alpha \in A\).
    
    We call the atlas \(\atlas\) \define{maximal}\index{maximal atlas} if it contains every chart that is compatible with all of its members.
    It can be shown that every atlas is contained in a unique maximal atlas, \(\overbar{\atlas}\).
    Sometimes it's useful to assume that an atlas is maximal, and we can always do this because adding compatible charts to an atlas doesn't change the smooth structure of the manifold.
    We can extend the notion of compatibility to whole atlases, and we say that two atlases are compatible if their union is an atlas, and in this case both induce the same smooth structure on \(M\), and the maximal atlas containing them is the same.
    
    \begin{exm}{}{}
        \begin{itemize}
            \item \(\reals^m\) is a manifold of dimension \(m\) and it has an atlas given by the single chart \((\reals^m, \id_{\reals^m})\).
            \item The sphere, \(S^n = \{(x_1, \dotsc, x_{n + 1}) \in \reals^{n+1} \mid x_1^2 + \dotsb + x_{n+1}^2 = 1\}\), is a manifold of dimension \(n\), and it is covered by two charts \((S^n \setminus \{(1, 0, \dotsc, 0)\}, s_+)\) and \((S^n \setminus \{(-1, 0, \dotsc, 0)\}, s_-)\) where \(s_+\) and \(s_-\) are stereographic projection from the points \((\pm 1, 0, \dotsc, 0)\).
            \item The \(m\)-torus, \(\torus^m \coloneqq \reals^m/\integers^m\) (with the quotient topology, and \(\integers^m\) has the discrete topology, which is also the subset topology as a subset of \(\reals^m\) with the standard topology) is a manifold.
            Two vectors \(x, y \in \reals^m\) project to the same point in \(\torus^m\) if their difference, \(x - y\), is an integer vector (that is, it's in \(\integers^m\)).
            Denote the projection by \(\pi \colon \reals^m \twoheadrightarrow \torus^m\), and this acts on \(x \in \reals^m\) by \(\pi(x) = [x] = x + \integers^m\).
            A set \(U \subseteq \torus^m\) is open if and only if \(\pi^{-1}(U)\) is open in \(\reals^m\) (this is just the definition of the quotient topology).
            An atlas on \(\torus^m\) is given by taking the \(\reals^m\)-indexed open cover
            \begin{equation}
                U_\alpha = \{[x] \mid x \in \reals^m \text{ and } \abs{x - \alpha} < 1/2\},
            \end{equation}
            and the corresponding coordinate maps \(\varphi_\alpha \colon U_\alpha \to \reals^m\) defined by
            \begin{equation}
                \varphi_\alpha([x]) = x
            \end{equation}
            for \(x \in \reals^m\) with \(\abs{x - \alpha} < 1/2\).
            This is well defined since by the construction of the \(U_\alpha\) there is only one \(x \in \reals^m\) satisfying this distance constraint.
            It can be shown that the transition maps are translation by an integer vector.
            One way to think of the quotient torus \(\reals^m/\integers^m\) is as a fixed \(1 \times 1 \times \dotsb \times 1\) cube in \(\reals^m\) with periodic boundary conditions, which we can view as the cube repeating in all directions forever.
            Then the transition maps just move us from one cube to the same point in some other cube.
            Finally, note that the torus may equivalently be defined as \(\torus^m = S^1 \times \dotsb \times S^1\) with \(m\) copies of the circle, \(S^1\), and the product topology.
            \item Fixing a smooth function \(f \colon U \to \reals^n\) from some open subset \(U \subseteq \reals^m\) the curve \(\{(x, f(x)) \in U \times \reals^n \mid x \in U\}\) is a smooth manifold.
            \item \(\generalLinear(n, \reals) \coloneqq \{A \in \matrices{n}{\reals} \mid \det A \ne 0\}\) is a smooth manifold.
            One way to show this is to realise that \(\generalLinear(n, \reals) = \det^{-1}(\reals \setminus \{0\})\).
            That is, \(\generalLinear(n, \reals)\) is the preimage of the set \(\reals \setminus \{0\}\) under the determinant.
            The determinant is a smooth map \(\reals^{n^2} \to \reals\) (note that \(\matrices{n}{\reals} \isomorphic \reals^{n^2}\) as vector spaces), since the determinant is just a polynomial in the entries in the matrix, which are the components of the vectors in \(\reals^{n^2}\).
            The set \(\reals \setminus \{0\}\) is open.
            It is known that the preimage of an open set under a smooth map is open.
            Thus, we may consider \(\generalLinear(n, \reals) \subseteq \reals^{n^2}\) as an open subset, and it is known that any open subset of \(\reals^k\) is a manifold (more generally, an open subset of a manifold is again a manifold).
        \end{itemize}
    \end{exm}
    
    One important thing about manifolds is that they have a well defined notion of dimension.
    If some space is locally homeomorphic to \(\reals^m\) in some area, but locally homeomorphic to \(\reals^n\) in another and \(m \ne n\) then it cannot be a manifold.
    For example, consider a pair of lines that cross at a point.
    Each line is a manifold on its own, but their union isn't, since we can't really define dimension in a neighbourhood of the intersection.
    
    \section{Smooth Maps}
    \begin{dfn}{Smooth Map}{}
        Let \((M, \{(U_\alpha, \varphi_\alpha)\}_{\alpha \in A})\) and \((N, \{(V_\beta, \psi_\beta)\}_{\beta \in B})\) be smooth manifolds of dimension \(m\) and \(n\) respectively.
        A \define{smooth map}\index{smooth map!between manifolds}, \(f \colon M \to N\), is a function such that for all charts \((U_\alpha, \varphi_\alpha)\) and \((V_\beta, \psi_\beta)\) the map \(\psi_\beta \circ f \circ \varphi_\alpha^{-1}\) is smooth as a map between the open sets \(\varphi_\alpha(U_\alpha) \subseteq \reals^m\) and \(\psi_\beta(V_\beta) \subseteq \reals^n\).
    \end{dfn}
    
    The idea here is that we can pass smoothness of a map between manifolds to smoothness of corresponding patches of Euclidean space, where we can use \cref{def:smooth map euclidean spaces}.
    The important thing here is that we have the commutative diagram
    \begin{equation}
        \tikzexternaldisable
        \begin{tikzcd}[column sep=small]
            & U_\alpha \arrow[rrr, "f"] \arrow[d, "\varphi_\alpha"'] &&& V_\beta \arrow[d, "\psi_\beta"]\\
            \reals^m \arrow[r, phantom, "\supseteq"] & \varphi_\alpha(U_\alpha) \arrow[rrr, "\psi_\beta \circ f \circ \varphi_\alpha^{-1}"'{yshift=-0.05cm}] &&& \psi_\beta(V_\beta) \arrow[r, phantom, "\subseteq"] & \reals^n
        \end{tikzcd}
        \tikzexternalenable
    \end{equation}
    
    \section{Tangent Space}
    The tangent space to a manifold is most immediate when we view that manifold as a subset of \(\reals^n\).
    For example, the tangent space to the circle, \(S^1\), at some point \(p\) is just the tangent line to that point.
    This line is a copy of \(\reals\), and so comes with lots of structure, and in particular is a vector space.
    If we go up a dimension we can consider the sphere, \(S^2\), and the tangent space at some point \(p\) is the tangent plane at that point.
    This plane is a copy of \(\reals^2\), and is, again, a vector space.
    
    The formal definition of the tangent space just formalises this by defining the tangent space to be the space of all tangent curves through \(p\).
    
    \begin{dfn}{Tangent Space}{}
        Let \((M, \{(U_\alpha, \varphi_\alpha)\}_{\alpha \in A})\) be a smooth manifold.
        Fix some point \(p \in M\).
        The tangent space of \(M\) at \(p\) is the quotient space
        \begin{equation}
            T_pM \coloneqq \bigsqcup_{\alpha \text{ s.t. }p \in U_\alpha} \reals^m / {\tangrel}.
        \end{equation}
        Here we use the disjoint union, but we could instead use the normal union and explicitly label each point in \(\reals^m\) with \(\alpha\) from the corresponding term in the union:
        \begin{equation}
            T_pM = \bigcup_{\alpha \text{ s.t. } p \in U_\alpha} (\{\alpha\} \times \reals^m)/{\tangrel}.
        \end{equation}
        Then the corresponding equivalence relation is
        \begin{equation}
            (\alpha, \xi) {\tangrel} (\beta, \eta) \iff \dl(\varphi_\beta \circ \varphi_\alpha^{-1})(x)\xi = \eta, \quad x = \varphi_\alpha(p).
        \end{equation}
        That is, two pairs \((\alpha, \xi)\) and \((\beta, \eta)\) are equivalent if one is the pushforward of the other along the transition maps.
        The equivalence class of \((\alpha, \xi) \in A \times \reals^m\) with \(p \in U_\alpha\) is denoted \([\alpha, \xi]_p\).
    \end{dfn}
    
    This quotient space, \(T_pM\), is a real vector space of dimension \(m\).
    
    \begin{exm}{}{}
        Consider the circle, \(S^1 = \{(x, y) \in \reals^2 \mid x^2 + y^2 = 1\}\), with the charts \(U_{\pm} = S^1 \setminus \{(0, \mp 1)\}\) where
        \begin{equation}
            \varphi_+(x, y) = \frac{x}{1 + y}, \qand \varphi_-(x, y) = \frac{x}{1 - y}.
        \end{equation}
        We have
        \begin{equation}
            \varphi_-^{-1}(t) = \left( \frac{2t}{t^2 + 1}, \frac{t^2 - 1}{t^2 + 1} \right), \qand \varphi_+^{-1}(t) = \left( \frac{2t}{t^2 + 1}, \frac{1 - t^2}{t^2 + 1} \right).
        \end{equation}
        
        Take the point \(p = (\sqrt{3}/2, 1/2) \in S^1\).
        Then we have
        \begin{equation}
            \varphi_-(p) = \sqrt{3}, \quad \varphi_+(p) = \frac{1}{\sqrt{3}}.
        \end{equation}
        The transition maps are
        \begin{equation}
            \varphi_+ \circ \varphi_-^{-1}(t) = \frac{1}{t} = \varphi_- \circ \varphi_+^{-1}(t).
        \end{equation}
        
        Consider the pair \((-, q) \in A \times \reals\) where \(A = \{{+}, {-}\}\) is the index set of the chart and \(\reals\) is the one-dimensional vector space corresponding to the tangent space to the one-dimensional manifold \(S^1\).
        What is the equivalence class of \((-, q)\) in the tangent space at \(p\)?
        That is, what is \([-, q]_p\)?
        
        To compute this we take the definition of the equivalence relation, that \((-, q) \tangrel (+, \tilde{q})\) if and only if
        \begin{equation}
            \dl (\varphi_+ \circ \varphi_-^{-1})(x)q = \tilde{q}
        \end{equation}
        where \(x = \varphi_-(p) = \sqrt{3}\).
        Then we are looking for some real number, \(\tilde{q}\), given by evaluating
        \begin{equation}
            \dl(\varphi_+ \circ \varphi_-^{-1})(\sqrt{3})q.
        \end{equation}
        Using the definition of the differential we have that
        \begin{align}
            \dl(\varphi_+ \circ \varphi_-^{-1})(x)q &= \diff{}{t}\bigg|_{t=0} \varphi_+ \circ \varphi_-^{-1}(x + tq)\\
            &= \diff{}{t}\bigg|_{t=0} \frac{1}{x + tq}\\
            &= -\frac{q}{(x + tq)^2}\bigg|_{t = 0}\\
            &= -\frac{q}{x^2}
        \end{align}
        and so taking \(x = \sqrt{3}\) we get
        \begin{equation}
            \dl(\varphi_+ \circ \varphi_-^{-1})(\sqrt{3})q = -\frac{q}{3}.
        \end{equation}
        So, \((-, q) \tangrel (+, -q/3)\), and so
        \begin{equation}
            [-, q]_p = \{(-, q), (+, -q/3)\}.
        \end{equation}
        Thus, the tangent space to the circle at \(p = (\sqrt{3}/2, 1/2)\) is
        \begin{align}
            T_{(\sqrt{3}/2, 1/2)}S^1 &= \{[-, q]_{(\sqrt{3}/2, 1/2)} \mid p \in \reals\}\\
            &= \{\{(-, q), (+, -q/3)\} \mid p \in \reals\}.
        \end{align}
    \end{exm}
    
    Often we want to consider all of the tangent spaces at once.
    To do this we construct the tangent bundle.
    
    \begin{dfn}{Tangent Bundle}{}
        The \defineindex{tangent bundle} of a manifold \(M\) is
        \begin{equation}
            TM \coloneqq \bigsqcap_{p \in M} T_pM.
        \end{equation}
    \end{dfn}
    
    The tangent bundle of an \(m\)-dimensional smooth manifold is a \(2m\)-dimensional smooth manifold in a way that we shall define later once we have some more machinery.
    
    \section{Derivatives}
    Suppose that we have two smooth manifolds, \((M, \{(U_\alpha, \varphi_\alpha)\}_{\alpha \in A})\) and \((N, \{(V_\beta, \psi_B	)\}_{\beta \in B})\), of dimensions \(m\) and \(n\) respectively, and a smooth map \(f \colon M \to N\).
    Can we construct a map
    \begin{equation}
        T_pM \to T_{f(p)}N?
    \end{equation}
    The answer is yes, and to derive\footnote{no pun intended} a sensible definition of this map we should look at what maps we really have available to us.
    
    The map we're after is \([\alpha, \xi]_p \mapsto [\beta, \eta]_{f(p)}\), we just need to specify what \(\eta\) is, it should in general depend on \(\xi\), \(\alpha\), \(\beta\), \(p\), and \(f\).
    Note that it only makes sense to consider \(\alpha\) and \(\beta\) such that \(p \in U_\alpha\) and \(f(p) \in V_\beta\), and we may further assume that \(p\) has a neighbourhood that is contained entirely in \(V_\beta\), so we'll assume that \(f(U_\alpha) \subseteq V_\beta\).
    
    The maps we have involving the manifolds directly may be summarised in the following diagram:
    \begin{equation}
        \tikzexternaldisable
        \begin{tikzcd}[column sep=2.5cm]
            U_\alpha \arrow[r, "f"] \arrow[d, "\varphi_\alpha"'] & V_\beta \arrow[d, "\psi_\beta"]\\
            \varphi_\alpha(U_\alpha) \arrow[r, "\psi_\beta \circ f \circ \varphi_\alpha^{-1} = f_{\beta\alpha}"'] & \psi_\beta(V_\beta)\mathrlap{.}
        \end{tikzcd}
        \tikzexternalenable
    \end{equation}
    The action of these maps is
    \begin{equation}
        \tikzexternaldisable
        \begin{tikzcd}
            p \arrow[r, mapsto] \arrow[d, mapsto] & f(p) \arrow[d, mapsto]\\
            \varphi_\alpha(p) \arrow[r, mapsto] & \psi_\beta(f(p))\mathrlap{.}
        \end{tikzcd}
        \tikzexternalenable
    \end{equation}
    The map \(f_{\beta\alpha}\) is between an open subset of \(\reals^m\) and an open subset of \(\reals^n\).
    This allows us to consider its differential, \(\dl f_{\beta\alpha}\), as defined in \cref{def:differential of map between Rn}.
    That is, we have the map
    \begin{equation}
        \dl f_{\beta\alpha}(x) \colon \reals^m \to \reals^n.
    \end{equation}
    So, given some \(x \in \reals^m\), we can define \(\eta\) to be \(\dl f_{\beta\alpha}(x)\xi\).
    The only question then is what \(x\) should we use.
    The only piece of data we haven't yet used is \(p\), so we should take \(x = \varphi_\alpha(p)\).
    This leads to us defining
    \begin{equation}
        \eta = \dl f_{\beta\alpha}(\varphi_\alpha(p))\xi.
    \end{equation}
    
    \begin{dfn}{Derivative}{}
        Let \((M, \{(U_\alpha, \varphi_\alpha)\}_{\alpha \in A})\) and \((N, \{(V_\beta, \psi_B	)\}_{\beta \in B})\) be smooth manifolds, and let \(f \colon M \to N\) be a smooth map.
        We define the \defineindex{derivative}, \defineindex{differential}, or \defineindex{pushforward} of \(f\) at \(p\) as the linear map
        \begin{equation}
            f_*(p) = \dl{f}_p = \dl{f}(p) \colon T_pM \to T_{f(p)}N 
        \end{equation}
        given by
        \begin{equation}
            \dl{f}(p)[\alpha, \xi]_p \coloneqq [\beta, \dl{f_{\beta\alpha}}(\varphi_\alpha(p))\xi]_{f(p)}
        \end{equation}
        where \(p \in U_\alpha\) and \(f(p) \in V_\beta\) and \(f_{\beta\alpha} = \psi_\beta \circ f \circ \varphi_\alpha^{-1}\).
        Note that \(\dl{f_{\beta\alpha}}\) is as defined in \cref{def:differential of map between Rn}.
    \end{dfn}
    
    \begin{exm}{}{exm:derivative of map into Rn}
        Consider the special case of the derivative when \(N = (\reals^n, \{(\reals^n, \id_{\reals^n})\})\) is \(\reals^n\) with a single chart.
        This means that the indexing set for charts on \(N\) is the singleton, \(B = \{\bullet\}\).
        Then for \(q \in N\) we can identify \([\bullet, \eta]_q \in T_qN\) with \(\eta \in \reals^n\), since there's only one label \(\bullet\), giving us a canonical isomorphism \(T_qN \isomorphic \reals^n\).
        If \(f \colon M \to N\) is smooth in this case, given \(p \in M\) and \([a, \xi]_p \in T_pM\), we have the derivative
        \begin{equation}
            \dl{f}(p)[a, \xi]_p = [\bullet, \dl{f_{\beta\alpha}}(\varphi_\alpha(p))\xi]_{f(p)}.
        \end{equation}
        Using this canonical isomorphism we identify this with
        \begin{equation}
            \dl{f}(p)[a, \xi]_p = \dl{f_{\bullet\alpha}}(\varphi_\alpha(p))\chi.
        \end{equation}
        This further simplifies since \(f_{\bullet\alpha} = \psi_\bullet \circ f \circ \varphi_\alpha^{-1} = \id_{\reals^n} \circ f \circ \varphi_\alpha^{-1} = f \circ \varphi_\alpha^{-1}\), and so, after identifying \(T_{f(p)}N\) and \(\reals^n\), we have
        \begin{equation}
            \dl{f}(p)[a, \xi]_p = \dl{(f \circ \varphi_\alpha^{-1})}(\varphi_\alpha(p))\xi.
        \end{equation}
        Once \(f\) is known this can be calculated using \cref{def:differential of map between Rn}.
    \end{exm}

    \begin{exm}{}{}
        We can apply this formula to maps defined on just an open subset of \(M\), rather than all of \(M\).
        The most important case of this is when the smooth map in consideration is a chart map, \(f = \varphi_{\tilde{\alpha}} \colon U_{\tilde{\alpha}} \to \reals^m\).
        In this case we may still identify \(N\) as \((\reals^n, \{(\reals^n, \id_{\reals^m})\})\), and so the results of \cref{exm:derivative of map into Rn} apply.
        In particular, after identifying \(T_{\varphi_{\tilde{\alpha}}}N\) with \(\reals^n\) we have
        \begin{equation}
            \dl{\varphi_{\tilde{\alpha}}}(p)[\alpha, \xi]_p = \dl{(\varphi_{\tilde{\alpha}} \circ \varphi_\alpha^{-1})}(\varphi_\alpha(p))\xi.
        \end{equation}
        This isn't that interesting, unless we take \(\alpha = \tilde{\alpha}\), then we have
        \begin{align}
            \dl{\varphi_\alpha}(p)[\alpha, \xi]_p &= \dl{(\varphi_\alpha\circ \varphi_\alpha^{-1})}(\varphi_\alpha(p))\xi\\
            &= \diff{}{t}\bigg|_{t=0} (\varphi_\alpha \circ \varphi_\alpha^{-1})(\varphi_\alpha(p) + t\xi)\\
            &= \diff{}{t}\bigg|_{t=0} \id(\varphi_\alpha(p) + t\xi)\\
            &= \diff{}{t}\bigg|_{t=0} (\varphi_\alpha(p) + t\xi)\\
            &= \xi.
        \end{align}
        This allows us to identify \([\alpha, \xi]_p \in T_pM\) with \(\xi \in \reals^m\), giving an isomorphism \(T_pM \isomorphic \reals^m\).
        So, \(\dl{\varphi_\alpha}\) is the canonical vector space isomorphism determined by \(\alpha\).
    \end{exm}
    
    \begin{ntn}{}{}
        Let \(\{e_i\}_{i \in \{1, \dotsc, m\}}\) be the standard basis for \(\reals^m\).
        Then for the tangent vector \([\alpha, e_i]_p \in T_pM\) we write
        \begin{equation}
            [\alpha, e_i]_p = \diffp{}{x^i}(p) = \diffp{}{x^i}\bigg|_p.
        \end{equation}
        This is (for now) just notation, we're not thinking of these as derivatives.
        We'll see why this is a sensible notation shortly.
    \end{ntn}
    
    \section{Submersions, Immersions, and Embeddings}
    In this section we state some definitions and results which are standard in the theory of smooth manifolds, but for which we will have little use.
    
    \begin{dfn}{Submersion}{}
        A \defineindex{submersion} of a smooth manifold, \(M\), in a smooth manifold, \(N\), is a smooth map, \(f \colon M \to N\), whose derivative, \(\dl{f}\), exists and is a surjective linear map
        \begin{equation}
            \dl{f}(p) \colon T_pM \to T_{f(p)}N
        \end{equation}
        for all \(p \in M\).
        
        Points \(p \in M\) at which \(\dl{f}(p)\) is a surjective linear map are called \define{regular points}\index{regular point}.
        So, a submersion is a map for which every \(p \in M\) is regular.
    \end{dfn}
    
    \begin{dfn}{Immersion}{}
        An \defineindex{immersion} of a smooth manifold, \(M\), in a smooth manifold, \(N\), is a smooth map, \(f \colon M \to N\), whose derivative, \(\dl{f}\), exists and is an injective linear map
        \begin{equation}
            \dl{f}(p) \colon T_pM \to T_{f(p)}N
        \end{equation}
        for all \(p \in M\).
        
        If an immersion is also a topological embedding, meaning that \(f(M)\) (with the subspace topology in \(N\)) is homeomorphic to \(M\), then we call \(f\) an \defineindex{embedding}.
    \end{dfn}
    
    The notion of an embedding uses the fact that \(f(M)\) is a topological space in its own right, and in fact, it's a manifold in its own right.
    This leads to the notion of a submanifold.
    
    \begin{dfn}{Submanifold}{}
        An \defineindex{embedded submanifold} of \(M\) is a subset, \(S \subseteq M\), which is itself a manifold and for which the inclusion \(S \hookrightarrow M\) is an embedding.
        
        More explicitly, \(S\) is a topological space in the subspace topology such that for every \(p \in S\) there exists a neighbourhood, \(U_\alpha\), of \(p \in M\) making a chart \((U_\alpha, \varphi_\alpha \colon U_\alpha \to \Omega_\alpha \subseteq \reals^m)\) (in some maximal atlas for \(M\)) such that \(S \cap U_\alpha = \varphi_\alpha^{-1}(\reals^k \cap \Omega_\alpha)\) for some fixed \(k \in \naturals\), which is the dimension of the submanifold, \(S\).
        Note that there is some freedom in how we embed \(\reals^k\) in \(\reals^m\) in order to take this intersection, and in general we only require that there is an embedding that makes this work, not that it works for all embeddings.
    \end{dfn}
    
    There is a weaker notion of an immersed submanifold in which the inclusion is merely required to be an immersion.
    
    A classic example of a submanifold is a smooth curve on some surface.
    This will correspond to the intersection \(\reals \cap \Omega_\alpha\).
    
    \begin{thm}{Regular Value Theorem}{}
        If \(f \colon M \to N\) is a smooth map of manifolds then \(f^{-1}(y)\) (that is, the preimage of some point \(y \in N\)) is a smooth submanifold of \(M\) for every \(y \in N\) if \(y\) is a regular value.
    \end{thm}
    
    \chapter{Vector Fields}
    \section{Definitions}
    \begin{dfn}{Vector Field}{}
        Let \(M\) be a smooth manifold.
        A \defineindex{vector field} on \(M\) is an assignment of a tangent vector, \(X(p) \in T_pM\), to each \(p \in M\).
        Thus, a vector field on \(M\) is a map \(X \colon M \to TM\) such that \(X(p) = X_p \in T_pM\) for all \(p \in M\).
    \end{dfn}
    
    \begin{exm}{}{exm:vector fields on Rn}
        Consider a chart \((U_\alpha, \varphi_\alpha \colon U_\alpha \to \Omega_\alpha \subseteq \reals^m)\) for the manifold \(M\).
        Let \(X\) be a vector field on \(M\).
        For each chart index, \(\alpha\), there is a corresponding vector field, \(X_\alpha \colon \Omega_\alpha \to \reals^m\) (identifying each \(T_pM\) with \(\reals^m\)), defined by
        \begin{equation}
            X_\alpha(\varphi_\alpha(p)) = \dl{\varphi_\alpha}(X(p)).
        \end{equation}
        Note that this is a vector field on \(\Omega_\alpha\) viewed as a submanifold of \(\reals^m\).
    \end{exm}
    
    \begin{dfn}{Smooth Vector Field}{}
        Let \(M\) be a manifold and \(X\) a vector field on \(M\).
        Then \(X\) is a \defineindex{smooth vector field} if \(X_\alpha\) (as defined in \cref{exm:vector fields on Rn}) is a smooth map between subsets of Euclidean space.
        Equivalently, \(X\) is smooth if \(X \colon M \to TM\) is a smooth map between manifolds.
        
        The set of all smooth vector fields on \(M\) is denoted \(\vectorFields(M)\). 
    \end{dfn}
    
    Since \(X \colon M \to TM\) takes values in \(T_pM \isomorphic \reals^m\) we inherit a lot of structure from \(\reals^m\).
    In particular, we can define pointwise addition,
    \begin{equation}
        (X + Y)(p) = X(p) + Y(p),
    \end{equation}
    and scalar multiplication,
    \begin{equation}
        (\lambda X)(p) = \lambda \mkern2mu X(p),
    \end{equation}
    for \(X, Y \in \vectorFields(M)\) and \(\lambda \in \reals\), and the results are still smooth vector fields.
    Since we also have the zero vector field, defined by \(0(p) = 0 \in \reals^m\), and negation (take \(\lambda = -1\)) we can easily see that \(\vectorFields(M)\) is a real vector space.
    
    Further, we can define multiplication by a smooth function, \(f \in C^{\infty}(M)\), by
    \begin{equation}
        (fX)(p) = f(p) X(p),
    \end{equation}
    and the result is again a smooth vector field.
    This is a bit like scaling, but now we scale by a different amount at each point of the manifold.
    This, combined with addition of vector fields tells us is that \(\vectorFields(M)\) is a \(C^{\infty}(M)\)-module.
    The vector space structure is then a special case where we restrict only to constant functions in \(C^{\infty}(M)\).
    
    \begin{exm}{}{}
        Let \(\{e_i\}_{i \in \{1, \dotsc, m\}}\) be the standard basis of \(\reals^m\).
        Let \(M\) be an \(m\)-dimensional smooth manifold with coordinate chart \((U_\alpha, \varphi_\alpha \colon U_\alpha \to \Omega_\alpha \subseteq \reals^m)\).
        We can define a vector field on \(U_\alpha\),
        \begin{equation}
            \diffp{}{x^i} \colon U_\alpha \to TM,
        \end{equation}
        by taking
        \begin{equation}
            \diffp{}{x^i}(p) = [\alpha, e_i]_p.
        \end{equation}
        This is such that
        \begin{equation}
            (\dl{\varphi_\alpha}(p))\left( \diffp{}{x^i}(p) \right) = e_i
        \end{equation}
        for all \(p \in U_\alpha\).
        Since \(\dl{\varphi_\alpha}(p)\) is always an isomorphism \(T_pM \to \reals^m\) we see that \(\{\difsp{}{x^i}(p)\}_{i\in \{1, \dotsc, m\}}\) is a basis for \(T_pM\), since it's mapped to a basis by an isomorphism.
        We call \(\{\difsp{}{x^i}\}_{i \in \{1,\dotsc, m\}}\) a local \defineindex{frame} on \(U_\alpha\), since it assigns to each point a basis vector.
        
        We write \(e_i\) for the constant vector field that assigns to each point in \(M\) the constant vector \(e_i \in T_pM\).
        We may then express any constant vector field as \(\sum_i X^ie_i\) for some real constants \(X^i\), and any smooth vector field as \(\sum_i X^i e_i\) where the \(X^i\) are smooth functions \(M \to \reals\).
    \end{exm}
    
    \section{Algebras}
    \begin{dfn}{Algebra}{}
        An \defineindex{algebra} over a field, \(\field\), is a vector space, \(A\), equipped with a bilinear map \(\cdot \colon A \times A \to A\).
    \end{dfn}
    
    \begin{dfn}{Derivation}{}
        If \(A\) is an algebra then a \defineindex{derivation} is a linear map \(D \colon A \to A\) satisfying the \defineindex{Leibniz rule} for all \(a, b \in A\):
        \begin{equation}
            D(a \cdot b) = D(a) \cdot b + a \cdot D(b).
        \end{equation}
    \end{dfn}
    
    The classic example of a derivation is, of course, a derivative acting on some algebra of functions, such as \(C^{\infty}(\reals)\), where the Leibniz rule is exactly the product rule.
    
    \begin{dfn}{Lie Algebra}{}
        A \defineindex{Lie algebra}, \(\lie{g}\), is a vector space over \(\field\), equipped with a bilinear map, \(\bracket{-}{-} \colon \lie{g} \times \lie{g} \to \lie{g}\), which is alternating, meaning \(\bracket{x}{x} = 0\) for all \(x \in \lie{g}\), and the Jacobi identity, that for all \(x, y, z \in \lie{g}\)
        \begin{equation}
            \bracket{x}{\bracket{y}{z}} + \bracket{y}{\bracket{z}{x}} + \bracket{z}{\bracket{x}{y}} = 0.
        \end{equation}
        
        A homomorphism of Lie algebras is a linear map \(\varphi \colon \lie{g} \to \lie{h}\) which preserves the bracket, so \(\varphi(\bracket{x}{y}) = \bracket{\varphi(x)}{\varphi(y)}\).
    \end{dfn}
    
    Let \(V\) be a vector space, then \(\End V\) is a Lie algebra where we define the bracket by \(\bracket{\varphi}{\psi} = \varphi \circ \psi - \psi \circ \varphi\).
    Let \(A\) be an algebra and write \(\derivations A\) for the subset of \(\End A\) consisting of derivations.
    Then it can be shown that \(\derivations A\) is a subalgebra of the Lie algebra \(\End A\).
    To do so one only needs to show that the commutator of two derivations is again a derivation.
    
    \begin{thm}{}{}
        Write \(\derivations(C^{\infty}(M))\) for the Lie algebra of derivations of the algebra of smooth functions. There is an isomorphism of \(C^{\infty}(M)\)-modules, \(\Phi \colon \vectorFields(M) \to \derivations(\complex^{\infty}(M))\), and this equips \(\vectorFields(M)\) with a natural Lie algebra structure.
        \begin{proof}
            Let \(\Omega \subseteq \reals^m\) be open and \(p \in \Omega\).
            Let \(X = \sum_i X^i e_i\) be a smooth vector field on \(\Omega\) expressed in the standard basis.
            Let \(f \colon \Omega \to \reals\) be a smooth function.
            We define another smooth function, \(\partial_Xf\), on \(\Omega\) by
            \begin{align}
                (\partial_X f)(p) &= \dl{f}_p(X_p)\\
                &= \dl{f}_p\left( {\textstyle\sum_i} X^i(p)e_i(p) \right)\\
                &= \sum_{i=1}^m \diffp{f}{x^i}(p) X^i(p).
            \end{align}
            This is smooth since it's a product of smooth functions \(\difsp{f}{x^i}, X^i \colon \Omega \to \reals\).
            Further, the map \(f \mapsto \partial_X f\) is a derivation on \(C^{\infty}(\Omega)\), since we have
            \begin{equation}
                \partial_X(fg) = \dl{(fg)}X = f \dd{g}X + g \dd{f}X = f\partial_Xg + g\partial_Xf.
            \end{equation}
            Now taking \(X\) to be a vector field on some smooth manifold, \(M\), and \(f \in C^{\infty}(M)\) we can define \(\partial_Xf(p) = \dl{f_p}X_p\), which we may also write as\(\partial_{X_p}f\).
            The map \(X \mapsto \partial_X\) is a morphism of \(C^{\infty}(M)\)-modules \(\vectorFields(M) \to \derivations(C^{\infty}(M))\).
            It is an isomorphism because % TODO: why is it an isomorphism?
        \end{proof}
    \end{thm}
    
    Using this result it is standard to identify a smooth vector field, \(X\), with the corresponding derivation\footnote{In fact, this is a standard way to define a vector field, and then one proves the above in reverse.}, \(\partial_X \in \derivations(C^{\infty}(M))\).
    Note that every tangent vector at a point, \(X_p\), defines a linear map \(\partial_{X_p} \colon C^{\infty}(M) \to \reals\) satisfying
    \begin{equation}
        \partial_{X_p}(fg) = g(p) \partial_{X_p}(f) + f(p) \partial_{X_p}(g),
    \end{equation}
    which is almost the Leibniz rule, except for the fact that \(\partial_{X_p}\) isn't an endomorphism.
    
    
    
    
    % Appdendix
%    \appendixpage
%    \begin{appendices}
%        
%    \end{appendices}
%
%	\backmatter
%	\renewcommand{\glossaryname}{Acronyms}
%	\printglossary[acronym]
%	\printindex
\end{document}
