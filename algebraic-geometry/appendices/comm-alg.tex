\chapter{Commutative Algebra}
Here we collect some results from commutative algebra which we'll make use of in the course.
This won't be very well organised, and is more for reference than actual reading.
The conditions to be included here are pretty much \enquote{I had to look it up} or \enquote{I had to think about it for more than 10 seconds} while writing these notes, or \enquote{I thought it was worth recapping}.

\section{Ideals}

\begin{dfn}{Prime Ideal}{}
    A proper ideal, \(\ideal{p} \subideal R\), is \define{prime}\index{prime ideal} if whenever \(ab \in \ideal{p}\) for \(a, b \in R\) then either \(a \in \ideal{p}\) or \(b \in \ideal{p}\).
    
    Equivalently, \(\ideal{p}\) is prime if \(R/\ideal{p}\) is an integral domain.
\end{dfn}

\begin{dfn}{Maximal Ideal}{}
    A proper ideal, \(\ideal{m} \subideal R\), is \define{maximal}\index{maximal ideal} if whenever there is another ideal, \(I \subideal R\), with \(\ideal{m} \subseteq I\) then either \(I = \ideal{m}\) or \(I = R\).
    
    Equivalently, \(\ideal{m}\) is maximal if \(R/\ideal{m}\) is a field.
\end{dfn}

\begin{lma}{}{lma:product of ideals subset of intersection}
    Let \(R\) be a ring with ideals \(I\) and \(J\).
    Then \(IJ \subseteq I \cap J\).
    \begin{proof}
        If \(a \in I\) and \(b \in J\) then \(ab \in I\) and \(ab \in J\) by definition of an ideal.
        Then \(ab \in I \cap J\).
    \end{proof}
\end{lma}

\begin{lma}{}{lma:radical of product is radical of intersection}
    Let \(R\) be a ring with ideals \(I\) and \(J\).
    Then \(\sqrt{IJ} = \sqrt{I \cap J} = \sqrt{I} \cap \sqrt{J}\).
    \begin{proof}
        We prove a circle of inclusions.
        We start with \(\sqrt{IJ} \subseteq \sqrt{I \cap J}\), which follows from \cref{lma:product of ideals subset of intersection}.
        
        If \(a \in \sqrt{I \cap J}\) then \(a^k \in I \cap J\) for some \(k \in \naturals\).
        Thus, \(a^k \in I\) and \(a^k \in J\).
        Hence, \(a \in \sqrt{I} \cap \sqrt{J}\).
        
        If \(a \in \sqrt{I} \cap \sqrt{J}\) then \(a^k \in I\) and \(a^{\ell} \in J\) for some \(k, \ell \in \naturals\).
        Then \(a^k a^{\ell} = a^{k + \ell} \in IJ\), and so \(a \in \sqrt{IJ}\).
    \end{proof}
\end{lma}

\begin{lma}{}{lma:prime ideal is radical}
    Every prime ideal is radical.
    \begin{proof}
        Let \(\ideal{p}\) be a prime ideal of a ring, \(R\).
        Consider \(\sqrt{\ideal{p}}\).
        If \(a \in \sqrt{\ideal{p}}\) then there exists some \(k \in \naturals\) such that \(a^k \in \ideal{p}\).
        Suppose that \(k\) is minimal in making this true.
        If \(k = 1\) then \(a \in \ideal{p}\).
        If \(k > 1\) then by the definition of a prime ideal have \(x \cdot x^{k-1} \in \ideal{p}\) implying \(x \in \ideal{p}\) or \(x^{k-1} \in \ideal{p}\).
        However, the later cannot be the case because \(k\) was assumed minimal.
        Therefore, \(x \in \ideal{p}\), and since \(\ideal{p} \subseteq \sqrt{\ideal{p}}\) (\cref{lma:ideal is subset of its radical}) it must be that \(\ideal{p} = \sqrt{\ideal{p}}\).
    \end{proof}
\end{lma}

\section{Noetherian Rings}
\begin{dfn}{Noetherian}{}
    A ring, \(R\), is noetherian if it satisfies the ascending chain condition.
    That is, if every chain of ideals,
    \begin{equation}
        I_1 \subseteq I_2 \subseteq I_3 \subseteq \dotsb
    \end{equation}
    terminates, so \(I_{n+1} = I_n\) for sufficiently large \(n\).
\end{dfn}

Note that all fields are noetherian, and so is \(\integers\).

\begin{lma}{}{lma:noetherian iff all ideals finitely generated}
    Let \(R\) be a ring.
    The following are equivalent:
    \begin{enumerate}
        \item \(R\) is a noetherian.
        \item Every ideal of \(R\) is finitely generated.
    \end{enumerate}
\end{lma}

\subsection{Hilbert's Basis Theorem}
\begin{thm}{Hilbert's Basis Theorem}{thm:hilberts basis theorem}
    If \(R\) is a noetherian ring then \(R[x]\) is also Noetherian.
\end{thm}

\begin{crl}{}{crl:poly ring over noetherian is noetherian}
    If \(R\) is a noetherian ring then \(R[x_1, \dotsc, x_n]\) is noetherian.
\end{crl}

