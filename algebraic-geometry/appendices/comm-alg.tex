\chapter{Commutative Algebra}
Here we collect some results from commutative algebra which we'll make use of in the course.
This won't be very well organised, and is more for reference than actual reading.
The conditions to be included here are pretty much \enquote{I had to look it up} or \enquote{I had to think about it for more than 10 seconds} while writing these notes, or \enquote{I thought it was worth recapping}.

\section{Ideals}

\begin{dfn}{Prime Ideal}{}
    A proper ideal, \(\ideal{p} \subideal R\), is \define{prime}\index{prime ideal} if whenever \(ab \in \ideal{p}\) for \(a, b \in R\) then either \(a \in \ideal{p}\) or \(b \in \ideal{p}\).
    
    Equivalently, \(\ideal{p}\) is prime if \(R/\ideal{p}\) is an integral domain.
\end{dfn}

\begin{dfn}{Maximal Ideal}{}
    A proper ideal, \(\ideal{m} \subideal R\), is \define{maximal}\index{maximal ideal} if whenever there is another ideal, \(I \subideal R\), with \(\ideal{m} \subseteq I\) then either \(I = \ideal{m}\) or \(I = R\).
    
    Equivalently, \(\ideal{m}\) is maximal if \(R/\ideal{m}\) is a field.
\end{dfn}

\begin{lma}{}{lma:product of ideals subset of intersection}
    Let \(R\) be a ring with ideals \(I\) and \(J\).
    Then \(IJ \subseteq I \cap J\).
    \begin{proof}
        If \(a \in I\) and \(b \in J\) then \(ab \in I\) and \(ab \in J\) by definition of an ideal.
        Then \(ab \in I \cap J\).
    \end{proof}
\end{lma}

\begin{lma}{}{lma:radical of product is radical of intersection}
    Let \(R\) be a ring with ideals \(I\) and \(J\).
    Then \(\sqrt{IJ} = \sqrt{I \cap J} = \sqrt{I} \cap \sqrt{J}\).
    \begin{proof}
        We prove a circle of inclusions.
        We start with \(\sqrt{IJ} \subseteq \sqrt{I \cap J}\), which follows from \cref{lma:product of ideals subset of intersection}.
        
        If \(a \in \sqrt{I \cap J}\) then \(a^k \in I \cap J\) for some \(k \in \naturals\).
        Thus, \(a^k \in I\) and \(a^k \in J\).
        Hence, \(a \in \sqrt{I} \cap \sqrt{J}\).
        
        If \(a \in \sqrt{I} \cap \sqrt{J}\) then \(a^k \in I\) and \(a^{\ell} \in J\) for some \(k, \ell \in \naturals\).
        Then \(a^k a^{\ell} = a^{k + \ell} \in IJ\), and so \(a \in \sqrt{IJ}\).
    \end{proof}
\end{lma}

\begin{lma}{}{lma:prime ideal is radical}
    Every prime ideal is radical.
    \begin{proof}
        Let \(\ideal{p}\) be a prime ideal of a ring, \(R\).
        Consider \(\sqrt{\ideal{p}}\).
        If \(a \in \sqrt{\ideal{p}}\) then there exists some \(k \in \naturals\) such that \(a^k \in \ideal{p}\).
        Suppose that \(k\) is minimal in making this true.
        If \(k = 1\) then \(a \in \ideal{p}\).
        If \(k > 1\) then by the definition of a prime ideal have \(x \cdot x^{k-1} \in \ideal{p}\) implying \(x \in \ideal{p}\) or \(x^{k-1} \in \ideal{p}\).
        However, the later cannot be the case because \(k\) was assumed minimal.
        Therefore, \(x \in \ideal{p}\), and since \(\ideal{p} \subseteq \sqrt{\ideal{p}}\) (\cref{lma:ideal is subset of its radical}) it must be that \(\ideal{p} = \sqrt{\ideal{p}}\).
    \end{proof}
\end{lma}

\begin{lma}{Chinese Remainder Theorem}{lma:chinese remainder theorem}
    Let \(R\) be a ring with ideals \(I\) and \(J\).
    Consider the ring homomorphism
    \begin{equation}
        \varphi \colon R \to \frac{R}{I} \times \frac{R}{J}
    \end{equation}
    which sends \(r\) to \((r \bmod I, r \bmod J)\).
    Then
    \begin{enumerate}
        \item \(\varphi\) is injective if and only if \(I \cap J = \langle 0 \rangle\);
        \item \(\varphi\) is surjective if and only if \(I + J = R\).
    \end{enumerate}
    \begin{proof}
        Notice that \(\ker \varphi\) consists of all \(r \in R\) such that \((r \bmod I, r \bmod J) = (0, 0)\), which means precisely that \(r \in I\) and \(r \in J\), so \(r \in I \cap J\).
        Thus, if \(I \cap J = \langle 0 \rangle\) then \(\ker \varphi = \langle 0 \rangle\), so \(\varphi\) is injective.
        The same logic in reverse proves the converse.
        
        Suppose that \(\varphi\) is surjective.
        Then \((1, 0)\) is in the image of \(\varphi\), so there exists some element \(r \in R\) with \((r \bmod I, r \bmod J) = (1, 0)\), which means \(r \bmod J = 0\) so \(r \in J\).
        Then \(1 - r \bmod I = 1 - 1 = 0\), so \(1 - r \in I\) and \(r \in J\), so \(1 = (1 - r) + r \in I + J\) so \(I + J = R\).
        
        Now suppose that \(I + J = R\).
        Then there exist \(a \in I\) and \(b \in J\) such that \(a + b = 1\), so \(b = 1 - a\), meaning \(b \bmod I = 1 - a \bmod I = 1 - 1 = 0\) and \(a + b \bmod I = a \bmod I = 1\).
        Thus, \((1, 0)\) is in the image of \(\varphi\).
        Similarly, \((0, 1)\) is in the image of \(\varphi\).
        Since \((1, 0)\) and \((0, 1)\) generate \(R/I \times R/J\) we are done.
    \end{proof}
\end{lma}

\begin{dfn}{Reduced Ring}{}
    A ring is \defineindex{reduced} if it has no nozero nilpotents.
\end{dfn}

\begin{lma}{}{lma:quotient by radical is reduced}
    The quotient of a ring by a radical ideal is reduced.
    \begin{proof}
        Let \(R\) be a ring and \(I \subideal R\) a radical ideal.
        That is,
        \begin{equation}
            I = \sqrt{I} = \{r \in R \mid r^n \in I \text{ for some } n \in \naturals\}.
        \end{equation}
        Suppose that \(r + I\) is nilpotent in \(R/I\).
        Then \((r + I)^n = 0\) for some \(n \in \naturals\), but \((r + I)^n = r^n + I\), and so this is \(0\) in \(R/I\) if and only if \(r^n \in I\), but then \(r \in I\) since \(I\) is radical.
        Thus, \(r + I = 0 + I\) is zero, and so there are no nonzero nilpotents, and thus \(R/I\) is reduced.
    \end{proof}
\end{lma}

\begin{lma}{}{lma:quotient reduced implies ideal radical}
    Let \(R\) be a ring with \(I \subideal R\) an ideal such that \(R/I\) is a reduced ring.
    Then \(I\) is a radical ideal of \(R\).
    \begin{proof}
        Let \(r \in R\) be such that \(r^n \in J\).
        Then in \(R/I\) we have \(0 = 0 + I = r^n + I = (r + I)^n\), and since \(R/I\) is reduced this implies that \(r + I = 0 + I\), and thus \(r \in I\), so \(I\) is radical.
    \end{proof}
\end{lma}

\begin{dfn}{Local Ring}{def:local ring}
    A \defineindex{local ring} is a ring with a unique maximal ideal.
\end{dfn}

\section{Noetherian Rings}
\begin{dfn}{Noetherian Ring}{}
    A ring, \(R\), is \define{Noetherian}\index{Noetherian!ring} if it satisfies the ascending chain condition.
    That is, if every chain of ideals,
    \begin{equation}
        I_1 \subseteq I_2 \subseteq I_3 \subseteq \dotsb
    \end{equation}
    terminates, so \(I_{n+1} = I_n\) for sufficiently large \(n\).
\end{dfn}

Note that all fields are Noetherian, and so is \(\integers\).

\begin{dfn}{Noetherian Module}{}
    Let \(R\) be a ring, and let \(M\) be an \(R\)-module.
    We say that \(M\) is \define{Noetherian}\index{Noetherian!module} if it satisfies the ascending chain condition.
    That is, if every chain of submodules,
    \begin{equation}
        M_1 \subseteq M_2 \subseteq M_3 \subseteq \dotsb
    \end{equation}
    terminates, so \(M_{n+1} = M_n\) for sufficiently large \(n\).
\end{dfn}

Note that \(R\) is Noetherian as a ring exactly when \(R\) is Noetherian as an \(R\)-module.

\begin{lma}{}{lma:noetherian iff all ideals finitely generated}
    Let \(R\) be a ring.
    The following are equivalent:
    \begin{enumerate}
        \item \(R\) is a Noetherian.
        \item Every ideal of \(R\) is finitely generated.
    \end{enumerate}
\end{lma}

\begin{lma}{}{lma:module Noetherian equivalents}
    Let \(R\) be a ring and \(M\) an \(R\)-module.
    The following are equivalent:
    \begin{enumerate}
        \item \(M\) is Noetherian.
        \item Every submodule of \(M\) is finitely generated.
        \item \label{itm:module noetherian if maximal submodule exists}Every nonempty family of submodules of \(M\) has a maximal element.
    \end{enumerate}
\end{lma}

\begin{lma}{}{lma:submodule of Noetherian is Noetherian}
    Let \(M\) be a Noetherian \(R\)-module.
    Then any submodule of \(M\) is also Noetherian.
    \begin{proof}
        Let \(N\) be a submodule of \(M\).
        Then any ascending chain, \(N_1 \subseteq N_2 \subseteq \dotsb\), in \(N\) is also an ascending chain in \(M\), and thus stabilises.
        Thus \(N\) is also Noetherian.
    \end{proof}
\end{lma}

\begin{lma}{}{lma:quotient of Noetherian is Noetherian}
    Let \(M\) be a Noetherian \(R\)-module.
    Then any quotient of \(M\) is also Noetherian.
    \begin{proof}
        Let \(N\) be a submodule of \(M\) and consider the quotient \(M/N\) with the quotient map \(\pi \colon M \twoheadrightarrow M/N\).
        Let \(P_1 \subseteq P_2 \subseteq \dotsb\) be an ascending chain in \(M/N\).
        Then setting \(M_i = \pi^{-1}(P_i)\) defines an ascending chain, \(M_1 \subseteq M_2 \subseteq \dotsb\), in \(M\), which must stabilise.
        Thus, for sufficiently large \(n\) we have \(M_{n+1} = M_n\).
        Since \(\pi\) is surjective this implies that \(P_{n+1} = \pi(M_{n+1}) = \pi(M_n) = P_n\), and thus our original chain in \(M/N\) stabilises.
    \end{proof}
\end{lma}

\subsection{Hilbert's Basis Theorem}
\begin{thm}{Hilbert's Basis Theorem}{thm:hilberts basis theorem}
    If \(R\) is a Noetherian ring then \(R[x]\) is also Noetherian.
\end{thm}

\begin{crl}{}{crl:poly ring over noetherian is noetherian}
    If \(R\) is a Noetherian ring then \(R[x_1, \dotsc, x_n]\) is Noetherian.
\end{crl}

\section{Primary Ideals}
\begin{dfn}{Primary Ideal}{}
    Let \(R\) be a ring.
    An ideal, \(Q \subideal R\), with \(Q \ne R\) is called \defineindex{primary} if for all \(a, b \in R\) with \(a b \in Q\) we have \(a \in Q\) or \(b^n \in Q\) for some \(n \in \naturals\).
\end{dfn}

Note that this definition is equivalent to \(b \in \sqrt{Q}\).

Every prime ideal is primary, with \(n = 1\).
Conversely, the definition of a primary ideal is exactly such that \(\sqrt{Q}\) is a prime ideal whenever \(Q\) is a primary ideal.

The idea to keep in mind, which is true for PIDs, is that primary ideals are powers of prime ideals.
For example, \((p^n)\) is a primary ideal of \(\integers\) for \(p\) prime (so \((p)\) is a prime ideal).

\begin{dfn}{Primary Decomposition}{def:primary decomposition}
    Let \(R\) be a ring and \(I \subideal R\) an ideal.
    Then a \defineindex{primary decomposition} of \(I\) is a finite set of primary ideals, \(Q_1, \dotsc, Q_r\), such that \(I = Q_1 \cap \dotsb \cap Q_r\).
\end{dfn}

\begin{lma}{}{lma:noetherian ring ideals have primary decomposition}
    In a Noetherian ring every ideal admits a primary decomposition.
\end{lma}

\section{Localisation}
\begin{dfn}{Multiplicatively Closed Set}{}
    Let \(R\) be a commutative ring.
    A set, \(S\), is called \defineindex{multiplicatively closed} if \(S\) is closed under multiplication and contains \(1\).
\end{dfn}

\begin{dfn}{Localisation}{}
    Let \(R\) be a ring, and let \(S\) be a multiplicatively closed subset of \(R\).
    The \defineindex{localisation} of \(R\) at \(S\) is the ring, 
    \begin{equation}
        S^{-1}R = (R \times S) / {\sim}
    \end{equation}
    where \((r, s) \sim (r', s')\) if there exists some \(t \in S\) such that
    \begin{equation}
        t(sr' - s'r) = 0.
    \end{equation}
    The operations in this ring are
    \begin{align}
        (r, s) + (r', s') &= (rs' + r's, ss'),\\
        (r, s) (r', s') &= (rr', ss').
    \end{align}
\end{dfn}

The idea is that \((r, s)\) is really the fraction, \(\frac{r}{s}\).
This is where the definition of the operations comes from.
The key example being \(R = \integers\) and \(S = \integers \setminus \{0\}\), in which case \(S^{-1}R = \rationals\).
The equivalence relation is then enforcing that things like \(1/2\) and \(2/4\) are considered the same.
The \(t\) is needed to deal with the case where there are zero divisors.
The multiplicative identity of \(S^{-1}R\) is \(1/1\), and the additive identity is \(0/s\) for any \(s \in S\) (they're all equivalent).
There's an embedding, \(R \hookrightarrow S^{-1}R\), \(r \mapsto r/1\).

Note that if \(R\) is an integral domain then \(S^{-1}R\) is its field of fractions.

If \(0 \in S\) then \(S^{-1}R = 0\).

The localisation of \(S\) in \(R\) has the universal property that the map \(i \colon R \hookrightarrow S^{-1}R\) is universal in that if \(f \colon R \to T\) is a ring homomorphism sending every element of \(S\) to a unit of \(T\) then there exists a unique ring homomorphism, \(g \colon S^{-1}R \to T\), such that \(f = g \circ i\).
That is,
\begin{equation}
    \begin{tikzcd}
        R \arrow[r, "i"] \arrow[dr, "f"'] & S^{-1}R \arrow[d, dashed, "\exists ! g"]\\
        & T.
    \end{tikzcd}
\end{equation}

\begin{ntn}{}{}
    Let \(R\) be a commutative ring and \(x \in R\).
    Then we write \(R_x\) for \(S^{-1}R\) with \(S = \{1, x, x^2, \dotsc\}\).
    
    Let \(\ideal{p}\) be a prime ideal of \(R\).
    Then we write \(R_{\ideal{p}}\) for \(S^{-1}R\) with \(S = R \setminus \ideal{p}\).
\end{ntn}

This notation can be a bit confusing.
For \(R_x\) we've forced \(x\) to be invertible.
For \(R_{\ideal{p}}\) we've forced everything \emph{not} in \(\ideal{p}\) to be invertible.
