% !TeX program = lualatex
\documentclass[fleqn]{NotesClass}

\strictpagecheck

\usepackage{csquotes}
\usepackage{enumitem}

\usepackage{tikz}
\usetikzlibrary{external}
\tikzexternalize[prefix=tikz-external/]

\usepackage{tikz-cd}
\AtBeginEnvironment{tikzcd}{\tikzexternaldisable}
\AtEndEnvironment{tikzcd}{\tikzexternalenable}

\usepackage[pdfauthor={Willoughby Seago},pdftitle={Notes from Representation Theory Course},pdfkeywords={representation theory},pdfsubject={Representation Theory}]{hyperref}  % Should be loaded second last (cleveref last)
\colorlet{hyperrefcolor}{blue!60!black}
\hypersetup{colorlinks=true, linkcolor=hyperrefcolor, urlcolor=hyperrefcolor}
\usepackage[
capitalize,
nameinlink,
noabbrev
]{cleveref} % Should be loaded last

% My packages
\usepackage{NotesBoxes}
\usepackage{NotesMaths2}

\setmathfont[range={\int, \oint, \otimes, \oplus, \bigotimes, \bigoplus}]{Latin Modern Math}


% Highlight colour
%\definecolor{highlight}{HTML}{710D78}
%\definecolor{my blue}{HTML}{2A0D77}
%\definecolor{my red}{HTML}{770D38}
%\definecolor{my green}{HTML}{14770D}
%\definecolor{my yellow}{HTML}{E7BB41}

% Title page info
\title{Representation Theory}
\author{Willoughby Seago}
\date{January 13th, 2024}
\subtitle{Notes from}
\subsubtitle{University of Glasgow}
\renewcommand{\abstracttext}{These are my notes from the SMSTC course \emph{Lie Theory} taught by Prof Christian Korff. These notes were last updated at \printtime{} on \today{}.}

% Commands
% Maths
\renewcommand{\field}{\symbb{k}}
\newcommand{\id}{\symrm{id}}
\DeclareMathOperator{\End}{End}
\DeclareMathOperator{\Hom}{Hom}
\newcommand{\action}{\mathbin{.}}
\newcommand{\op}{\symrm{op}}
\makeatletter
\newcommand{\c@egory}[1]{\symsfup{#1}}
\newcommand{\Vect}[1][\field]{#1\text{-}\c@egory{Vect}}
\newcommand{\AMod}[1][A]{#1\text{-}\c@egory{Mod}}
\newcommand{\ModA}[1][A]{\c@egory{Mod}\text{-}#1}
\newcommand{\Ab}{\c@egory{Ab}}
\newcommand{\Rep}{\c@egory{Rep}}
\makeatother
\DeclareMathOperator{\im}{im}
\newcommand{\isomorphic}{\cong}
\DeclareMathOperator{\Span}{span}
\ExplSyntaxOn
% Create LaTeX interface command
\NewDocumentCommand{\cycle}{ O{\,} m }{  % optional arg is separator, mandatory
    %arg is comma separated list
    (
    \willoughby_cycle:nn { #1 } { #2 }
    )
}

\clist_new:N \l_willougbhy_cycle_clist  % Create new clist variable
\cs_new_protected:Npn \willoughby_cycle:nn #1 #2 {  % create LaTeX3 function
    \clist_set:Nn \l_willougbhy_cycle_clist { #2 }  % set clist variable with
    %clist #2 passed by user
    \clist_use:Nn \l_willougbhy_cycle_clist { #1 }  % print list separated by #1
}
\ExplSyntaxOff
\newcommand{\universalEnveloping}{\symcal{U}}
\DeclarePairedDelimiterX{\bracket}[2]{[}{]}{#1, #2}

\begin{document}
    \frontmatter
    \titlepage
    \innertitlepage{}
    \tableofcontents
    % \listoffigures
    \mainmatter
    \chapter{Introduction}
    We fix some standard notation here:
    \begin{itemize}
        \item \(\field\) will denote an algebraically closed field, except for when we explicitly mention that the field needn't be algebraically closed.
        \item \(A\) will denote an associative unital algebra.
        \item Letters like \(V\), \(U\), and \(W\) will denote vector spaces over \(\field\).
        \item Letters like \(M\) and \(N\) will denote modules.
    \end{itemize}
    
    \chapter{Initial Definitions}
    \section{Algebra}
    \begin{dfn}{Algebra}{}
        An \defineindex{algebra} is a \(\field\)-vector space, \(A\), equipped with a bilinear map,
        \begin{align}
            m \colon A \times A &\to A\\
            (a, b) &\mapsto m(a, b) = ab.
        \end{align}
        
        If this map satisfies the condition that
        \begin{equation}
            m(a, m(b, c)) = m(m(a, b), c), \text{ or equivalently } a(bc) = (ab)c,
        \end{equation}
        for all \(a, b, c \in A\) then we call \(A\) an \defineindex{associative algebra}.
        
        If \(A\) posses a distinguished element, \(1 \in A\), such that \(m(1, a) = a = m(a, 1)\), or equivalently \(1a = a = a1\) for all \(a \in A\) then we say that \(A\) is a \defineindex{unital algebra}.
        
        If \(m(a, b) = m(b, a)\), or equivalently \(ab = ba\), for all \(a, b \in A\) then we say that \(A\) is a \defineindex{commutative algebra}.
    \end{dfn}
    
    Whenever we say, otherwise unqualified, \enquote{algebra} we will mean associative unital algebra unless we specify otherwise.
    We will not assume commutativity of a general algebra.
    
    The condition of associativity can be written as a commutative diagram,
    \begin{equation}
        \begin{tikzcd}
            A \times A \times A \arrow[r, "m \times \id_A"] \arrow[d, "\id_A \times m"'] & A \times A \arrow[d, "m"]\\
            A \times A \arrow[r, "m"'] & A\mathrlap{,}
        \end{tikzcd}
    \end{equation}
    
    \begin{remark}{}{}
        This diagram goes part of the way to the more abstract definition that \enquote{an associative unital (commutative) algebra is a (commutative) monoid in the category of vector spaces}.
        This definition is nice because it is both very general and dualises to the notion of a coalgebra.
        See the \textit{Hopf Algebra} notes for more details.
    \end{remark}
    
    \begin{exm}{}{}
        \begin{itemize}
            \item \(A = \field\) is an algebra with the product given by the product in the field;
            \item \(A = \field[x_1, \dotsc, x_n]\), the ring of polynomials in the variables \(x_i\) with coefficients in \(\field\), is an algebra under the addition and multiplication of polynomials.
            \item \(A = \field \langle x_1, \dotsc, x_n \rangle\), the free algebra on \(x_i\), may be considered as the algebra of polynomials in non-commuting variables, \(x_i\).
            \item \(A = \End V\) for \(V\) a \(\field\)-vector space is an algebra with multiplication given by composition of morphisms.
        \end{itemize}
    \end{exm}
    
    \begin{dfn}{Group Algebra}{}
        Let \(G\) be a group.
        The \defineindex{group algebra} or \defineindex{group ring} \(\field G = \field[G]\) is defined to be the set of finite formal linear combinations
        \begin{equation}
            \sum_{g \in G} c_g g
        \end{equation}    
        where \(c_g \in \field\) is nonzero for only finitely many values \(g\).
        Addition is defined by
        \begin{equation}
            \sum_{g \in G} c_g g + \sum_{g \in G} d_g g = \sum_{g \in G} (c_g + d_g) g.
        \end{equation}
        Multiplication is defined by requiring that it distributes over addition and that the product of two terms in the above sums is given by
        \begin{equation}
            (c_g g) (d_h h) = (c_g d_h) (gh)
        \end{equation}
        where multiplication on the left is in \(\field G\), the multiplication \(c_g d_h\) is in \(\field\), and the multiplication \(gh\) is in \(G\).
        
        If we do the same construction replacing \(\field\) with a ring, \(R\), then we get the group ring, \(RG\), which is not an algebra but instead an \(R\)-module.
    \end{dfn}
    
    \begin{dfn}{Algebra Homomorphism}{}
        Let \(A\) and \(B\) be \(\field\)-algebras.
        An \defineindex{algebra homomorphism} is a linear map \(f \colon A \to B\) such that \(f(ab) = f(a)f(b)\) for all \(a, b \in A\).
        
        If \(A\) and \(B\) are unital, with units \(1_A\) and \(1_B\) respectively, then we further require that \(f(1_A) = 1_B\).
        
        We denote by \(\Hom(A, B)\) or \(\Hom_{\field}(A, B)\) the set of all algebra homomorhpisms \(A \to B\).
    \end{dfn}
    
    If \(m_A\) and \(m_B\) denote the multiplication maps of \(A\) and \(B\) respectively then we may think of a homomorphism, \(f\), as a linear map which \enquote{commutes} with the multiplication map, that is \(f \circ m_A = m_B \circ f\).
    
    Alternatively, an algebra, \(A\) is both an abelian group under addition, and a monoid under multiplication, and an algebra homomorhpism is both a group and monoid homomorphism with respect to these structures.
    
    \section{Representations and Modules}
    \label{sec:representaitons and modules}
    There are two competing terminologies in the field, with slightly different notation and emphasis depending on which we use.
    We'll use the more modern notion of modules most of the time, but will occasionally and interchangeably use the notion of representations as well.
    
    \begin{dfn}{Representation}{}
        Let \(V\) be a \(\field\)-vector space and \(A\) a \(\field\)-algebra.
        Any \(\rho \in \Hom(A, \End V)\) is called a \defineindex{representation} of \(A\).
        That is, a representation of \(A\) is an algebra homomorphism \(\rho \colon A \to \End V\).
    \end{dfn}
    
    \begin{dfn}{Module}{}
        Let \(A\) be a \(\field\)-algebra.
        A \define{left \(\symbf{A}\)-module}\index{left A-module@left \(A\)-module}, \(M\), is an abelian group, with the binary operation denoted \(+\), equipped with a \defineindex{left action}
        \begin{align}
            \action \colon A \times M &\to M\\
            (a, m) &\mapsto a \action m
        \end{align}
        such that for all \(a, b \in A\) and \(m, n \in M\) we have\footnote{Note that M1 and M2 simply say that this is a group action on the set \(M\), and M3 and M4 two impose that this group action is compatible with both the group operation and addition in the algebra.}
        \begin{itemize}
            \item[M1] \((ab)\action m = a\action (b\action m)\) (note that \((ab)\) is the product in \(A\));
            \item[M2] \(1 \action m = m\).
            \item[M3] \(a\action(m + n) = a\action m + a\action n\);
            \item[M4] \((a + b)\action m = a\action m + b\action m\);
        \end{itemize}
        
        One can similarly define a \define{right \(\symbf{A}\)-module}\index{right A-module@right \(A\)-module}, \(M\), as an abelian group with a \defineindex{right action}
        \begin{align}
            \action \colon M \times A &\to M\\
            (m, a) &\mapsto m \action a
        \end{align}
        such that for all \(a, b \in A\) and \(m, n \in M\) we have
        \begin{itemize}
            \item[M1] \((m + n) \action a = m \action a + n \action a\);
            \item[M2] \(m \action (a + b) = m \action a + m \action b\);
            \item[M3] \(m \action (ab) = (m \action a) \action b\);
            \item[M4] \(m \action 1 = m\).
        \end{itemize}
        
        A \define{two-sided \(\symbf{A}\)-module}\index{two-sided A-module@two-sided \(A\)-module}\index{\(A\)-module}\index{module} is then an abelian group, \(M\), which is simultaneously a left and right \(A\)-module satisfying
        \begin{equation}
            a \action (m \action b) = (a \action m) \action b
        \end{equation}
        for all \(a, b \in A\) and \(m \in M\).
    \end{dfn}
    
    When it doesn't risk confusion we will write \(a \action m\) as \(am\) and \(m \action a\) as \(ma\).
    
    Note that a module is a generalisation of the notion of a vector space.
    In fact, if \(A = \field\) then a module is exactly a vector space.
    
    More compactly, one can define a right \(A\)-module as a left \(A^{\op}\)-module, where \(A^{\op}\) is the \defineindex{opposite algebra} of \(A\), defined to be the same underlying vector space with multiplication \(*\) defined by \(a * b = ba\), where \(ba\) is the multiplication in \(A\).
    Because of this we will almost never have reason to work with right modules, we can always turn them into a left module over the opposite algebra instead.
    
    Note that if \(A\) is commutative every left \(A\)-module is a right \(A\)-module and vice versa, and also a two-sided module.
    
    Without further clarification the term \enquote{module} will mean
    \begin{itemize}
        \item a left module if \(A\) is not necessarily commutative;
        \item a two sided module if \(A\) is commutative.
    \end{itemize}
    
    A representation of \(A\) and an \(A\)-module carry exactly the same information.
    Given a representation, \(\rho \colon A \to \End V\) we may define a group action on \(V\) by \(a \action v = \rho(a)v\).
    Composition in \(\End V\) is exactly repeated application of this action: \([\rho(a)\rho(b)]v = \rho(a)[\rho(b)v]\) (M1).
    The unit of \(\End V\) is the identity morphism, \(\id_V\), and \(1 \in A\) must map to \(\id_V\), so \(\rho(1)v = \id_V v = v\) (M2).
    Linearity of \(\rho(a)\) means that \(\rho(a)(v + w) = \rho(a)v + \rho(v)w\) (M3).
    Linearity of \(\rho\) means that \(\rho(a + b)v = \rho(a)v + \rho(b)v\) (M4).
    
    Conversely, given an \(A\)-module, \(M\), we can define scalar multiplication by \(\lambda \in \field\) on \(M\) by \(\lambda m = (\lambda 1) m\) where \(\lambda 1\) is scalar multiplication in \(A\).
    This makes \(M\) a vector space, and we may define a morphism \(\rho \colon A \to \End M\) by defining \(\rho(a)\) by \(\rho(a) = a \action m\), which uniquely determines \(\rho(a)\), say by considering the action on some fixed basis of \(M\).
    
    Further, these two constructions are inverse, given a module if we construct the corresponding representation then construct the corresponding module from that we get back the original module, and vice versa.
    This means that the notion of a representation and a module really are the same, and we don't need to distinguish between them.
    We will use whichever terminology and notation is better suited to the problem, which is usually the module terminology and notation.
    
    \begin{prp}{}{}
        Let \(V\) be a \(\field\)-vector space, \(G\) a group, and \(\rho \colon G \to GL(V)\) a group homomorphism.
        We may define a \(\field G\)-module by extending this map linearly, defining
        \begin{equation}
            \left( \sum_{g \in G} c_g g \right) \action v = \sum_{g \in G} c_g \rho(g)v.
        \end{equation}
        Conversely, given a left \(\field G\)-module on \(V\) we may define a group homomorphism \(\rho \colon G \to \generalLinear(V)\) by defining \(\rho(g)\) to be the linear operation \(v \mapsto g \action v\).
        \begin{proof}
            This is just a special case of the equivalence of representations and modules discussed above.
        \end{proof}
    \end{prp}
    
    Note that a \defineindex{group representation} is defined to be a group homomorphism \(\rho \colon G \to \generalLinear(V)\).
    The above result shows that a group representation of \(G\) is exactly the same as an algebra representation of \(\field G\), so we can just study algebras.
    
    \begin{dfn}{Regular Representation}{}
        Let \(V = A\) be an algebra and define \(\rho \colon A \to \End A\) by \(\rho(a)b = ab\).
        This is called the \defineindex{left regular representation}.
        Similarly, the \defineindex{right regular representation} is given by defining \(\rho(a)b = ba\).
    \end{dfn}
    
    \section{Direct Sums}
    The goal of much of representation theory is to classify possible representations.
    To do this we usually decompose representations into smaller parts that can be more easily classified.
    This decomposition is done by the direct sum.
    
    \begin{dfn}{Direct Sum}{}
        Let \(M\) and \(N\) be \(A\)-modules.
        The \defineindex{direct sum}, \(M \oplus N\), is the \(A\)-module given by the direct sum of the underlying abelian groups equipped with the action
        \begin{equation}
            a(m \oplus n) = am \oplus an
        \end{equation}
        for all \(a \in A\), \(m \in M\) and \(n \in N\).
    \end{dfn}
    
    The required properties follow immediately from the definition:
    \begin{itemize}
        \item[M1] \((ab)(m \oplus n) = (ab)m \oplus (ab)n = a(bm) \oplus a(bn) = a(bm \oplus bn) = a(b(m \oplus n))\);
        \item[M2] \(1(m \oplus n) = 1m \oplus 1n = m \oplus n\);
        \item[M3] \(a((m \oplus n) + (m' \oplus n')) = a((m + m') \oplus (n + n')) = a(m + m') \oplus a(n + n') = (am + am') \oplus (an + an') = (am \oplus an) + (am' \oplus an') = a(m \oplus n) + a(m' \oplus n')\);
        \item[M4] \((a + b)(m \oplus n) = (a + b)m \oplus (a + b)n = (am + bm) \oplus (an + bn) = (am \oplus an) + (bm \oplus bn) = a(m \oplus n) + b(m \oplus n)\).
    \end{itemize}
    
    \begin{dfn}{Submodule}{}
        Let \(M\) be a left \(A\)-module.
        An abelian subgroup \(N \trianglelefteq M\) is a \define{\(\symbf{A}\)-submodule}\index{submodule} if \(AN \subseteq N\).
        In this case we say that \(N\) is \defineindex{invariant} under the action of \(A\).
    \end{dfn}
    
    Note that by \(AN\) we mean
    \begin{equation}
        AN = \{an \mid a \in A , n \in N\}.
    \end{equation}
    So \(AN \subseteq N\) means that \(an \in N\) for all \(a \in A\) and \(n \in N\).
    Thus, invariance means that no element of \(N\) leaves \(N\) under the action of \(A\).
    
    \begin{dfn}{Trivial Submodule}{}
        Every \(A\)-module, \(M\), admits two submodules, \(M\) itself and the zero module, \(0\), which contains only \(0\).
        We call these \define{trivial submodules}\index{trivial submodule}.
    \end{dfn}
    
    Note that some texts call only \(0\) the trivial submodule, and make the distinction of a submodule vs a \emph{proper} submodule, the distinction being that \(M\) is not a proper submodule of \(M\).
    Then when we say \enquote{nontrivial submodule} these texts will say \enquote{nontrivial proper submodule}.
    
    \begin{dfn}{Simple Submodule}{}
        Let \(M\) be an \(A\)-module.
        We say that \(M\) is \defineindex{simple} or \defineindex{irreducible} if it contains no nontrivial submodules.
    \end{dfn}
    
    Typically \enquote{simple} is used for modules and \enquote{irreducible} is used more for representations, although irreducible is used for both.
    
    \begin{dfn}{Semisimple}{}
        Let \(M\) be an \(A\)-module.
        Then \(M\) is \defineindex{semisimple} or \defineindex{completely reducible} if it can be written as a direct sum of finitely many simple modules.
    \end{dfn}
    
    That is, \(M\) is semisimple if 
    \begin{equation}
        M = \bigoplus_{i=1}^n N_i = N_1 \oplus \dotsb \oplus N_n
    \end{equation}
    where each \(N_i\) is simple.
    Note that we define the empty sum to be the zero module, so the zero module is considered semisimple (and also simple, since it contains only itself as a submodule).
    
    Again, \enquote{semisimple} is typically used only for modules, and \enquote{completely reducible} is used primarily for representations.
    
    \begin{dfn}{Indecomposable}{}
        Let \(M\) be an \(A\)-module.
        Then \(M\) is \defineindex{indecomposable} if \(M\) cannot be written as a direct sum of nontrivial modules.
    \end{dfn}
    
    The nontrivial requirement here just rules out decompositions of the form\footnote{Note that with our definition of the direct sum this really only holds up to isomorphism, since \(M\) has elements \(m\) whereas \(M \oplus 0\) has elements \((m, 0)\). However, we're yet to define morphisms between modules, and once we do we'll see that \(\oplus\) is the product in the category of modules, and as such is only defined up to isomorphism, so we may as well momentarily take the isomorphism that makes this equality true.} \(M = M \oplus 0\).
    
    Note that every simple (irreducible) module is indecomposable, since if it had a decomposition \(M = N_1 \oplus N_2\) with \(N_i\) nontrivial then their is a canonical copy of each \(N_i\) as a submodule of \(M\).
    The converse does not hold in general, not all indecomposable modules are irreducible.
    It is possible that \(M\) contains a submodule, \(N\), but that there is no submodule \(N'\) such that \(M = N \oplus N'\).
    Contrast this to finite dimensional vector spaces where we can take \(N'\) to be the orthogonal complement (with respect to some inner product) of \(N\) and this direct sum holds.
    We can still form the orthogonal complement of a submodule, but it will not, in general, be a submodule.
    There are, however, many special cases, such as finite dimensional complex representations of (group algebras) finite groups, where the orthogonal complement can be defined in such a way that it is a submodule, and in this case indecomposable and irreducible coincide.
    
    One of the main goals of representation theory is to classify all indecomposable modules of a given algebra.
    This then gives us an understanding of \emph{all} modules over that algebra, since any nonsimple or decomposable module may be realised as a direct sum of these classified indecomposable modules.
    
    \section{Module Homomorphisms}
    \begin{dfn}{Module Homomorphism}{}
        Let \(M\) and \(N\) be \(A\)-modules.
        An \define{\(\symbf{A}\)-module homomorphism}\index{module homomorphism} or \defineindex{intertwiner} is a homomorphism of the underlying abelian groups \(\varphi \colon M \to N\) which \enquote{commutes} with the action of \(A\), by which we mean
        \begin{equation}
            \varphi(a \action m) = a \action \varphi(m)
        \end{equation}
        for all \(a \in A\) and \(m \in M\).
        
        An invertible \(A\)-module homomorphism is called an \defineindex{isomorphism} of \(A\)-modules.
        
        Homomorphisms of right \(A\)-modules may be defined similarly.
    \end{dfn}
    
    \begin{ntn}{}{}
        We write \(\Hom_A(M, N)\) for the set of \(A\)-module homomorphisms \(M \to N\).
        Note that \(\Hom_A(M, N) \subseteq \Hom_{\Ab}(M, N)\) where \(\Hom_{\Ab}(M, N)\) is the set of all homomorphisms \(M \to N\) of the underlying abelian groups.
    \end{ntn}
    
    Note that in \(\varphi(a \action m)\) \(a\) is acting on an element of \(M\), and in \(a \action \varphi(m)\) \(a\) is acting on an element of \(N\), so these are in general different actions.
    Writing \(a \action {}\) for the map \(x \mapsto a \action x\) we can express the condition of commuting action as the commutativity of the diagram
    \begin{equation}
        \begin{tikzcd}
            M \arrow[r, "\varphi"] \arrow[d, "a \action {}"'] & N \arrow[d, "a \action {}"]\\
            M \arrow[r, "\varphi"'] & N
        \end{tikzcd}
    \end{equation}
    for all \(a \in A\).
    
    \begin{lma}{}{}
        Isomorphisms of \(A\)-modules are exactly bijective morphisms of \(A\)-modules.
        \begin{proof}
            Let \(\varphi \colon M \to N\) be a bijective morphism of \(A\)-modules.
            Then the (set-theoretic) inverse, \(\varphi^{-1} \colon N \to M\), exists.
            We claim that this is a morphism of \(A\)-modules.
            This follows by taking \(n \in N\) to be the image of \(m \in M\) under \(\varphi\), giving
            \begin{equation}
                \varphi^{-1}(a \action n) = \varphi^{-1}(a \action \varphi(m)) = \varphi^{-1}(\varphi(a \action m)) = a \action m = a \action \varphi^{-1}(m).
            \end{equation}
            
            Conversely, if \(\varphi \colon M \to N\) is an isomorphism of \(A\)-modules it must necessarily be that \(\varphi^{-1}\) is the (set-theoretic) inverse of the underlying function of \(\varphi\), and so \(\varphi\) must be bijective.
        \end{proof}
    \end{lma}
    
    If we instead talk of representations \((V, \rho)\) and \((W, \sigma)\) then a homomorphism of representations, \(\varphi \colon V \to W\), must satisfy \(\varphi(\rho(a)v) = \sigma(a)\varphi(v)\).
    Further, by linearity of \(\rho\) and \(\sigma\) and the fact that \(\rho(1) = \id_V\) and \(\sigma(1) = \id_W\) we have that for \(\lambda \in \field\)
    \begin{equation}
        \varphi(\lambda m) = \varphi(\rho(1)\lambda m) = \varphi(\rho(\lambda 1) m) = \sigma(\lambda 1)\varphi(m) = \lambda \sigma(1) \varphi(m) = \lambda \varphi(m).
    \end{equation}
    This shows that \(\varphi\) must be a linear map \(\varphi \colon V \to W\).
    In fact, we can \emph{define} a homomorphism of representations to be a linear map \(\varphi \colon M \to N\) satisfying \(\varphi(\rho(a)m) = \sigma(a)\varphi(m)\).
    We will also write \(\Hom_A(V, W)\) for the set of representation morphisms \(V \to W\).
    Note then that \(\Hom_A(V, W) \subseteq \Hom_{\Vect[\field]}(V, W)\) where \(\Hom_{\Vect[\field]}(V, W)\) is the set of linear maps \(V \to W\) of the underlying vector spaces.
    Using the notation \(\Hom_A\) for both modules and representations is justified by the following remark.
    
    \begin{remark}{}{}
        There is a category, \(\AMod\) (\(\ModA\)), with left (right) \(A\)-modules as objects and \(A\)-module homomorphisms as morphisms.
        Similarly, there is a category \(\Rep(A)\) of representations of \(A\) with objects being representations \((V, \rho)\) and morphisms being homomorphisms of representations.
        
        In \cref{sec:representaitons and modules} we showed that we have a mapping \(F \colon \AMod \to \Rep(A)\) constructing a representation from a module, and a mapping \(G \colon \Rep(A) \to \AMod\) constructing a module from a representation.
        In the discussion above we extend this mapping to define a representation homomorphism from a module homomorphism.
        We can also ignore the requirement of linearity with respect to scalar multiplication in the definition of a representation homomorphism to recover a module homomorphism.
        Further, applying either of these constructions to the appropriate identity map just gives the identity, and both constructions preserve composition.
        These operations on homomorphisms are also inverses of each other.
        Thus, \(F\) and \(G\) are functors and we have \(FG = \id_{\Rep(A)}\) and \(GF = \id_{\AMod}\).
        Thus, \(\AMod\) and \(\Rep(A)\) are isomorphic as categories, justifying the fact that we will soon cease to distinguish between them.
    \end{remark}
    
    \begin{lma}{}{}
        The category \(\AMod\) defined above is indeed a category.
        \begin{proof}
            First note that \(\id_M \colon M \to M\) is an \(A\)-module homomorphism for any \(A\)-module, \(M\), since we have
            \begin{equation}
                \id_M(a \action m) = a \action m = a \action \id_M(m).
            \end{equation}
            Now note that if \(\varphi \colon M \to N\) and \(\psi \colon N \to P\) are module homomorphisms then \(\psi \circ \varphi \colon M \to P\) is a module homomorphism since
            \begin{equation*}
                (\psi \circ \varphi)(a \action m) = \psi(\varphi(a \action m)) = \psi(a \action \varphi(m)) = a \action \psi(\varphi(m)) = a \action (\psi \circ \varphi)(m)
            \end{equation*}
            for all \(a \in A\) and \(m \in M\).
            Finally, composition is just composition of the underlying functions, which is associative.
        \end{proof}
    \end{lma}
    
    \section{Schur's Lemma}
    We can now give one of the first results of representation theory.
    It places a restriction on the types of morphisms we can have between modules when one or more of the modules is simple.
    We give the result as a proposition and a corollary, although for historical reasons it's called a lemma.
    The proposition is more general, and the corollary is a special case.
    Both are known as Schur's lemma, with context determining if we use the more general result or the special case.
    
    Before we can prove this result however we need a couple of results about kernels and images of module morphisms.
    
    \begin{lma}{}{}
        Let \(\varphi \colon M \to N\) be a morphism of modules.
        Then \(\ker \varphi\) is a submodule of \(M\) and \(\im \varphi\) is a submodule of \(N\).
        \begin{proof}
            \Step{\(\ker \varphi\)}
            We know that \(\ker \varphi\) is a subgroup of \(M\), so we only need to show that it is invariant under the action of \(A\).
            Take \(m \in \ker \varphi\), that is \(m \in M\) is such that \(\varphi(m) = 0\), and \(a \in A\).
            Then
            \begin{equation}
                \varphi(a \action m) = a \action \varphi(m) = a \action 0.
            \end{equation}
            For arbitrary \(m' \in M\) we have
            \begin{equation}
                a \action 0 = a \action (m' - m') = (a \action m') - (a \action m') = 0
            \end{equation}
            so \(a \action 0 = 0\) for any \(a \in A\), and thus \(\varphi(a \action m) = a \action 0 = 0\), so \(a \action m \in \ker \varphi\).
            
            \Step{\(\im \varphi\)}
            We know that \(\im \varphi\) is a subgroup of \(M\), so we only need to show that it is invariant under the action of \(A\).
            Take \(n \in \im \varphi\) and \(a \in A\).
            There exists some \(m \in M\) such that \(n = \varphi(m)\).
            Then
            \begin{equation}
                a \action n = a \action \varphi(m) = \varphi(a \action m)
            \end{equation}
            and \(a \action m \in M\) so \(a \action n \in \im \varphi\).
        \end{proof}
    \end{lma}
    
    \begin{prp}{Schur's Lemma}{prp:schurs lemma}
        Let \(\field\) be any (not necessarily algebraically closed) field, and let \(A\) be an algebra over \(\field\).
        Let \(M\) and \(N\) be \(A\)-modules and let \(\varphi \colon M \to N\) be a morphism of \(A\)-modules.
        Then
        \begin{enumerate}
            \item if \(M\) is simple either \(\varphi = 0\) or \(\varphi\) is injective;
            \item if \(N\) is simple either \(\varphi = 0\) or \(\varphi\) is surjective.
        \end{enumerate}
        Combined if \(M\) and \(N\) are simple then either \(\varphi = 0\) or \(\varphi\) is an isomorphism.
        \begin{proof}
            \Step{\(M\) Simple}
            Let \(M\) be simple, so its only submodules are \(0\) and \(M\).
            We know that \(\ker \varphi\) is a submodule of \(M\), so there are two cases to consider:
            \begin{itemize}
                \item If \(\ker \varphi = M\) then every element of \(M\) maps to \(0\), so \(\varphi = 0\).
                \item If \(\ker \varphi = 0\) then \(\varphi\) is injective\footnote{We know that for group homomorphisms if the kernel is trivial then the map is injective, and injectivity is a set-theoretic property, so it still holds when we add the extra structure of the \(A\)-action}.
            \end{itemize}
            
            \Step{\(N\) Simple}
            Let \(N\) be simple, so its only submodules are \(0\) and \(N\).
            We know that \(\im \varphi\) is a submodule of \(N\), so there are two cases to consider:
            \begin{itemize}
                \item If \(\im \varphi = 0\) then every element of \(M\) maps to \(0\), so \(\varphi = 0\).
                \item If \(\im \varphi = N\) then \(\varphi\) is surjective.
            \end{itemize}
        \end{proof}
    \end{prp}
    
    \begin{crl}{Schur's Lemma}{crl:schurs lemma}
        Let \(\field\) be an algebraically closed field, and let \(A\) be an algebra over \(\field\).
        Let \(V\) be a finite dimensional representation of \(A\).
        Then any representation homomorphism \(\varphi \colon V \to V\) is a multiple of the identity.
        That is, \(\varphi = \lambda \id_V\) for \(\lambda \in \field\).
        Note that \(\lambda = 0\) subsumes the trivial case.
        \begin{proof}
            Let \(\lambda \in \field\) be an eigenvalue of \(\varphi\) with corresponding eigenvector \(v \in V\).
            Note that eigenvalues exist because
            \begin{enumerate}[label={\alph*)}]
                \item \(V\) is finite dimensional so the determinant may be defined as a polynomial in the entries of some matrix representing \(\varphi\) in a fixed basis; and
                \item \(\field\) is algebraically closed, so this polynomial has roots.
            \end{enumerate}
            Then by definition \(\varphi(v) = \lambda v\) which we can rearrange to \((\varphi - \lambda \id_V) v = 0\).
            Thus, \(v \in \ker(\varphi - \lambda \id_V)\), and since eigenvectors are, by definition, nonzero this means that \(\ker(\varphi - \lambda \id_V) \ne 0\), so \(\varphi - \lambda \id_V\) is not injective, so by Schur's lemma (\cref{prp:schurs lemma}) we must have that \(\varphi - \lambda \id_V = 0\).
            Thus, \(\varphi = \lambda \id_V\).
        \end{proof}
    \end{crl}
    
    \begin{crl}{}{crl:commutative algebra irreps are one dimensiona}
        Let \(A\) be a commutative algebra over an algebraically closed field, \(\field\).
        Then all nontrivial finite dimensional irreducible representations of \(A\) are one dimensional.
        \begin{proof}
            Let \(V\) be a finite dimensional irreducible representation of \(A\).
            For \(a \in A\) define a map \(\varphi_a \colon V \to V\) by \(v \mapsto \varphi_a(v) = a \action v\).
            This is an intertwiner: take \(b \in A\) and \(v \in V\), then we have
            \begin{equation}
                \varphi_a(b \action v) = a \action (b \action v) = (ab) \action v = (ba) \action v = b \action (a \action v) = b \action \varphi_a(v).
            \end{equation}
            Note that this is only true because \(ab = ba\).
            
            By Schur's lemma (\cref{crl:schurs lemma}) there exists some \(\lambda_a \in \field\) such that \(\varphi_a = \lambda_a \id_V\).
            Then \(a \action v = \varphi_a(v) = \lambda_a v\), so every \(a \in A\) acts as scalar multiplication.
            This means that any subspace is invariant, since every subspace is, by definition, invariant under scalar multiplication.
            Thus, the only way that a representation can have no nontrivial invariant subspaces if if it only has trivial subspaces, which is only true if it is one dimensional (zero dimensional being ruled out by the assumption that the representation is nontrivial).
        \end{proof}
    \end{crl}
    
    \begin{exm}{}{}
        Consider \(A = \field[x]\), which is a commutative algebra.
        We can determine all irreducible representations of \(A\).
        
        A representation, \(\rho \colon \field[x] \to \End V\), is fully determined by the value of \(\rho(x)\), since given an arbitrary polynomial, \(f(x) = \sum_{i=1}^{n} a_i x^i\), its action on \(v \in V\) is determined through linearity by
        \begin{equation}
            f(x) \action v = \rho(f(x)) v = \rho\left( \sum_{i=1}^{n} a_i x^i \right) v = \sum_{i=1}^n a_i \rho(x)^i v.
        \end{equation}
        
        Further, by \cref{crl:commutative algebra irreps are one dimensiona} we know that any irreducible representation of \(\field[x]\) is one dimensional, so it must be that \(\rho(v) = \lambda v\) for some \(\lambda \in \field\).
        
        Let \(V_\lambda\) denote the one-dimensional representation in 
        which \(x\) acts as scalar multiplication by \(\lambda\).
        We claim that \(V_\lambda \isomorphic V_{\mu}\) if and only if \(\lambda = \mu\).
        Suppose that \(\varphi \colon V_\lambda \to V_\mu\) is an isomorphism.
        Then \(\varphi(x \action v) = \varphi(\lambda v) = \lambda \varphi(v)\) and \(\varphi(x \action v) = x \action \varphi(v) = \mu \varphi(v)\).
        Thus, \(\lambda = \mu\).
        
        So, we have classified all irreducible representations of \(\field[x]\), they are precisely the one dimensional vector spaces, \(V_\lambda\) for \(\lambda \in \field\) in which \(\rho(x) = \lambda \id_{V_\lambda}\).
        
        This result generalises to polynomials in an arbitrary number of variables, \(\field[x_1, \dotsc, x_n]\).
        Then a representation is fully determined by the values of \(\rho(x_1)\) through \(\rho(x_n)\).
        Thus an irreducible representation is a one dimensional vector space, \(V_{\lambda_1, \dotsc, \lambda_n}\) in which \(\rho(x_i) = \lambda_i \id_{V_{\lambda_1, \dotsc, \lambda_n}}\).
        
        Go back to the case of \(A = \field[x]\).
        For a nontrivial (\(\lambda \ne 0\)) finite dimensional irreducible representation, \(V_\lambda\), instead of starting with the action of \(x\) we can perform a change of variables and work with \(y = x/\lambda\).
        Then we get the representation \(V_1\).
        This means that all finite dimensional irreducible representations of \(\field[x]\) are essentially the same, up to rescaling.
        This also means that they're pretty boring.
        
        Indecomposable representations of \(\field[x]\) are more interesting on the other hand.
        Let \(V\) be a finite dimensional representation.
        We can fix a basis and look at matrices.
        Suppose \(B \in \End V\), then since we work over an algebraically closed field we know that the Jordan normal form of \(B\) exists after a basis change, allowing us to write the matrix of \(B\) as
        \begin{equation}
            B = 
            \begin{pmatrix}
                J_{\lambda_1, n_1} \\
                & J_{\lambda_2, n_2} \\
                & & \ddots \\
                & & & J_{\lambda_k, n_k}
            \end{pmatrix}
        \end{equation}
        where \(J_{\lambda_i, n_i}\) is the \(n_i \times n_i\) Jordan block matrix
        \begin{equation}
            J_{\lambda_i, n_i} = 
            \begin{pmatrix}
                \lambda_i & 1 \\
                & \lambda_i & 1\\
                & & \ddots & \ddots\\
                & & & \lambda_i & 1\\
                & & & & \lambda_i
            \end{pmatrix}
            .
        \end{equation}
        This block diagonal decomposition of \(B\) gives us a corresponding direct sum decomposition of \(V\).
        Each Jordan block cannot be diagonalised (with the exception of the \(1 \times 1\) Jordan blocks which are trivially diagonal).
        Thus we cannot further decompose \(B\) and so we cannot further decompose \(V\).
        The result is that
        \begin{equation}
            V = \bigoplus_{i=1}^{k} V_{\lambda_i, n_i}
        \end{equation}
        where \(V_{\lambda_i, n_i} = \field^{n_i}\) is an \(n_i\)-dimensional vector space upon which the action of \(B\) is given by \(J_{\lambda_i, n_i}\).
        Then taking \(B = \varphi(x)\) defines a representation of \(\field[x]\) on \(V\), and specifically we have the subrepresentations \(V_{\lambda_i, n_i}\) in which \(x\) acts as the Jordan block \(J_{\lambda_i, n_i}\).
    \end{exm}
    
    \section{Ideals and Quotients}
    \begin{dfn}{Ideals}{}
        Let \(A\) be an algebra.
        A subspace, \(I \subseteq A\), such that \(AI \subseteq I\) is called a \defineindex{left ideal}.
        Similarly if \(IA \subseteq I\) then we call \(I\) a \defineindex{right ideal}.
        A \defineindex{two-sided ideal} is simultaneously a left and right ideal.
    \end{dfn}
    
    Note that by \(AI\) we mean \(AI = \{a i \mid a \in A, i \in I\}\), so the condition that \(I\) is a left ideal is that \(ai \in I\) for all \(a \in A\) and \(i \in I\).
    
    \begin{exm}{}{}
        \begin{itemize}
            \item Any algebra, \(A\), always has \(0\) and \(A\) as ideals.
            If these are the only ideals then we call \(A\) \defineindex{simple}.
            \item Any left (right) ideal is a submodule of the left (right) regular representation.
            This is simply identifying that \(A\) is an \(A\)-module with the action being left (right) multiplication and as such the notion of an ideal coincides with that of a submodule.
            Note that the notion of a simple module coincides with the notion of a simple algebra under this identification.
            \item If \(f \colon A \to B\) is an algebra morphism then \(\ker f\) is a two-sided ideal.
            We know that \(\ker f\) is a subspace of \(A\), so just note that if \(a \in \ker f\) then \(f(a) = 0\) and we have
            \begin{equation}
                f(ba) = f(b)f(a) = f(b)0 = 0
            \end{equation}
            and
            \begin{equation}
                f(ab) = f(a)f(b) = 0f(a) = 0
            \end{equation}
            so \(ab\) and \(ba\) are in \(\ker f\).
        \end{itemize}
    \end{exm}
    
    We will say \enquote{ideal} when we mean either a left ideal.
    Note that in the commutative case all left ideals are right ideals and hence two-sided ideals, so we don't need to distinguish the three cases.
    
    \begin{ntn}{}{}
        Let \(A\) be an algebra and \(S \subseteq A\) a subset of \(A\).
        Denote by \(\langle S \rangle\) the two-sided ideal generated by \(S\).
        That is,
        \begin{equation}
            \langle S \rangle = \Span\{asb \mid s \in S, \text{ and } a, b \in A\}.
        \end{equation}
    \end{ntn}
    
    For example, consider \(\field[x]\).
    Then \(\langle x \rangle\) consists of all polynomials that can be factorised as \(xf(x)\) where \(f(x)\) is an arbitrary polynomial, so \(f(x) = \sum_{i=0}^n a_i x^i\).
    Thus, \(x f(x) = \sum_{i=0} a_i x^{i + 1}\), so \(\langle x \rangle\) consists of all polynomials with zero constant term.
    More generally, \(\rangle x - a \rangle\) for \(a \in \field\) consists of all polynomials which factorise as \((x - a)f(x)\) for an arbitrary polynomial \(f(x)\), and thus this is the ideal consisting of all polynomials with \(a\) as a root.
    
    The point of defining ideals is really in order to define quotients.
    In this way ideals are to algebras as normal subgroups are to groups.
    
    \begin{dfn}{Quotient}{}
        Let \(A\) be an algebra and \(I \subseteq A\) an ideal.
        We define the \define{quotient}\index{quotient!algebra} to be the algebra \(A/I\) whose elements are equivalence classes
        \begin{equation}
            [a] = a + I \coloneq \{a' \in A \mid a - a' \in I\}.
        \end{equation}
        Addition and scalar multiplication are defined by
        \begin{equation}
            [a] + [b] = (a + I) + (b + I) = [a + b] = a + b + I
        \end{equation}
        and
        \begin{equation}
            \lambda[a] = [\lambda a]
        \end{equation}
        for \(a, b \in A\) and \(\lambda \in \field\).
    \end{dfn}
    
    \begin{lma}{}{}
        The quotient of an algebra by an ideal is again an algebra.
        \begin{proof}
            Let \(A\) be an algebra and \(I \subseteq A\) an ideal.
            Note that the quotient of a vector space by any subspace is again a vector space, so we need only define a multiplication operation on this vector space.
            We do so by defining
            \begin{equation}
                [a][b] = (a + I)(b + I) \coloneq [ab] = ab + I.
            \end{equation}
            We need to show that this is well-defined and satisfies the properties of multiplication in an algebra.
            
            \Step{Well-Defined}
            Let \(a, a' \in A\) be representatives of the same equivalence class, \([a] = [a']\).
            Then by definition \(a - a' \in I\).
            For \(b \in A\) we then have
            \begin{equation}
                [a][b] = [ab] = [a'b + (a - a')b] = [a'b] = [a'][b].
            \end{equation}
            Here we've used the fact that \(a - a' \in I\) and \(I\) is an ideal so \((a - a')b \in I\), and we can add any element of \(I\) inside an equivalence class without leaving the equivalence class.
            Similarly, one can show that \([a][b] = [a][b']\) whenever \([b] = [b']\).
            Thus, this product is well-defined.
            
            \Step{Algebra}
            Linearity in the first argument follows from a direct calculation using the properties of quotient spaces:
            \begin{multline}
                [(a + \lambda a')b] = [a b + \lambda a' b] = [ab] + \lambda [a' b]\\
                = [a][b] + \lambda [a'][b]= ([a] + \lambda[a'])[b] = [a + \lambda a'][b]
            \end{multline}
            for \(a, a', b \in A\) and \(\lambda \in \field\).
            Linearity in the second argument follows similarly.
            Associativity follows from
            \begin{equation}
                [a]([b][c]) = [a][bc] = [a(bc)] = [(ab)c] = [ab][c] = ([a][b])[c].
            \end{equation}
            Unitality follows from
            \begin{equation}
                [1][a] = [1a] = [a], \qqand [a][1] = [a1] = [a].
            \end{equation}
        \end{proof}
    \end{lma}
    
    \subsection{Generators and Relations}
    One of the most common ways to define an algebra is as a quotient of another algebra by some ideal given in terms of generators.
    The most common starting place is the free algebra, \(\field\langle x_1, \dotsc, x_m \rangle\).
    We can then take \(f_1, \dotsc, f_n \in \field\langle x_1, \dotsc, x_m\rangle\), and form an ideal, \(\langle f_1, \dotsc, f_n \rangle\).
    Then we may form the algebra
    \begin{equation}
        A = \field\langle x_1, \dotsc, x_m \rangle / \langle f_1, \dotsc, f_n \rangle.
    \end{equation}
    Intuitively, elements of this are non-commutative polynomials in the \(x_i\) subject to the constraint that anywhere that we can manipulate the polynomial to be written with \(f_i\) we can set that \(f_i\) equal to zero.
    
    For example, let \(f_{i,j} = x_i x_j - x_j x_i\) for \(i, j = 1, \dotsc, m\).
    Consider the algebra \(A = \field \langle x_1, \dotsc, x_m \rangle / \langle f_{i,j} \rangle\) consists of non-commutative polynomials in \(x_i\) subject to the condition that \(x_i x_j - x_j x_i = 0\), which is to say \(x_i x_j = x_j x_i\), which is exactly the condition that the \(x_i\) \emph{do} commute with each other.
    
    Another example is \(A = \field \langle x_1, \dotsc, x_n \rangle / \langle x_i^2 - e, x_ix_{i+1}x_i - x_{i+1}x_ix_{i+1} \rangle\).
    This sets \(x_i^2 = e\) and \(x_ix_{i+1}x_i = x_{i+1}x_ix_{i+1}\) (called the \defineindex{braid relation}).
    These are exactly the relations defining the symmetric group, \(S_n\), when we interpret \(x_i\) as the transposition \(\cycle{i,i+1}\).
    We're also taking linear combinations of these \(x_i\), so \(A = \field S_n\).
    
    \subsection{Quotient Modules}
    \begin{dfn}{Quotient Module}{}
        Let \(M\) be an \(A\)-module and \(N\) a submodule of \(M\).
        We define the \define{quotient module}\index{quotient!module}, \(M/N\), to be the module consisting of equivalence classes
        \begin{equation}
            [m] = m + N \coloneq \{m' \in M \mid m - m' \in M\}.
        \end{equation}
        Addition in this module is defined by
        \begin{equation}
            [m] + [m'] = [m + m']
        \end{equation}
        for \(m, m' \in M\) and the action of \(A\) is given by
        \begin{equation}
            a \action [m] = [a \action m]
        \end{equation}
        for \(a \in A\) and \(m \in M\).
    \end{dfn}
    
    \begin{lma}{}{}
        The quotient of a module by a submodule is again a module.
        \begin{proof}
            Let \(M\) be an \(A\)-module with \(N \subseteq M\) a submodule.
            Then \(N\) is a subgroup of an abelian group, and so is automatically a normal subgroup.
            Then we know that \(M/N\) is an abelian group also.
            
            Suppose that \([m] = [m']\), that is \(m\) and \(m'\) are representatives of the same equivalence class.
            Then \(m' - m \in N\).
            We then have
            \begin{multline}
                a \action [m] = a \action [m' + (m - m')] = [a \action (m' + (m - m'))]\\
                = [a \action m' + a \action (m - m')] = [a \action m'] = a \action [m'].
            \end{multline}
            Here we've used the fact that \(m' - m \in N\) and \(N\) is a submodule so \(a \action (m' - m) \in N\) as well.
            So, the action of \(a \in A\) on \([m] = [m']\) is well-defined.
            
            It remains to show that the action of \(A\) on \(M/N\) makes it an \(A\)-module:
            \begin{itemize}
                \item[M1] \((ab) \action [m] = [(ab) \action m] = [a \action (b \action m)] = a \action [b \action m] = a \action (b \action [m])\);
                \item[M2] \(1 \action [m] = [1 \action m] = [m]\);
                \item[M3] \(a \action ([m] + [n]) = a \action [m + n] = [a \action (m + n)] = [a \action m + a \action n] = [a \action m] + [a \action n] = a \action [m] + a \action [n]\);
                \item[M4] \((a + b) \action [m] = [(a + b) \action m] = [a \action m + b \action m] = [a \action m] + [b \action m] = a \action [m] + b \action [m]\)
            \end{itemize}
            for all \(a, b \in A\) and \(m, n \in M\).
        \end{proof}
    \end{lma}
    
    \begin{remark}{}{}
        Consider the left regular representation of \(A\).
        As we have mentioned ideals of \(A\) are precisely submodules of the regular representation.
        It follows that \(A/I\) is a left \(A\)-module precisely when \(I\) is a left ideal.
    \end{remark}
    
    \chapter{Tensor Products}
    \section{Tensor Product of Modules}
    We first define the tensor product of \(R\)-modules (\(R\) a ring). 
    This definition can also be applied to \(A\)-modules (\(A\) an algebra) without modification.
    
    \begin{dfn}{Tensor Product}{}
        Let \(R\) be a ring, \(M\) a right \(R\)-module, and \(N\) a left \(R\)-module.
        Then the \define{tensor product}\index{tensor product!of R-modules@of \(R\)-modules}, \(M \otimes_R N\), is the abelian group \begin{equation}
            \frac{F(\{m \otimes n \mid m \in M, n \in N\})}{I}
        \end{equation}
        where \(F(X)\) denotes the free abelian group on the set \(X\) and \(I\) is the normal subgroup generated from all elements of the form
        \begin{itemize}
            \item \((m + m') \otimes n - m \otimes n - m' \otimes n\);
            \item \(m \otimes (n + n') - m \otimes n - m \otimes n'\);
            \item \((m \action r) \otimes n - m \otimes (r \action n)\)
        \end{itemize}
        with \(m, m' \in M\), \(n, n' \in N\) and \(r \in R\).
    \end{dfn}
    
    \begin{wrn}
        The tensor product does not, in general, have the structure of an \(R\)-module.
        It is just an abelian group.
        In a sense the \(R\)-actions of \(M\) and \(N\) are \enquote{used up} in the construction and don't \enquote{survive} to produce a sensible notion of an \(R\)-action on \(M \otimes_R N\).
    \end{wrn}
    
    \begin{ntn}{}{}
        When \(R\) is clear from context we will write \(M \otimes N\) instead of \(M \otimes_R N\).
        Conversely, if needed we'll write \(m \otimes_R n\) for elements of \(M \otimes_R N\) if there are multiple ways to define the tensor product.
    \end{ntn}
    
    Intuitively, \(M \otimes_R N\) consists of sums of elements which we write as\footnote{We should write \([m \otimes n]\) or something similar, since what we actually have is the equivalence class of \(m \otimes n\) in \(F(\{m \otimes n\})/I\).} \(m \otimes n\) with \(m \in M\) and \(n \in N\).
    So, one element of \(M \otimes_R N\) might be
    \begin{equation}
        m_1 \otimes n_1 + m_2 \otimes n_2 + m_3 \otimes n_3
    \end{equation}
    with \(m_i \in M\) and \(n_i \in N\).
    Note that there are no factors of \(R\) here, this is purely an operation in the free group.
    The quotient imposes that in \(M \otimes_R N\) we have the relations
    \begin{align}
        (m + m') \otimes n &= m \otimes n + m' \otimes n;\\
        m \otimes (n + n') &= m \otimes n + m \otimes n';\\
        (m \action r) \otimes n &= m \otimes (r \action n).
    \end{align}
    
    As we mentioned the tensor product of a right and left \(R\)-module is not, in general, an \(R\)-module in any consistent way.
    In order for the tensor product to be a module we need to have some extra module structure present in one of the two modules which then remains after the tensor product is formed.
    Of course, this extra structure must be compatible with the existing structure, and it turns out that the following is exactly the right definition for this purpose.
    
    \begin{dfn}{Bimodule}{}
        Left \(A\) and \(B\) be associative unital \(\field\)-algebras.
        An \define{\(\symbf{(A, B)}\)-bimodule}\index{bimodule} is an abelian group, \(M\), which is both a left \(A\)-module and a right \(B\) module in such a way that
        \begin{equation}
            (a \action m) \action b = a \action (m \action b)
        \end{equation}
        for all \(a \in A\), \(b \in B\), and \(m \in M\).
    \end{dfn}
    
    \begin{exm}{}{}
        Let \(V\) be a \(\field\)-vector space and a left \(A\)-module.
        Then \(V\) is an \((A, \field)\)-bimodule where \(a \action v\) is just the action of \(A\) on \(V\) as an \(A\)-module and \(v \action \lambda = \lambda v\) is just scalar multiplication by elements of \(\field\).
        That this is a bimodule follows because
        \begin{equation}
            a \action (v \action \lambda) = a \action (\lambda v) = \lambda (a \action v) = (a \action v) \action \lambda
        \end{equation}
        having used the fact that the action of \(a\) on \(v\) is \(\field\)-linear.
        
        In fact, we can define a bimodule first (just combining the definitions of a left and right module), then a left \(A\)-module is an \((A, \field)\)-bimodule, and a right \(A\)-module is a \((\field, A)\)-bimodule.
    \end{exm}
    
    \begin{lma}{}{}
        Let \(M\) be an \((A, B)\)-bimodule, and \(N\) a left \(B\)-module.
        Then \(M \otimes_B N\) is a left \(A\)-module with \(a \action (m \otimes n) \coloneqq (a \action m) \otimes n\).
        \begin{proof}
            First note that as an \((A, B)\)-bimodule \(M\) is, in particular, a right \(B\)-module.
            Thus, the tensor product \(M \otimes_B N\) is defined as the quotient of a free abelian group by an ideal, and so is again an abelian group.
            It remains only to show that this abelian group equipped with the action of \(A\) on the first factor is an \(A\)-module.
            
            To do so take an arbitrary element of \(M \otimes_B N\), which is of the form \(\sum_{i \in I} m_i \otimes n_i\) where \(I\) is some finite indexing set, \(m_i \in M\) and \(n_i \in N\).
            We are free to define the action of \(A\) on this element to be
            \begin{equation}
                a \action \left( {\textstyle \sum_{i \in I}} m_i \otimes n_i \right) \coloneqq {\textstyle \sum_{i \in I}} (a \action m_i) \otimes n_i.
            \end{equation}
            Then when \(I\) is a singleton this reduces to \(a \action (m \otimes n) = (a \action m) \otimes n\) as required.
            
            We can now prove that this makes \(M \otimes_B N\) a left \(A\)-module:
            \begin{itemize}
                \item[M1] \((ab) \action \sum_{i} m_i \otimes n_i = \sum_{i} ((ab) \action m_i) \otimes n_i = \sum_{i} (a \action (b \action m_i)) \otimes n_i = a \action \sum_i (b \action m_i) \otimes n_i = a \action \left( b \action \sum_i m_i \otimes n_i \right)\);
                \item[M2] \(1 \action \sum_i m_i \otimes n_i = \sum_i (1 \action m_i) \otimes n_i = \sum_i m_i \action n_i\);
                \item[M3] \(a \action \left( \sum_{i \in I} m_i \otimes n_i + \sum_{j \in J} m_j \otimes n_j \right) = a \action \left( \sum_{i \in I \sqcup J} m_i \otimes n_i \right) = \sum_{i \in I \sqcup J} (a \action m_i) \otimes n_i = \sum_{i \in I} (a \action m_i) \otimes n_i + \sum_{j \in J} (a \action m_j) \otimes n_j\);
                \item[M4] \((a + b) \action \sum_i m_i \otimes n_i = \sum_i ((a + b) \action m_i) \otimes n_i = \sum_i (a \action m_i + b \action m_i) \otimes n_i = \sum_i (a \action m_i) \otimes n_i + (b \action m_i) \otimes n_i = a \action \sum_i m_i \otimes n_i + b \action \sum_i m_i \otimes n_i\). 
            \end{itemize}
        \end{proof}
    \end{lma}
    
    Similarly, if \(M\) is a right \(A\)-module and \(N\) is an \((A, B)\)-bimodule then \(M \otimes_A N\) is a right \(B\)-module with the action given by \((m \otimes n) \action b = m \otimes (n \action b)\).
    
    \begin{exm}{}{}
        Any \(\field\)-vector space, \(V\), is a \((\field, \field)\)-bimodule, defining \(\lambda \action v = \lambda v = v \action \lambda\) for \(\lambda \in \field\) and \(v \in V\).
        If \(U\) is some other vector space then we can form the \(\field\)-module \(V \otimes_{\field} U\), which is of course just the usual tensor product of vector spaces.
        
        In fact, this works for any commutative algebra, \(A\), we can take any \(A\)-module as an \((A, A)\)-bimodule, so if \(M\) and \(N\) are \(A\)-modules then \(M \otimes_A N\) is an \(A\)-module.
    \end{exm}
    
    \subsection{Universal Property}
    The tensor product may also be defined via a universal property.
    
    \begin{lma}{}{}
        Let \(M\) be an right \(A\)-module, and let \(N\) be a left \(A\)-module.
        Then for any abelian group, \(G\), and any group homomorphism \(f \colon M \times N \to G\) satisfying ... there is a unique group homomorhpism \(\overbar{f} \colon M \otimes_A N \to G\) such that \(\overbar{f}(m \otimes n) = f(m, n)\) for all \(m \in M\) and \(n \in N\).
        That is, the diagram
        \begin{equation}
            \begin{tikzcd}
                M \times N \arrow[r, "{-}\otimes{-}"] \arrow[dr, "f"'] & M \otimes_A N \arrow[d, "\exists ! \overbar{f}"]\\
                & G
            \end{tikzcd}
        \end{equation}
        commutes.
        \begin{proof}
            To make this diagram commutes we can define \(\overbar{f}(m \otimes n) = f(m, n)\).
            The fact that \(\overbar{f}\) is a group homomorhpism means that this uniquely defines the value of \(\overbar{f}\) on any element of \(M \otimes_A N\) by
            \begin{equation}
                \overbar{f}\left( {\textstyle \sum_i} m_i \otimes n_i \right) = {\textstyle\sum_i} f(m_i, n_i). \qedhere
            \end{equation}
        \end{proof}
    \end{lma}
    
    Note that \(\Hom_A(M, N)\) inherits the module structure of \(N\) via pointwise operations.
    Let \(M\) be an \((A, B)\)-bimodule, \(N\) a \((B, C)\)-bimodule, and \(P\) an \((A, C)\)-bimodule for three algebras, \(A\), \(B\), and \(C\).
    Then we can form the tensor product \(M \otimes_B N\), which is an \(A\)-module, and we can consider the hom-set \(\Hom_A(M \otimes_B N, P)\), of left \(A\)-module homomorphisms, this is itself an \(A\)-module, and in fact is an \((A, A)\)-bimodule.
    We can also form the hom-set \(\Hom_C(N, P)\) of right \(C\)-module homomorhpisms, which is an left \(A\)-module under pointwise action using the \(A\)-module structure of \(P\).
    Then we can take the hom-set \(\Hom_B(M, \Hom_C(N, P))\), which is an \(A\)-module under pointwise the action.
    Then it turns out that we actually have an isomorphism
    \begin{equation}
        \Hom_A(M \otimes_B N, P) \xrightarrow{\isomorphic} \Hom_B(M, \Hom_C(N, P))
    \end{equation}
    given by sending \(f\) to \(g\) defined by \(g(m)(n) = f(m \otimes n)\).
    This isomorphism is natural in all objects, and thus this is an adjunction.
    
    \section{Tensor Algebra}
    \begin{dfn}{Tensor Algebra}{}
        Let \(V\) be a vector space over \(\field\).
        Then the \defineindex{tensor algebra}, \(TV\), is defined to be
        \begin{equation}
            \bigoplus_{n=0}^{\infty} V^{\otimes n} = \field \oplus V \oplus (V \otimes V) \oplus (V \otimes V \otimes V) \oplus \dotsb.
        \end{equation}
        Multiplication is defined by \(ab = a \otimes b \in V^{\otimes(n = m)}\) for \(a \in V^{\otimes n}\) and \(b \in V^{\otimes m}\), and extended linearly.
    \end{dfn}
    
    \begin{lma}{}{}
        Let \(V\) be an \(n\)-dimensional vector space over \(\field\).
        Then \(TV\) is isomorphic to \(\field\langle x_1, \dotsc, x_n \rangle\), the free algebra on \(n\) indeterminates.
        \begin{proof}
            Pick a basis for \(V\).
            Identify this basis with the \(x_i\).
            Elements of \(TV\) are linear combinations of tensor products of these basis elements, so we can identify them with polynomials in non-commuting variables.
            For example, given the basis \(\{e_i\}\) for \(V\) we have that \(e_1 \otimes e_2 \otimes e_1\) maps to \(x_1x_2x_1\), and \(e_1 \otimes e_2 + e_1 \otimes e_3 \otimes e_2\) maps to \(x_1x_2 + x_1x_3x_2\).
        \end{proof}
    \end{lma}
    
    The nice thing about the tensor algebra is that it gives us a basis free way to work with the free algebra, that is a way that is independent of the choice of generators.
    As it is there is no commutativity imposed on the product in \(TV\), we can impose some commutativity condition by taking quotients.
    
    \begin{dfn}{Quotients of the Tensor Algebra}{}
        Let \(V\) be a vector space over \(\field\).
        We define following quotients:
        \begin{itemize}
            \item \(SV \coloneqq TV/\langle v \otimes w - w \otimes v \rangle\), the \defineindex{symmetric algebra}; and
            \item \(\Lambda V \coloneqq TV/\langle v \otimes w + w \otimes v \rangle\), the \defineindex{exterior algebra}.
        \end{itemize}
        If \(\lie{g} = V\) is a Lie algebra then we may define the quotient \(\universalEnveloping(\lie{g}) \coloneqq TV/\langle v \otimes w - w \otimes v - \bracket{v}{w} \rangle\), the \defineindex{universal enveloping algebra}.
    \end{dfn}
    
    The idea is that for \(SV\) we impose that\footnote{identifying elements with their equivalence class} \(v \otimes w = w \otimes v\), which makes \(SV\) isomorphic to \(\field[x_1, \dotsc, x_n]\) for \(n = \dim V\).
    For \(\Lambda V\) we impose that \(v \otimes w = -w\otimes v\) (usually the product here is written as \(v \wedge w\)).
    Finally, for \(\universalEnveloping(\lie{g})\) we impose that the bracket, \(\bracket{v}{w}\) is exactly the commutator \(v \otimes w - w \otimes v\).
    This last case is nice because it allows us to treat the abstract bracket as if it were a commutator.
    
    Note that the tensor algebra, as well as the quotients \(SV\) and \(\Lambda V\), are graded algebras, meaning that they have decompositions as direct sums:
    \begin{equation}
        SV = \bigoplus_{n = 0}^{\infty} S^nV, \qqand \Lambda V = \bigoplus_{n = 0}^{\infty} \Lambda^n V.
    \end{equation}
    Here \(S^nV\) (\(\Lambda^nV\)) is the \(n\)th (anti)symmetric tensor power of \(V\), that is, it's \(V^{\otimes n}\) modulo the relation that factors (anti)commute.
    Note that \(S^nV\) is isomorphic to the subalgebra of \(\field[x_1, \dotsc, x_n]\) consisting of homogeneous polynomials of degree \(n\).
    
    \chapter{Jacobson's Density Theorem}
    \section{Semisimple Representations}
    Recall that a module is semisimple if it is a direct sum of simple modules, and a simple module is one with no nontrivial submodules.
    
    \begin{exm}{}{}
        Let \(V\) be an \(n\)-dimensional simple \(A\)-module.
        Then \(\End V\) is an \(A\)-module as well, with \(A\) acting by left matrix multiplication (after fixing some basis so that elements of \(\End V\) can be identified with matrices and then identifying elements of \(A\) acting on \(\End V\) with the corresponding linear operator on \(V\)).
        With this construction \(\End V\) is semisimple, in particular
        \begin{equation}
            \End V \isomorphic \underbrace{V \oplus \dotsb \oplus V}_{n \text{ terms}} \eqcolon nV.
        \end{equation}
        This isomorphism is given by fixing some basis, \(\{v_1, \dotsc, v_n\} \subseteq V\), and then defining a linear map \(\End V \to nV\) by \(\varphi \mapsto (\varphi(v_1), \dotsc, \varphi(v_n))\).
        Viewing \(v_i\) as column matrices \(\varphi(v_i)\) is simply the \(i\)th column of the matrix corresponding to \(\varphi\) in this basis.
    \end{exm}
    
    In this example \(\End V\) ends up being a direct sum of a single simple module.
    In the general semisimple case any simple module can appear in the decomposition.
    If we restrict ourselves to finite dimensions then we can get a pretty good handle on which simple modules appear in such a decomposition.
    In particular, any finite-dimensional semisimple module, \(V\), may be decomposed as
    \begin{equation}
        V = \bigoplus_{i \in I} m_i V_i
    \end{equation}
    with \(m_i \in \integers_{\ge 0}\) and \(V_i\) running over all finite dimensional simple modules.
    We call \(m_i\) the \defineindex{multiplicity} of \(V_i\) in \(V\).
    Note that since this decomposition is unique up to the order of the terms.
    
    \begin{lma}{}{}
        Let \(V\) be a finite dimensional semisimple \(A\)-module, with decomposition
        \begin{equation}
            V = \bigoplus_{i \in I} m_i V_i
        \end{equation}
        with \(m_i \in \integers_{\ge 0}\) and \(V_i\) simple.
        Then the multiplicity, \(m_i\), is given by
        \begin{equation}
            m_i = \dim( \Hom_A(V_i, V) ).
        \end{equation}
        \begin{proof}
            We make use of the fact that\footnote{\(\Hom(V_i, -)\) is right adjoint (to \(-\otimes_AV_i\)) and as such preserves colimits}
            \begin{equation}
                \Hom_A(V_i, V' \oplus V'') \isomorphic \Hom_A(V_i, V') \oplus \Hom_A(V_i, V'').
            \end{equation}
            This extends to all finite direct sums.
            
            Note that \(\Hom_A(V_i, V)\) is an \((A, \field)\)-bimodule with the left action \((a \action \varphi)(v) = \varphi(a \action v)\) and right action \((\varphi \action \lambda)(v) = \lambda \varphi(v)\).
            Further, \(V_i\) is a right \(\field\)-module with the action \(v \action \lambda = \lambda v = (\lambda 1_A) \action v\).
            Thus, \(\Hom_A(V_i, V) \otimes_{\field} V_i\) is a left \(A\)-module.
            
            We can define a map
            \begin{equation}
                \label{eqn:map between + hom Vi V x Vi and V}
                \begin{aligned}
                    \psi \colon \bigoplus_{i \in I} \Hom_A(V_i, V) \otimes_{\field} V_i &\to V\\
                    \bigoplus_{i \in I} \varphi_i \otimes v_i &\mapsto \sum_i \varphi_i(v_i).
                \end{aligned}
            \end{equation}
            This is an \(A\)-module isomorphism:
            \begin{align}
                \psi\left( a \action {\textstyle \bigoplus_{i \in I}} \varphi_i \otimes v_i \right) &= \psi\left( {\textstyle \bigoplus_{i\in I}} \varphi_i \otimes (a \action v_i) \right)\\
                &= {\textstyle \sum_{i \in I}} \varphi_i(a \action v_i)\\
                &= {\textstyle \sum_{i \in I}} a \action \varphi_i(v_i)\\
                &= a \action {\textstyle \sum_{i \in I}} \varphi_i(v_i)\\
                &= a \action \psi\left( {\textstyle \bigoplus_{i \in I}} \varphi_i \otimes v_i \right).
            \end{align}
            Linearity is clear from the definition.
            It remains only to show that this map is invertible.
            By linearity it is sufficient to show that the map
            \begin{align}
                \Hom(V_i, V) \otimes V_i &\to V\\
                \varphi_i \otimes v_i &\mapsto \varphi_i(v_i)
            \end{align}
            is an isomorphism.
            Since \(V_i\) is simple Schur's lemma tells us that this map is either zero or surjective.
            It is clearly not zero, since we can simply choose some vector \(v_i\) and some nonzero map \(\varphi_i\) on which \(\varphi_i(v_i) \ne 0\).
            Thus, this map is surjective.
            A surjective linear map between finite dimensional modules is an isomorphism.
            Hence, the map in \cref{eqn:map between + hom Vi V x Vi and V} is an isomorphism.
            
            We then have
            \begin{align}
                \dim V &= \dim\left( {\textstyle \bigoplus_{i \in I}} \Hom_A(V_i, V) \right)\\
                &= {\textstyle \sum_{i \in I}} \dim(\Hom_A(V_i, V)) \dim(V_i)
            \end{align}
            and
            \begin{align}
                \dim V &= \dim\left( {\textstyle \bigoplus_{i \in I}} m_i V_i \right)\\
                &= {\textstyle \sum_{i \in I}} m_i \dim (V_i).
            \end{align}
            Since these are finite sums and this must hold for arbitrary semisimple modules \(V\), including the case where \(V = V_i\) is actually simple, we must have that
            \begin{equation*}
                m_i = \dim(\Hom_A(V_i)). \qedhere
            \end{equation*}
        \end{proof}
    \end{lma}
    
    
    % Appdendix
%	\appendixpage
%	\begin{appendices}
%	
%	\end{appendices}

    \backmatter
    \renewcommand{\glossaryname}{Acronyms}
    \printglossary[acronym]
    \printindex
\end{document}