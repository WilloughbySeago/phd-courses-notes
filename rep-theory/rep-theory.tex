% !TeX program = lualatex
\documentclass[fleqn]{NotesClass}

\strictpagecheck

\usepackage{csquotes}

\usepackage{tikz}
\usetikzlibrary{external}
\tikzexternalize[prefix=tikz-external/]

\usepackage{tikz-cd}
\AtBeginEnvironment{tikzcd}{\tikzexternaldisable}
\AtEndEnvironment{tikzcd}{\tikzexternalenable}

\usepackage[pdfauthor={Willoughby Seago},pdftitle={Notes from Representation Theory Course},pdfkeywords={representation theory},pdfsubject={Representation Theory}]{hyperref}  % Should be loaded second last (cleveref last)
\colorlet{hyperrefcolor}{blue!60!black}
\hypersetup{colorlinks=true, linkcolor=hyperrefcolor, urlcolor=hyperrefcolor}
\usepackage[
capitalize,
nameinlink,
noabbrev
]{cleveref} % Should be loaded last

% My packages
\usepackage{NotesBoxes}
\usepackage{NotesMaths2}

\setmathfont[range={\int, \oint, \otimes, \oplus, \bigotimes, \bigoplus}]{Latin Modern Math}


% Highlight colour
%\definecolor{highlight}{HTML}{710D78}
%\definecolor{my blue}{HTML}{2A0D77}
%\definecolor{my red}{HTML}{770D38}
%\definecolor{my green}{HTML}{14770D}
%\definecolor{my yellow}{HTML}{E7BB41}

% Title page info
\title{Representation Theory}
\author{Willoughby Seago}
\date{January 13th, 2024}
\subtitle{Notes from}
\subsubtitle{University of Glasgow}
\renewcommand{\abstracttext}{These are my notes from the SMSTC course \emph{Lie Theory} taught by Prof Christian Korff. These notes were last updated at \printtime{} on \today{}.}

% Commands
% Maths
\renewcommand{\field}{\symbb{k}}
\newcommand{\id}{\symrm{id}}
\DeclareMathOperator{\End}{End}
\DeclareMathOperator{\Hom}{Hom}
\newcommand{\action}{\mathbin{.}}
\newcommand{\op}{\symrm{op}}

\begin{document}
    \frontmatter
    \titlepage
    \innertitlepage{}
    \tableofcontents
    % \listoffigures
    \mainmatter
    \chapter{Introduction}
    We fix some standard notation here:
    \begin{itemize}
        \item \(\field\) will denote an algebraically closed field, except for when we explicitly mention that the field needn't be algebraically closed.
        \item \(A\) will denote an associative unital algebra.
        \item Letters like \(V\), \(U\), and \(W\) will denote vector spaces over \(\field\).
        \item Letters like \(M\) and \(N\) will denote modules.
    \end{itemize}
    
    \chapter{Initial Definitions}
    \section{Algebra}
    \begin{dfn}{Algebra}{}
        An \defineindex{algebra} is a \(\field\)-vector space, \(A\), equipped with a bilinear map,
        \begin{align}
            m \colon A \times A &\to A\\
            (a, b) &\mapsto m(a, b) = ab.
        \end{align}
        
        If this map satisfies the condition that
        \begin{equation}
            m(a, m(b, c)) = m(m(a, b), c), \text{ or equivalently } a(bc) = (ab)c,
        \end{equation}
        for all \(a, b, c \in A\) then we call \(A\) an \defineindex{associative algebra}.
        
        If \(A\) posses a distinguished element, \(1 \in A\), such that \(m(1, a) = a = m(a, 1)\), or equivalently \(1a = a = a1\) for all \(a \in A\) then we say that \(A\) is a \defineindex{unital algebra}.
        
        If \(m(a, b) = m(b, a)\), or equivalently \(ab = ba\), for all \(a, b \in A\) then we say that \(A\) is a \defineindex{commutative algebra}.
    \end{dfn}
    
    Whenever we say, otherwise unqualified, \enquote{algebra} we will mean associative unital algebra unless we specify otherwise.
    We will not assume commutativity of a general algebra.
    
    The condition of associativity can be written as a commutative diagram,
    \begin{equation}
        \begin{tikzcd}
            A \times A \times A \arrow[r, "m \times \id_A"] \arrow[d, "\id_A \times m"'] & A \times A \arrow[d, "m"]\\
            A \times A \arrow[r, "m"'] & A\mathrlap{,}
        \end{tikzcd}
    \end{equation}
    
    \begin{remark}{}{}
        This diagram goes part of the way to the more abstract definition that \enquote{an associative unital (commutative) algebra is a (commutative) monoid in the category of vector spaces}.
        This definition is nice because it is both very general and dualises to the notion of a coalgebra.
        See the \textit{Hopf Algebra} notes for more details.
    \end{remark}
    
    \begin{exm}{}{}
        \begin{itemize}
            \item \(A = \field\) is an algebra with the product given by the product in the field;
            \item \(A = \field[x_1, \dotsc, x_n]\), the ring of polynomials in the variables \(x_i\) with coefficients in \(\field\), is an algebra under the addition and multiplication of polynomials.
            \item \(A = \field \langle x_1, \dotsc, x_n \rangle\), the free algebra on \(x_i\), may be considered as the algebra of polynomials in non-commuting variables, \(x_i\).
            \item \(A = \End V\) for \(V\) a \(\field\)-vector space is an algebra with multiplication given by composition of morphisms.
        \end{itemize}
    \end{exm}
    
    \begin{dfn}{Group Algebra}{}
        Let \(G\) be a group.
        The \defineindex{group algebra} or \defineindex{group ring} \(\field G = \field[G]\) is defined to be the set of finite formal linear combinations
        \begin{equation}
            \sum_{g \in G} c_g g
        \end{equation}    
        where \(c_g \in \field\) is nonzero for only finitely many values \(g\).
        Addition is defined by
        \begin{equation}
            \sum_{g \in G} c_g g + \sum_{g \in G} d_g g = \sum_{g \in G} (c_g + d_g) g.
        \end{equation}
        Multiplication is defined by requiring that it distributes over addition and that the product of two terms in the above sums is given by
        \begin{equation}
            (c_g g) (d_h h) = (c_g d_h) (gh)
        \end{equation}
        where multiplication on the left is in \(\field G\), the multiplication \(c_g d_h\) is in \(\field\), and the multiplication \(gh\) is in \(G\).
        
        If we do the same construction replacing \(\field\) with a ring, \(R\), then we get the group ring, \(RG\), which is not an algebra but instead an \(R\)-module.
    \end{dfn}
    
    \begin{dfn}{Algebra Homomorphism}{}
        Let \(A\) and \(B\) be \(\field\)-algebras.
        An \defineindex{algebra homomorphism} is a linear map \(f \colon A \to B\) such that \(f(ab) = f(a)f(b)\) for all \(a, b \in A\).
        
        If \(A\) and \(B\) are unital, with units \(1_A\) and \(1_B\) respectively, then we further require that \(f(1_A) = 1_B\).
        
        We denote by \(\Hom(A, B)\) or \(\Hom_{\field}(A, B)\) the set of all algebra homomorhpisms \(A \to B\).
    \end{dfn}
    
    If \(m_A\) and \(m_B\) denote the multiplication maps of \(A\) and \(B\) respectively then we may think of a homomorphism, \(f\), as a linear map which \enquote{commutes} with the multiplication map, that is \(f \circ m_A = m_B \circ f\).
    
    Alternatively, an algebra, \(A\) is both an abelian group under addition, and a monoid under multiplication, and an algebra homomorhpism is both a group and monoid homomorphism with respect to these structures.
    
    \section{Representations and Modules}
    There are two competing terminologies in the field, with slightly different notation and emphasis depending on which we use.
    We'll use the more modern notion of modules most of the time, but will occasionally and interchangeably use the notion of representations as well.
    
    \begin{dfn}{Representation}{}
        Let \(V\) be a \(\field\)-vector space and \(A\) a \(\field\)-algebra.
        Any \(\rho \in \Hom(A, \End V)\) is called a \defineindex{representation} of \(A\).
        That is, a representation of \(A\) is an algebra homomorphism \(\rho \colon A \to \End V\).
    \end{dfn}
    
    \begin{dfn}{Module}{}
        Let \(A\) be a \(\field\)-algebra.
        A \define{left \(\symbf{A}\)-module}\index{left A-module@left \(A\)-module}, \(M\), is an abelian group, with the binary operation denoted \(+\), equipped with a \defineindex{left action}
        \begin{align}
            \action \colon A \times M &\to M\\
            (a, m) &\mapsto a \action m
        \end{align}
        such that for all \(a, b \in A\) and \(m, n \in M\) we have\footnote{Note that M1 and M2 simply say that this is a group action on the set \(M\), and M3 and M4 two impose that this group action is compatible with both the group operation and addition in the algebra.}
        \begin{itemize}
            \item[M1] \((ab)\action m = a\action (b\action m)\) (note that \((ab)\) is the product in \(A\));
            \item[M2] \(1 \action m = m\).
            \item[M3] \(a\action(m + n) = a\action m + a\action n\);
            \item[M4] \((a + b)\action m = a\action m + b\action m\);
        \end{itemize}
        
        One can similarly define a \define{right \(\symbf{A}\)-module}\index{right A-module@right \(A\)-module}, \(M\), as an abelian group with a \defineindex{right action}
        \begin{align}
            \action \colon M \times A &\to M\\
            (m, a) &\mapsto m \action a
        \end{align}
        such that for all \(a, b \in A\) and \(m, n \in M\) we have
        \begin{itemize}
            \item[M1] \((m + n) \action a = m \action a + n \action a\);
            \item[M2] \(m \action (a + b) = m \action a + m \action b\);
            \item[M3] \(m \action (ab) = (m \action a) \action b\);
            \item[M4] \(m \action 1 = m\).
        \end{itemize}
        
        A \define{two-sided \(\symbf{A}\)-module}\index{two-sided A-module@two-sided \(A\)-module}\index{\(A\)-module}\index{module} is then an abelian group, \(M\), which is simultaneously a left and right \(A\)-module satisfying
        \begin{equation}
            a \action (m \action b) = (a \action m) \action b
        \end{equation}
        for all \(a, b \in A\) and \(m \in M\).
    \end{dfn}
    
    When it doesn't risk confusion we will write \(a \action m\) as \(am\) and \(m \action a\) as \(ma\).
    
    Note that a module is a generalisation of the notion of a vector space.
    In fact, if \(A = \field\) then a module is exactly a vector space.
    
    More compactly, one can define a right \(A\)-module as a left \(A^{\op}\)-module, where \(A^{\op}\) is the \defineindex{opposite algebra} of \(A\), defined to be the same underlying vector space with multiplication \(*\) defined by \(a * b = ba\), where \(ba\) is the multiplication in \(A\).
    Because of this we will almost never have reason to work with right modules, we can always turn them into a left module over the opposite algebra instead.
    
    Note that if \(A\) is commutative every left \(A\)-module is a right \(A\)-module and vice versa, and also a two-sided module.
    
    Without further clarification the term \enquote{module} will mean
    \begin{itemize}
        \item a left module if \(A\) is not necessarily commutative;
        \item a two sided module if \(A\) is commutative.
    \end{itemize}
    
    A representation of \(A\) and an \(A\)-module carry exactly the same information.
    Given a representation, \(\rho \colon A \to \End V\) we may define a group action on \(V\) by \(a \action v = \rho(a)v\).
    Composition in \(\End V\) is exactly repeated application of this action: \([\rho(a)\rho(b)]v = \rho(a)[\rho(b)v]\) (M1).
    The unit of \(\End V\) is the identity morphism, \(\id_V\), and \(1 \in A\) must map to \(\id_V\), so \(\rho(1)v = \id_V v = v\) (M2).
    Linearity of \(\rho(a)\) means that \(\rho(a)(v + w) = \rho(a)v + \rho(v)w\) (M3).
    Linearity of \(\rho\) means that \(\rho(a + b)v = \rho(a)v + \rho(b)v\) (M4).
    
    Conversely, given an \(A\)-module, \(M\), we can define scalar multiplication by \(\lambda \in \field\) on \(M\) by \(\lambda m = (\lambda 1) m\) where \(\lambda 1\) is scalar multiplication in \(A\).
    This makes \(M\) a vector space, and we may define a morphism \(\rho \colon A \to \End M\) by defining \(\rho(a)\) by \(\rho(a) = a \action m\), which uniquely determines \(\rho(a)\), say by considering the action on some fixed basis of \(M\).
    
    Further, these two constructions are inverse, given a module if we construct the corresponding representation then construct the corresponding module from that we get back the original module, and vice versa.
    This means that the notion of a representation and a module really are the same, and we don't need to distinguish between them.
    We will use whichever terminology and notation is better suited to the problem, which is usually the module terminology and notation.
    
    \begin{prp}{}{}
        Let \(V\) be a \(\field\)-vector space, \(G\) a group, and \(\rho \colon G \to GL(V)\) a group homomorphism.
        We may define a \(\field G\)-module by extending this map linearly, defining
        \begin{equation}
            \left( \sum_{g \in G} c_g g \right) \action v = \sum_{g \in G} c_g \rho(g)v.
        \end{equation}
        Conversely, given a left \(\field G\)-module on \(V\) we may define a group homomorphism \(\rho \colon G \to \generalLinear(V)\) by defining \(\rho(g)\) to be the linear operation \(v \mapsto g \action v\).
        \begin{proof}
            This is just a special case of the equivalence of representations and modules discussed above.
        \end{proof}
    \end{prp}
    
    Note that a \defineindex{group representation} is defined to be a group homomorphism \(\rho \colon G \to \generalLinear(V)\).
    The above result shows that a group representation of \(G\) is exactly the same as an algebra representation of \(\field G\), so we can just study algebras.
    
    \begin{dfn}{Regular Representation}{}
        Let \(V = A\) be an algebra and define \(\rho \colon A \to \End A\) by \(\rho(a)b = ab\).
        This is called the \defineindex{left regular representation}.
        Similarly, the \defineindex{right regular representation} is given by defining \(\rho(a)b = ba\).
    \end{dfn}
    
    \section{Direct Sums}
    The goal of much of representation theory is to classify possible representations.
    To do this we usually decompose representations into smaller parts that can be more easily classified.
    This decomposition is done by the direct sum.
    
    \begin{dfn}{Direct Sum}{}
        Let \(M\) and \(N\) be \(A\)-modules.
        The \defineindex{direct sum}, \(M \oplus N\), is the \(A\)-module given by the direct sum of the underlying abelian groups equipped with the action
        \begin{equation}
            a(m \oplus n) = am \oplus an
        \end{equation}
        for all \(a \in A\), \(m \in M\) and \(n \in N\).
    \end{dfn}
    
    The required properties follow immediately from the definition:
    \begin{itemize}
        \item[M1] \((ab)(m \oplus n) = (ab)m \oplus (ab)n = a(bm) \oplus a(bn) = a(bm \oplus bn) = a(b(m \oplus n))\);
        \item[M2] \(1(m \oplus n) = 1m \oplus 1n = m \oplus n\);
        \item[M3] \(a((m \oplus n) + (m' \oplus n')) = a((m + m') \oplus (n + n')) = a(m + m') \oplus a(n + n') = (am + am') \oplus (an + an') = (am \oplus an) + (am' \oplus an') = a(m \oplus n) + a(m' \oplus n')\);
        \item[M4] \((a + b)(m \oplus n) = (a + b)m \oplus (a + b)n = (am + bm) \oplus (an + bn) = (am \oplus an) + (bm \oplus bn) = a(m \oplus n) + b(m \oplus n)\).
    \end{itemize}
    
    % Appdendix
%	\appendixpage
%	\begin{appendices}
%	
%	\end{appendices}

    \backmatter
    \renewcommand{\glossaryname}{Acronyms}
    \printglossary[acronym]
    \printindex
\end{document}