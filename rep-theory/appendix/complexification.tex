\chapter{Complex, Real and Quaternionic Types}
\section{Complexification}
Let \(G\) be a finite group and let \(M\) be a simple \(G\)-module over \(\reals\).
We first consider the complexification, \(M_{\complex} \coloneqq \complex \otimes_{\reals} M\) which is a complex vector space with scalar multiplication defined by \(w(z \otimes m) = (wz) \otimes m\) for \(w, z \in \complex\) and \(m \in M\).
It is convention to simply write \(zm\) for \(z \otimes m\).
This is simply extension of scalars, we can identify \(M_{\complex}\) as \enquote{\(M\) but we define formal multiplication by complex numbers}.

The complexified space, \(M_{\complex}\), is a \(G\)-module still, it just inherits the action of \(G\) on \(M\).
Specifically, \(g \action (z \otimes m) = z \otimes (g \action m)\) for \(g \in G\), \(z \in \complex\), and \(m \in M\).
We usually just write \(g \action zm = z(g \action m)\).

For \(x, y \in \reals\) with \(z = x + iy\) we can write \(z \otimes m\) as
\begin{equation}
    (x + iy) \otimes m = x \otimes m + iy \otimes m = x (1 \otimes m) + y (i \otimes m).
\end{equation}
This lets us decompose \(M_{\complex}\) as
\begin{equation}
    M_{\complex} = (1 \otimes M) \oplus (i \otimes M).
\end{equation}
This is somewhat sloppy notation, we really have \((\reals 1 \otimes_{\reals} M) \oplus (\reals i \otimes_{\reals} M)\), and since \(\reals c \isomorphic \reals\) and \(\reals \otimes M \isomorphic M\) for any real vector space, \(M\), we have that \(M_{\complex} \isomorphic M \oplus M\), where the first copy is thought of as the real part, and the second as the imaginary part.
This gives a second way of thinking of the complexification of \(M\), it's exactly the same process that we apply to get the complex numbers out of the reals.
We'll use the familiar notation of \(\complex\) to write \(m \in M_{\complex}\) as \(a + ib\) when we view it like this.
With this interpretation the action of \enquote{multiplication by \(i\)} is \(a + ib \mapsto -b + ia\).
This can be encoded in the block matrix
\begin{equation}
    J = 
    \begin{pmatrix}
        0 & -\id_V\\
        \id_V & 0
    \end{pmatrix}
    .
\end{equation}
The action of \(g \in G\) on \(M_{\complex}\) when viewed as \(M \oplus M\) is \(g \action (a + ib) = (g \action a) + i(g \action b)\).
This extends to \(g \in \reals G\).
Note that we still consider the real group algebra, and any complex coefficient we might want to include is treated separately by the action of \(x\id_{M_{\complex}} + Jy\).